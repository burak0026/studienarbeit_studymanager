\chapter{Zusammenfassung, Fazit und Ausblick}

\section{Zusammenfassung}
Im Rahmen dieser Arbeit wurde die speziell für den Studiengang Tinf20B3 entwickelte DHBW-Star-Plattform evaluiert. Die Evaluation basierte auf einem Fragebogen, der sich an Kursteilnehmer richtete. Die Umfrage umfasste Fragen zu verschiedenen Aspekten der Plattform, einschließlich  Benutzerfreundlichkeit, Design und  Funktionalität. Die Antworten konnten mithilfe einer Kombination aus quantitativen und qualitativen Techniken analysiert werden, um  Einblicke in die Benutzererfahrung zu gewinnen.\\
 
Umfrageergebnisse zeigen, dass die Mehrheit der Befragten die Plattform  nützlich findet und insbesondere ihr elegantes und modernes Design lobt. Die Linksammlung war die am höchsten bewertete Funktion und zeigt, wie wichtig es den Studierenden ist, einen zentralen Informationspunkt zu haben. Es gab jedoch einige Bereiche, in denen Verbesserungen vorgeschlagen wurden, beispielsweise die Dualis-Integration und benutzerdefinierte Studentenplan.
\newpage
\section{Fazit}
Die Untersuchung der Benutzererfahrung mit DHBW-Star liefert wertvolle Erkenntnisse darüber, wie die Bereitstellung von Informationen der DHBW verbessert werden kann. Es wurde deutlich, dass die Studenten ein besseres Design und mehr Funktionen wollten. Auch die zentrale Bereitstellung von Informationen  wurde als wichtig erachtet. 
\\
Zu den  Forschungsfragen lässt sich folgendes sagen:
\begin{itemize}
	\item [1.] Wie wird DHBW-Star von  Studierenden im Tinf20B3-Kurs genutzt und bewertet?\\
	Die Nutzung von DHBW-Star variiert,  einige Studierende  nutzen es täglich, andere weniger häufig. Der DHBW-Star erhielt insgesamt eine hohe Bewertung, insbesondere für sein minimalistisches Design und die zentrale Gliederkollektion. 
	\item[2.]Welche Verbesserungen und neuen Funktionen wünschen sich die Studierenden?\\
	Die Studierenden äußerten den Wunsch, Dualis in ihre individuellen Planer zu integrieren. Es wird auch darauf hingewiesen, dass es  bei der Darstellungsweise der Lebensmittelkarte Verbesserungspotenzial gibt.
\end{itemize}

\section{Ausblick}
Die Ergebnisse dieser Studienarbeit bilden eine solide Grundlage für zukünftige Verbesserungen des DHBW-Stars und anderer DHBW-Informationsangebote. Konkrete Verbesserungsvorschläge aus der Befragung sollten in zukünftigen Entwicklungsphasen  oder ähnlichen Projekten berücksichtigt werden. 
Es wäre auch sinnvoll, weitere Untersuchungen durchzuführen, um festzustellen, ob die in dieser Studie ermittelten Präferenzen und Bedürfnisse  auf andere Studiengänge und Universitäten anwendbar sind. Darüber hinaus könnte es interessant sein, die langfristigen Auswirkungen von  Änderungen auf die Benutzererfahrung und das Engagement der Studierenden zu untersuchen. 
Insgesamt zeigt diese Untersuchung, wie wichtig es ist, kontinuierlich Benutzerfeedback zu sammeln und auf die Bedürfnisse und Wünsche der Benutzer zu reagieren, um nützliche und effektive Informationsdienste zu schaffen. Die DHBW-Star-Plattform könnte ein wertvolles Werkzeug für DHBW-Studierende sein, und diese Forschung hat dazu beigetragen, den Weg für zukünftige Entwicklungen aufzuzeigen.