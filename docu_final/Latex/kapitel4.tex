\chapter{Umfrage/Datenerhebung}
Um eine Aussage über die Qualität der Webseite treffen zu können, wird eine Umfrage erstellt. In dieser sollen die Studenten, DHBW Star mit den offiziellen Webseiten vergleichen und bewerten.Aus den Ergebnissen sollen Schlussfolgerungen zum Design von DHBW-Star gezogen werden und Vorschläge für Verbesserungen des Informationsangebotes der DHBW folgen.

\section{Vorbereitung}
Um eine aussagekräftige Umfrage erstellen zu können, müssen vorher folgende Überlegung angestellt werden: \\
Was soll durch den Fragebogen gemessen werden? \\
Wer sind die Zielpersonen? \\
Welche Art von Fragen sind dafür geeignet? \\
Erst wenn diese Fragen geklärt sind, kann der Fragebogen erarbeitet werden.

Die ersten beiden Fragen sind von vorneherein klar:
Antwort: Ob DHBW Star den Studenten besser gefällt als die offiziellen Websites.
Zielpersonen sind alle Teilnehmer im Tinf20B3 Kurs, da DHBW-Star für sie entworfen und designt wurde.
Die Entscheidung der Fragen-Art fällt auf eine Mischung von Messskala und Freitext. Es soll erst gemessen werten wie DHBW-Star im Vergleich zu den offiziellen Webseiten gefällt und der Freitext die Möglichkeit geben, zu beschreiben was besser oder schlechter gefallen hat.
Dabei werden alle Elemente der Seite zunächst einzeln Bewertet und danach die gesamte Website.
Zuletzt wird noch nach weiteren Verbesserungen und fehlenden Features gefragt.
Bei der Formulierung der Fragen werden die Checklisten von K. Wolfgang Kallus verwendet.
\cite{fragebogenKallus}(S. 142 + 143)
Diese beschreiben verschieden Kriterien für die Fragen um sicherzustellen, dass sie aussagekräftige Antworten Liefern.

Um die Hürde für die Teilnehme und den Aufwand der Auswertung gering zu halten, wird die Umfrage Online durchgeführt.

\section{Umfrage erstellen}
Um die Umfrage zu erstellen, wird die Webseite \emph{www.surveymonkey.de} verwendet. Die Entscheidung für diese Webseite fiel auf Grund der deutsche Top-Level-Domain und wegen ihrer Benutzerfreundlichkeit und Intuitivität, die es auch Menschen ohne technischen Hintergrund ermöglicht, Umfragen zu erstellen und zu verwalten\cite{surveymonkey-create-surveys}.\\
Die Plattform bietet eine Reihe von Funktionen, die das Erstellen und Verwalten von Umfragen erleichtern. Dazu gehören die Möglichkeit, Fragen zu gruppieren, um die Umfragestruktur klar und logisch zu gestalten, sowie Optionen, erste Umfragefragen so zu gestalten, dass sie die Befragten nicht verunsichern \cite{surveymonkey-create-surveys}.\\
Darüber hinaus bietet die Plattform auch Unterstützung bei der Umfrageauswertung. Es bietet Tools zur Datenanalyse und  Präsentation von Ergebnissen und hilft dabei, die wichtigsten Botschaften aus  Daten zu identifizieren und zu kommunizieren \cite{surveymonkey-auswertung-einer-umfrage}. \\
Schließlich bietet SurveyMonkey auch die Möglichkeit, anonyme Antworten zu aktivieren. Dies ist besonders nützlich, wenn die Vertraulichkeit der Befragten wichtig ist \cite{surveymonkey-anonymous-responses}.\\
Testumfragen zu erstellen und aus erster Hand zu erfahren, wie Umfragen erstellt und ausgewertet werden, ist ebenfalls ein großer Vorteil von SurveyMonkey, da es möglich ist die Umfrage vor der endgültigen Veröffentlichung verfeinern und verbessern zu können. 
Mit dem kostenfreien Angebot können sich Umfragen bis zu zehn Fragen/Items erstellen. Es können allerdings auch maximal nur zehn Antworten ausgewertet werden. Was leider erst im Nachhinein aufgefallen ist.
Nach einem ersten Versuch, wurde das kostenlose Angebot als ausreichend für diese Umfrage befunden.
\\
Zuerst wird eine Testumfrage erstellt um einen ersten Eindruck für die Erstellung und Auswertung der Umfrage zu bekommen.
Danach folgt der endgültige Fragebogen. Er besteht aus den folgenden 10 Fragen:
\begin{itemize}
	\item[01] {\emph{Frage}: Wie oft hast du DHBW Star in den letzten 7 Tagen benutzt?\\
		\emph{Antwortmöglichkeiten}: kein mal, 1 mal, Mehr mal oder Täglich}
	\item[02]{\emph{F}:Auf welchem Gerät hast du DHBW Star benutzt?\\
		\emph{AM}: PC, Tablet und Mobile}
	\item[03]{\emph{F}: Wie nützlich findest du die Startseite?\\
		\emph{AM}: sehr nützlich, etwas nützlich oder unnütz}
	\item[04]{\emph{F}: Wie gefällt dir die Darstellung des Scheduler im Vergleich zum Rapla?\\
		\emph{AM}: viel besser, besser, gleich, schlechter oder viel Schlechter}
	\item[05]{\emph{F}: Was gefällt oder fehlt dir beim Scheduler?\\
		\emph{AM}: Freitext}
	\item[06]{\emph{F}: Wie gefällt dir die Darstellung des Food im Vergleich zum sw-ka.de?\\
		\emph{AM}: viel besser, besser, gleich, schlechter oder viel Schlechter}
	\item[07]{\emph{F}: Was gefällt oder fehlt dir beim Food?\\
		\emph{AM}: Freitext}
	\item[08]{\emph{F}: Wie nützlich findest du die Link-Sammlung?\\
		\emph{AM}: sehr nützlich, etwas nützlich oder unnütz}
	\item[09]{\emph{F}: Wie nützlich findest du DHBW Star?\\
		\emph{AM}: sehr nützlich, etwas nützlich oder unnütz}
	\item[10]{\emph{F}: Was Fehlt dir noch bei DHBW Star?\\
		\emph{AM}: Integration von Dualis, Integration von Moodle,\\ Integration ÖPNV Fahrplan, Umfragen für Kurs Erstellen,\\ Benachrichtigungen für Kurs, Individueller Scheduler (Vorlesungen ein- und ausblenden, eigene Termin hinzufügen), Todo Liste (Hausaufgaben / Abgaben)\\ und Sonstiges (bitte angeben)[ Freitext]}
\end{itemize}
Mit Frage 1 wird die Häufigkeit der Nutzung abgefragt. Dies soll Auskunft darüber geben, ob sie im Testzeitraum eventuell schon als Alternative zum Einsatz gekommen ist.
\\\\
Die zweite Frage soll einen Überblick über die verwendeten Geräte geben. Zudem kann so leichter erkannt werden, falls Probleme mit einem Gerätetype zusammenhängen.
\\\\
Frage 3 bis 7 messen die Qualität der Darstellung von DHBW Star im Verhältnis zu den offiziellen Angeboten. Die Freitex-Felder geben die Möglichkeit, seine Entscheidung zu begründen.
\\\\
Die Frage 8 misst die Nützlichkeit der Link-Sammlung und somit den Wunsch nach einer zentralen Informationsquelle.
\\\\
Die neunte Frage ist eine Art Kontrollfrage, da sie nur die vorherigen Fragen zusammen beantwortet.
\\\\
Zuletzt wird noch nach weiteren Verbesserungen und neuen Funktionen gefragt. Daraus könnten Vorschläge für die DHBW erarbeitet werden oder weitere Arbeiten folgen.
\\
Dadurch dass die Umfrage sehr kurz ist und aus einfachen Fragen besteht, sind automatisch die meisten Kriterien der Checklisten erfüllt. Allerdings gibt es auch Punkte die nicht erfüllt werden. 'Der Einfluss der Vergleichsrichtung beim Vergleich von Objekten' sagt aus, dass bei dem Vergleich zweier Objekte, ersteres stärker präsent ist und auch die Merkmale bestimmt welche verglichen werden. \cite{fragebogenRolf} [S.123]

Um dieses Problem zu umgehen müssten die Objekte in den Fragen vier und sechs den Befragten in zufälliger Reihenfolge gezeigt werden. Dies ist mit dem kostenfreien Angebot von \emph{SurveyMonkey} zwar nicht möglich, jedoch wird dies bei der Auswertung berücksichtigt.

\section{Die Befragten}
Die Umfrage findet nur innerhalb des Tinf20B3 Kurses statt. Dies hat Pragmatische Gründe: DHBW Star ist nur für diesen Kurs ausgelegt. So ist z.B. der \emph{Scheduler} nur für diesem Kurs und kann nicht ohne Änderung an der Software verändert werden.

Dies beschränkt zwar die Anzahl an möglichen Antwortenden, aber da \emph{www.surveymonkey.de} diese auf 10 beschränkt, wäre eine größere Gruppe kein Vorteil.