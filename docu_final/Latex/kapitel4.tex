\chapter{Umfrage/Datenerhebung}
Um eine aussage über die Qualität der Webseite treffen zu können, wird eine Umfrage Erstellt. In dieser sollen die Studenten, DHBW Star mit den den offiziellen Webseiten vergleichen und bewerten.

\subsection{Vorbereitung}
Um ein Aussagekräftige Umfrage erstellen zu können, müssen vorher folgende Überlegung angestellt werden:\\
Was soll durch den Fragebogen gemessen werden?\\
Wer sind die Zielpersonen?\\
Welche Art von fragen sind dafür geeignet?\\
Erst wenn diese Fragen geklärt sind, kann der Fragebogen erarbeitet werden.

Die ersten beiden fragen sind von vorneherein klar:
Ob DHBW Star den Studenten besser gefällt als die Offiziellen Webseiten.
Die Entscheidung für die Fragen Art fiel auf eine Mischung von Messskala und Freitext.
Dabei werden alle Elemente der Seite zunächst einzeln Bewertet und am Schluss die gesamte Webseite.
Zuletzt wird noch nach weiteren Verbesserungen und fehlende Features gefragt.
Bei der Formulierung der Fragen kommt die Checkliste von K. Wolfgang Kallus verwendet.
\todo {Quelle}
Um den die Hürde und den Aufwand gering zu halten, wird die Umfrage Online gemacht.

\subsection{Umfrage Erstellen}
Um die Umfrage zu erstellen, wird die Webseite \emph{www.surveymonkey.de} verwendet. Damit lassen kostenfrei, kleine Umfragen (bis zu 10 Fragen/Items) erstellen. Es können allerdings auch maximal nur 10 Antworten ausgewertet werden. Was leider erst im Nachhinein aufgefallen ist.
Die Entscheidung für die Webseite fiel zunächst weil sie ein Deutsche Toplevel Domain hat und nach einem ersten versuch als OK befunden wurde.

Zu erst wird eine Testumfrage erstellt um eine ersten Eindruck für die Erstellung und Auswertung der Umfrage zu bekommen.
Danach folgt der endgültige Fragebogen. Er besteht aus den folgenden 10 Fragen:
\begin{itemize}
	\item[01] {\emph{Frage}: Wie oft hast du DHBW Star in den letzten 7 tagen benutzt?\\
		\emph{Antwortmöglichkeiten}: kein mal, 1 mal, Mehr mal oder Täglich}
	\item[02]{\emph{F}:Auf welchem gerät hast du DHBW Star benutzt?\\
		\emph{AM}: PC, Tablet und Mobile}
	\item[03]{\emph{F}: Wie nützlich findest du die Startseite?\\
		\emph{AM}: sehr nützlich, etwas nützlich oder unnütz}
	\item[04]{\emph{F}: Wie gefällt dir die Darstellung des Scheduler im verglich zum Rapla?\\
		\emph{AM}: viel besser, besser, gleich, schlechter oder viel Schlechter}
	\item[05]{\emph{F}: Was gefällt oder fehlt dir beim Scheduler?\\
		\emph{AM}: Freitext}
	\item[06]{\emph{F}: Wie gefällt dir die Darstellung des Food im verglich zum sw-ka.de?\\
		\emph{AM}: viel besser, besser, gleich, schlechter oder viel Schlechter}
	\item[07]{\emph{F}: Was gefällt oder fehlt dir beim Food?\\
		\emph{AM}: Freitext}
	\item[08]{\emph{F}: Wie nützlich findest du die Link-Sammlung?\\
		\emph{AM}: sehr nützlich, etwas nützlich oder unnütz}
	\item[09]{\emph{F}: Wie nützlich findest du DHBW Star?\\
		\emph{AM}: sehr nützlich, etwas nützlich oder unnütz}
	\item[10]{\emph{F}: Was Fehlt dir noch bei DHBW Star?\\
		\emph{AM}: Integration von Dualis, Integration von Moodle,\\ Integration ÖPNV Fahrplan, Umfragen für Kurs Erstellen,\\ Benachrichtigungen für Kurs, Individueller Scheduler (Vorlesungen ein- und ausblenden, eigene Termin hinzufügen), Todo Liste (Hausaufgaben / Abgaben)\\ und Sonstiges (bitte angeben)[ Freitext]}
\end{itemize}
Mit Frage 1 wird die Häufigkeit der Nutzung abgefragt. Dies soll Auskunft darüber geben, ob sie im Testzeitraum eventuell schon als Alternative zum Einsatz gekommen ist.

Die zweite Frage soll eine Überblick über die verwendeten Geräte geben. Zudem kann so leichter erkannt werden, falls Probleme mit einem Gerätetype zusammen hängen.

Frage 3 bis 7 messen die Qualität der Darstellung von DHBW Star im Verhältnis zu den Offiziellen Angeboten. Die Freitex Felder geben die Möglichkeit, seine Entscheidung zu begründen.

Die Fragen 8 und 9 messen die Nützlichkeit der Link-Sammlung und der gesamten DHBW-Star App.

Zuletzt wird noch nach weiteren Verbesserungen und neuen Funktionen Gefragt. Daraus könnten sich zukünftige Arbeiten ergeben.

\todo {Daniel Schon fertig?}

 
