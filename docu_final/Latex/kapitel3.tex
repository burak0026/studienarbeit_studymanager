\chapter{Umsetzung der Applikation}
Test test tes tes 

\section{React-Applikation}
\subsection{Ziel der Applikation}
\subsubsection{Forscherfragen antwort}

\subsection{Grundkonzept der App}
\subsubsection{Benutzeroberfläche}
\subsubsection{Navigation und Struktur}
\subsubsection{Funktion und Komponenten}

\subsection{Grundkonzept der App}
\subsubsection{Home}
\subsubsection{Food}
\subsubsection{Scheduler}
\subsubsection{Links}

\subsection{Zusammenfassung}
\section{Docker}

Damit die React App vom System unabhängig funktioniert, läuft sie in einem Docker Container. Dadurch werden Probleme mit unterschiedlichen Programmversionen und Konfigurationen verhindert.
Da die geplante Mensa API nicht funktioniert, wird als Alternative ein Container verwendet, der die Mensa API lokal bereitstellt.

\subsection{Die Dockerfile, compose.yaml und .env Dateien}

Für den React Containers wird ein Dockerfile Datei benötigt. In dieser wird beschrieben wie der Container zusammengebaut wird. Ein neuer Container kann auf vorhanden aufbauen oder komplett neu erstellt werden.

\begin{lstlisting}[language=vhdl,
	frame=single,           % Ein Rahmen um den Code
	framexleftmargin=15pt,  % Rahmen link von den Zahlen
	style=algoBericht,
	label={Dockerfile},
	captionpos=b           % Caption unter den Code setzen
	caption={Dockerfile für DHBW-Star}]
FROM node:current-alpine

RUN mkdir /reactApp
WORKDIR /reactApp

# Install React and dependencies
COPY . .
RUN npm init -y
RUN npm install express cors axios
RUN npm install
RUN npm install ical.js
RUN npm install -D concurrently
RUN yarn build

# Start both React app and proxy server
CMD ["npm", "run", "start:all"]
\end{lstlisting}

Für DHBD Star wird der node Container von Dockerhub.com als Basis verwendet. Dieser besteht aus \emph{alpine}, einem minimalistischen Betriebssystem und NoteJS welcher für React benötigt wird.
Mit \emph{COPY} werden die Dateien des Projekts in den Container kopiert und mit \emph{RUN npm init} und \emph{RUN npm install} werden die benötigten Pakete für NoteJS installiert.
Zum Schluss wird mit \emph{RUN yarn build} das React Projekt gebaut und mit \emph{CMD ["npm", "run", "start:all"]} React und der Proxy gestartet. Der \emph{CMD} befehlt wird beim erstellen eines Container aus dem Image, immer ausgeführt. 

Damit die Mensa API und React zusammen starten, wird die \emph{compose.yaml} Datei und Docker compose verwendet.

\begin{lstlisting}[language=vhdl,
	frame=single,           % Ein Rahmen um den Code
	framexleftmargin=15pt,  % Rahmen link von den Zahlen
	style=algoBericht,
	label={Dockerfile},
	captionpos=b           % Caption unter den Code setzen
	caption={compose.yaml für DHBW-Star}]
version: '3.9'
services:
  react-app:
    build:
      dockerfile: ./Dockerfile
    tags:
      - "react:latest"
    ports:
      - "3003:3003"
      - "3002:3002"
    volumes:
      - ./src:/reactApp/src:ro
      - ./proxy-server.js:/reactApp/proxy-server.js:ro
      - ./.env:/reactApp/.env:ro
  mensa-api:
    image: meyfa/ka-mensa-api
    ports:
      - "3001:8080"
    environment:
      - MENSA_CORS_ALLOWORIGIN=*
\end{lstlisting}

Für den Service \emph{react-app} (DHBW Star) wird mit \emph{build} die oben beschrieben Dockerfiel Datei angeben. Somit baut der Befehl \emph{docker compose --build} ein neue Image mit dem React Projekt.
Der \emph{tags} gibt dem Container einen eindeutigen Namen.
Mit \emph{ports} werden die Ports vom Proxy(3002) und React (3003) nach außen frei gegeben.
Damit bei kleineren Änderungen im Projekt nicht immer ein neues Image erstellt werden muss, wird mit \emph{volums} die Projektdateien in den laufenden Container eingebunden.
Für die \emph{mensa-api} wird noch die Umgebungsvariable \emph{MENSA\_CORS\_ALLOWORIGIN=*} definiert.

Um die Container an unterschiedliche Gegebenheiten des Systems anpassen zu können, ohne etwas am React Projekt ändern zu müssen, bietet docer compose die Möglichkeit Umgebungsvariablen aus einer \emph{.env} Datei heraus zu definieren.
Da diese nicht teil von git sein sollte, wird eine \emph{.env\_template} angelegt. In dies stehen sind die verwendeten Umgebungsvariable mit Beispielen drin.
React verlangt für alle Umgebungsvariablen, die in der App zur Verfügung stehen sollen, dass sie mit \emph{REACT\_APP\_} beginnen.

\begin{lstlisting}[language=vhdl,
	frame=single,           % Ein Rahmen um den Code
	framexleftmargin=15pt,  % Rahmen link von den Zahlen
	style=algoBericht,
	label={Dockerfile},
	captionpos=b           % Caption unter den Code setzen
	caption={.env für DHBW-Star}]
REACT_APP_MENSA_ADDRESS=http://localhost:3001 
REACT_APP_PROXY_ADDRESS=http://localhost:3002
PROXY_PORT=3002
PORT=3003
\end{lstlisting}

\emph{REACT\_APP\_MENSA\_ADDRESS} und \emph{REACT\_APP\_PROXY\_ADDRESS} sind die Adressen der Services. Dies müssen mit der in \emph{compose.yaml} definierten Ports übereinstimmen. 

\section{Webserver}
Damit das Projekt auch über das Internet erreichbar ist, wird ein Root-Webserver mit einer Domain Adresse verwendet. Auf diesem ist bereits Docker, Nginx und Certbot Installiert.
Per github.com wird das Projekt auf den Server kopiert und die \emph{.env} Datei angepasst. Die Ports bleiben die Selben, nur das \emph{localhost} wird durch die Subdomains \emph{mensa.Domain} für den Mensa Container und \emph{ical.Domain} für den Proxy ersetzt.

Damit die Container unter ihrer Subdomain erreichbar sind, müssen diese in der Nginx Konfiguriert werden. In der Datei \emph{.../sites-enabled/star.conf} wird für jeden Service mit \emph{proxy\_pass} eine Weiterleitung eingerichtet.
Da auf dem Server bereits andere Webdienst laufen, ist Startseite unter der Subdomain \emph{star.Domain} erreichbar. Mit \emph{Certbot} werden noch SSL Zertifikate für alle Subdomains erstellt und automatisch eingerichtet.

Nach dem Neustart von Nginx, steht DHBW Star für alle Tester zur Verfügung.

