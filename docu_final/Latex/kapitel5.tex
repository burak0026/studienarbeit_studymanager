\chapter{Datenanalyse}
Für die Auswertung eines Fragebogens gibt es verschiedene Methoden, die zum Einsatz kommen können. Der hier verwendete Fragebogen verwendet drei Arten von Antwortmöglichkeiten
\begin{itemize}
	\item[Typ 1:] {Ordinalskala: Antworten mit einer Rangfolge \\
		(z.B. sehr nützlich, etwas nützlich oder unnütz) \\
		Fragen 1, 3, 4, 6, 8 und 9}
	\item[Typ 2:] {Nominalskala: Antworten ohne Rangfolge \\
		(z.B. PC, Tablett und Mobile) \\
		Fragen 2 und 10}
	\item[Typ 3:] {Freie Antworten \\
		Fragen 5, 7 und 10 (Sonstiges) }
\end{itemize}
Für die verschiedenen Typen gibt es jeweils ein Verfahren zur Bewertung.
Typ 1 und 2 werden mit der Häufigkeitsmethode ausgewertet, wobei bei Typ 1 auch noch nach Rangfolge ausgewertet werden kann. Bei Typ 3 kommt eine qualitative Methode zum Einsatz.

Da für die Auswertung maximal nur 10 Datenpunkte zu Verfügung stehen, können keine statistischen Verfahren verwendet werden.
\newpage
\section{Einschränkungen der Aussagekraft }
Die befragten Personen sind alle im gleichen Kurs, da DHBW-Star speziell für den Tinf20B3 Kurs entwickelt wurde. Dadurch ist die Stichprobe sehr klein, weshalb die Ergebnisse nur sehr eingeschränkt auf andere Studenten übertragen sind.


Zudem ist ein davon auszugehen, dass die Befragten eher wohlwollend bewerten, da sie die Entwickler persönlich kennen.

Es ist dennoch möglich abzuschätzen, ob das Design in die richtige Richtung geht und ob DHBW-Star eine Verbesserung bringen würde.

Die nachfolgende Auswertung ist dementsprechend auch eher vorsichtig mit ihren Aussagen und daraus resultierenden Schlussfolgerungen.

\section{Aufbereitung von surveymonkey.de}
Die Webseite www.surveymonkey.de bereitet die Antworten einer Umfrage angemessen auf. Die Ergebnisse vom Typen 1 und 2 sind jeweils tabellarisch als auch mit einem Balkendiagramm dargestellt.
Die von Antworten zu Typ 3 sind in einer Tabelle zur weiteren Auswertung festgehalten.

Da ein Export nur für zahlenden Kunden angeboten wird, müssen die Daten manuell kopiert und per Screenshot gesichert werden.

Ein erster Überblick zeigt bereits, dass die Bewertungen überwiegend positiv ausgefallen sind.
\newpage
\section{Die Daten }
Es folgt ein Überblick über die erhobenen Daten.

Sechs der Befragten haben angegeben, dass sie die DHBW-Star nur einmal benutzt haben.
Jeweils eine Person hat sie täglich und eine kein mal verwendet.
Die restlichen zwei, haben mehrmals angegeben. \\
\emph{Anmerkung: Das 'kein mal' hat sich im Nachhinein als ein Versehen des Befragten herausgestellt. }
\\
Die genutzten Geräte sind nur PCs von 8 Personen und Mobile von 4 Personen. \\
\\
Als unnütz befand sich nur einer die Startseite. Der Reste empfand sie als mindestens nützlich. \\
\\
Der \emph{Scheduler} wurde von zwei Personen als gleich gut beurteilt. Der restlichen Mehrheit gefiel es (viel) besser. \\
\\
Bei \emph{Food} ist es ähnlich wie bei \emph{Scheduler}. Auch zwei Personen haben es gleich gut gefallen. Allerdings hat nur eine Person viel besser angegeben. Die restlichen 7 einigten sich auf besser.
\\
Die \emph{Link-Sammlung} ist einheitlich als mindestens nützlich bewertet. \\
\\
Genau so ist es bei der Gesamtbewertung von DHBW-Star. Unnütz gab keiner an und der Reste teilte sich gleichmäßig zwischen etwas und sehr nützlich auf. \\
\\
Auf die letzte Frage, was DHBW-Star noch fehlt, hat jeder Vorschlag mindestens 2 Stimmen bekommen. Die meisten davon entfielen auf 'Integration von Dualis' und 'Individueller Scheduler' mit jeweils 7 Stimmen.
\newpage
\section{Bewertung}
In diesem Abschnitt werden die einzelnen Ergebnisse bewertet und versucht erklängen dafür zu geben.

Da sich nicht alles mit Daten aus der Umfrage erklären lässt, kommen auch persönliche Erfahrungen zum Tragen. Diese sind durch ein vorangestelltes \textbf{Pers:} zu erkennen. 

\begin{itemize}
\item[Frage 1:]
{\textbf{Pers:} Es ist zu vermuten, dass sich die meisten Befragten die Seite nur einmal angeschaut haben, weil sie sich bereits so sehr an die bisherigen Informationsangebote gewohnt haben. Denkbar ist auch das einige von ihnen eine eigene Lösung umgesetzt haben. }
\item[Frage 2:]
{\textbf{Pers:} War zu erwarten. Es gibt nur wenige Personen mit Tablet im Kurs. }
\item[Frage 3:]
{\textbf{Pers:} Auch wenn die Startseite von der großen Mehrheit als mindestens etwas nützlich beurteilt wurde, ist offensichtlich noch Raum für Verbesserungen.}
\item[F. 4 \& 5:]
{Die meisten sind finden den \emph{Scheduler} wegen seinem schlichten Design und der Day-Ansicht zwar besser, vermissen aber auch noch Funktionalitäten wie Filter bzw. das ausblenden von Vorlesungen. \\
	\textbf{Pers:} Da Rapla schon lange für Ärger im Kurs geführt hat, war das Ergebnis zu erwarten.}
\item[F. 6 \& 7:]
{Auch hier wird das Simpel Design als gut bewertet. Negativ wird ein Fehler im Software beurteilt, weshalb nicht automatisch die Karte des heutigen Tages angezeigt wird. \todo{ref zu Fehler (Kapitel 6)}}
\item [Frage 8:]
{Die \emph{Link-Sammlung} ist mit Abstand die bestbewertete Funktion von DHB-Star. \\
	\textbf{Pers:} Aus eigener Erfahrung sind nachfragen zu einer der Links auch einer der häufigsten Fragen im Gruppenchat, die immer wieder gestellt werden. }
\item [Frage 9:]
{Die positiven Bewertungen aus den vorherigen Fragen Spiegeln sich auch in der Frage zum Gesamteindruck wieder. Auch die Kritikpunkte sind in der 50/50 Aufteilung zu wieder zu erkennen. }
\item [Frage 10:]
{\textbf{Pers:} Diese Frage dient hauptsächlich dazu, weiter Verbesserungsvorschläge für das Informationsangebot der DHBW machen zu können. }
\end{itemize}

\section{Fazit}
Die Umfrage zeigt, dass ein schlichtes sowie modernes Design und mehr Funktionalität für die offiziellen Informationsangebote der DHBW gewünscht sind. Auch die Sammlung der Informationen an einer zentralen Stelle würde es den Studierenden erleichtern, sie zu finden.

Eine Möglichkeit hierfür wäre eine Website, die individuell konfigurierbar ist und Zugriff auf alle erforderlichen Informationen bieten. Es ist wichtig, die einzelnen Seiten nicht zu überladen und dass die Wege (Anzahl Klicks) zu den wichtigen Informationen nicht zu lang sind.

Ob und wie diese Aussagen auch auf andere Studenten zutreffen, kann an dieser Stelle nicht mit Sicherheit gesagt werden. Anzunehmen ist allerdings, dass eine Überarbeitung des Designs von \emph{Rapla} und eine zentrale Informationsstelle, sich für alle Studierenden positiv auswirken würde.