%&bericht

%%%%%%%%%%%%%%%%%%%%%%%%%%%%%%%%%%%%%%%%%%%%%%%%%%%%%%%%%%%%%%%%%%%%%%%%%%%%%%%
%% Descr:       Vorlage für Berichte der DHBW-Karlsruhe
%% Author:      Prof. Dr. Jürgen Vollmer, juergen.vollmer@dhbw-karlsruhe.de
%% $Id: bericht.tex,v 1.25 2020/03/13 15:07:45 vollmer Exp $
%%  -*- coding: utf-8 -*-
%%%%%%%%%%%%%%%%%%%%%%%%%%%%%%%%%%%%%%%%%%%%%%%%%%%%%%%%%%%%%%%%%%%%%%%%%%%%%%%

\documentclass[
   ngerman          % neue deutsche Rechtschreibung
  ,a4paper          % Papiergrösse
%  ,twoside          % Zweiseitiger Druck (rechts/links)
%  ,10pt             % Schriftgrösse
%  ,11pt
 ,12pt
  ,pdftex
%  ,disable         % Todo-Markierungen auschalten
]{report}

%\usepackage[fontsize=13pt]{scrextend}
\usepackage{setspace}

% Bitte die Codierung Ihrer Dateien auswählen:
% \usepackage[latin1]{inputenc}    % Für UNIX mit ISO-LATIN-codierten Dateien
% \usepackage[applemac]{inputenc}  % Für Apple Mac
% \usepackage[ansinew]{inputenc}   % Für Microsoft Windows
\usepackage[utf8]{inputenc}   % UTF-8 codierte Dateien
\usepackage{tabularx}     
                                   % Dieses Dokument ist unter Unix erstellt, daher
  
\usepackage{listings}
\usepackage{color}
\definecolor{lightgray}{rgb}{.9,.9,.9}
\definecolor{darkgray}{rgb}{.4,.4,.4}
\definecolor{purple}{rgb}{0.65, 0.12, 0.82}
                              		% wird diese Input-Codierung benutzt.
\lstdefinelanguage{JavaScript}{
	keywords={typeof, new, true, false, catch, function, return, null, catch, switch, var, if, in, while, do, else, case, break},
	keywordstyle=\color{blue}\bfseries,
	ndkeywords={class, export, boolean, throw, implements, import, this},
	ndkeywordstyle=\color{darkgray}\bfseries,
	identifierstyle=\color{black},
	sensitive=false,
	comment=[l]{//},
	morecomment=[s]{/*}{*/},
	commentstyle=\color{purple}\ttfamily,
	stringstyle=\color{red}\ttfamily,
	morestring=[b]',
	morestring=[b]"
}

\usepackage{bericht}
%% ACHTUNG, wenn man eine eigene Formatdatei (bericht.fmt) benutzt, werden Änderungen an bericht.sty
%% erst wirksam, wenn die Format-Datei neu erzeugt wurde!!!
%% Genauer alle Änderungen, die textuell vor der nächsten Zeile ".... endofdump...." stehen
%% werden erst wirksam, wenn die Formatdatei neu erzeugt wurde
\csname endofdump\endcsname

%%%%%%%%%%%%%%%%%%%%%%%%%%%%%%%%%%%%%%%%%%%%%%%%%%%%%%%%%%%%%%%%%%%%%%%%%%%%%%%
%% Angaben zur Arbeit
%%%%%%%%%%%%%%%%%%%%%%%%%%%%%%%%%%%%%%%%%%%%%%%%%%%%%%%%%%%%%%%%%%%%%%%%%%%%%%%

\newcommand{\Autor}{Burak Özkan und Daniel Schomburg}
\newcommand{\MatrikelNummer}{9015631 und 6218975}
\newcommand{\Kursbezeichnung}{Tinf20B3}

% Falls es kein Firmenlogo gibt:
%  \newcommand{\FirmenLogoDeckblatt}{}

%\newcommand{\BetreuerFirma}{Titel Vorname Nachname}
\newcommand{\BetreuerDHBW}{Daniel Lindner}

%%%%%%%%%%%%%%%%%%%%%%%%%%%%%%%%%%%%%%%%%%%%%%%%%%%%%%%%%%%%%%%%%%%%%%%%%%%%%%%%%%%%%

% Wird auf dem Deckblatt und in der Erklärung benutzt:
\newcommand{\Was}{Studienarbeit }

%%%%%%%%%%%%%%%%%%%%%%%%%%%%%%%%%%%%%%%%%%%%%%%%%%%%%%%%%%%%%%%%%%%%%%%%%%%%%%%%%%%%%

\newcommand{\Titel}{App zur Identifizierung von Problemen des Informationsangebotes der DHBW Karlsruhe}
\newcommand{\AbgabeDatum}{22. Mai 2023}

%\newcommand{\Dauer}{ Wochen}

% \newcommand{\Abschluss}{Bachelor of Engineering}
\newcommand{\Abschluss}{Bachelor of Science}

\newcommand{\Studiengang}{Informatik / Informationstechnik}

\hypersetup{%%
  pdfauthor={\Autor},
  pdftitle={\Titel},
  pdfsubject={\Was}
}

%%%%%%%%%%%%%%%%%%%%%%%%%%%%%%%%%%%%%%%%%%%%%%%%%%%%%%%%%%%%%%%%%%%%%%%%%%%%%%%

% Wenn \includeonly{..} benutzt wird, werden nur diese Kaptitel ausgegeben.
\includeonly{
  abk
 ,kapitel1
 ,kapitel2
 ,kapitel3
 ,kapitel4
 ,kapitel5
 ,kapitel6
 ,changelog
 ,anhang
}

%%%%%%%%%%%%%%%%%%%%%%%%%%%%%%%%%%%%%%%%%%%%%%%%%%%%%%%%%%%%%%%%%%%%%%%%%%%%%%%

% Benutzt man das "biblatex"-Paket, dann muß das hier stehen:
% siehe auch die mit BIBLATEX markierten Zeilen in bericht.sty
\bibliography{bericht}

\begin{document}

%%%%%%%%%%%%%%%%%%%%%%%%%%%%%%%%%%%%%%%%%%%%%%%%%%%%%%%%%%%%%%%%%%%%%%%%%%%%%%%

\begin{titlepage}
\begin{center}
\vspace*{-2cm}
\hfill\includegraphics[width=3cm]{dhbw-logo}\\[2cm]
{\Huge \Titel}\\[1cm]
{\Huge\scshape \Was}\\[1cm]
{\large für die Prüfung zum}\\[0.5cm]
{\Large \Abschluss}\\[0.5cm]
{\large des Studienganges \Studiengang}\\[0.5cm]
{\large an der}\\[0.5cm]
{\large Dualen Hochschule Baden-Württemberg Karlsruhe}\\[0.5cm]
{\large von}\\[0.5cm]
{\large\bfseries \Autor}\\[1cm]
{\large Abgabedatum \AbgabeDatum}
\vfill
\end{center}
\begin{tabular}{l@{\hspace{2cm}}l}
%Bearbeitungszeitraum	        & \Dauer 			\\
Matrikelnummer	                & \MatrikelNummer	\\
Kurs			         		& \Kursbezeichnung	\\
Gutachter der Studienakademie	& \BetreuerDHBW	  	\\
\end{tabular}
\end{titlepage}

%%%%%%%%%%%%%%%%%%%%%%%%%%%%%%%%%%%%%%%%%%%%%%%%%%%%%%%%%%%%%%%%%%%%%%%%%%%%%%%

\input{erklaerung.tex}

%%%%%%%%%%%%%%%%%%%%%%%%%%%%%%%%%%%%%%%%%%%%%%%%%%%%%%%%%%%%%%%%%%%%%%%%%%%%%%%

\begin{abstract}
Diese Studienarbeit evaluiert die Benutzerfreundlichkeit und Akzeptanz der DHBW-Star-Plattform, die speziell für den Studiengang Tinf20B3  an der Dualen Hochschule Baden-Württemberg (DHBW) entwickelt wurde. Der Hauptzweck dieser Forschung besteht darin, Benutzerfeedback zu sammeln und zu analysieren, um Erkenntnisse über die Benutzererfahrung und Verbesserungsmöglichkeiten zu gewinnen.

Diese Methode umfasste einen Fragebogen mit Fragen zu verschiedenen Aspekten der Plattform, einschließlich  Benutzerfreundlichkeit, Design und  Funktionalität. Die Antwortmöglichkeiten waren eine Mischung aus ordinalen, nominalen und offenen Fragen. Die Ergebnisse zeigen, dass die Mehrheit der Befragten die Plattform als nützlich empfinden und ihr einfaches, modernes Design und die zentrale Linksammlung hervorheben. Verbesserungsvorschläge sind unter anderem die Integration von Dualis und ein individuelle Planer.

Diese Untersuchung liefert wertvolle Einblicke in die Benutzererfahrung mit DHBW-Star und liefert konkrete Hinweise für zukünftige Verbesserungen. Dieses Ergebnis unterstreicht die Bedeutung des Nutzerfeedbacks bei der Entwicklung und Optimierung von Informationsdiensten im akademischen Umfeld.
\end{abstract}

\newpage
\pagenumbering{roman}
\tableofcontents           % Inhaltsverzeichnis hier ausgeben
\listoffigures             % Liste der Abbildungen
%\listoftables              % Liste der Tabellen
\lstlistoflistings         % Liste der Listings
%\listofequations           % Liste der Formeln

% Jetzt kommt der "eigentliche" Text
%\include{abk} 				% Abkürzungsverzeichnis
\pagenumbering{arabic}             
\begin{onehalfspace}
\chapter{Einleitung}
Die fortschreitende Digitalisierung hat in den letzten Jahren weitreichende Veränderungen in der Hochschulwelt mit sich gebracht. Immer mehr Bildungseinrichtungen nutzen digitale Informationskanäle, um Studierende, Mitarbeiter und andere Interessengruppen über aktuelle Ereignisse, Veranstaltungen und wichtige Informationen auf dem Laufenden zu halten\cite{aachenerzeitung2022}. Die Duale Hochschule Baden-Württemberg (DHBW) Karlsruhe bildet da keine Ausnahme. 
Mit langjähriger Erfahrung als Hochschule für angewandte Wissenschaften hat die DHBW Karlsruhe vielfältige Informationen geschaffen, um den Bedürfnissen ihrer Nutzer gerecht zu werden \cite{degruyter2021}. Bei der Bereitstellung von Informationen für eine große Anzahl von Benutzern können jedoch verschiedene Probleme auftreten.  Die Digitalisierung der Hochschulbildung stellt nicht nur akademische Inhalte in digitaler Form bereit, sondern eröffnet auch neue pädagogische Möglichkeiten zur Wissens- und Kompetenzvermittlung \cite{hochschulforumdigitalisierung}. Es ist daher Aufgabe der Hochschulleitung sowie der Konzeption konkreter Lehrveranstaltungen und Lernmaterialien, diese digitalen Möglichkeiten zu nutzen und weiterzuentwickeln \cite{springerlink2023}. 

Bei der Umsetzung dieser digitalen Transformation können jedoch auch Hindernisse auftreten, wie beispielsweise unsere natürliche Widerstandsfähigkeit gegenüber schnellen Veränderungen\cite{degruyter2021_2}. Dennoch müssen diese Herausforderungen angegangen werden, wenn wir die Chancen der Digitalisierung maximieren und die Hochschulbildung zukunftssicher machen wollen.

\section{Hintergrund und Kontext}
Die Duale Hochschule Baden-Württemberg (DHBW) Karlsruhe ist eine der größten Hochschulen für angewandte Wissenschaften in Deutschland. Mit über 8.000 Studierenden und mehr als 1.000 Mitarbeitern bietet die DHBW Karlsruhe ein breites Spektrum an Studiengängen in verschiedenen Fachbereichen an. Um den Anforderungen ihrer Nutzer gerecht zu werden, hat die DHBW Karlsruhe in den letzten Jahren eine Vielzahl von Informationsangeboten entwickelt, wie z.B. die Webseite, die Intranet-Plattform und verschiedene soziale Medien.
Trotz dieser Informationsangebote können jedoch Herausforderungen bei der Bereitstellung von Informationen für Studierende auftreten. Zum Beispiel können Informationen nicht immer auf den ersten Blick leicht zugänglich sein oder die Kommunikation kann nicht immer effektiv gestaltet werden. Daher ist es wichtig, die Bedürfnisse der Nutzer zu verstehen und Lösungen zu finden, um die Informationsbereitstellung zu optimieren.


\section{Zieldefinition}
Das Ziel dieser Studienarbeit ist es, eine Alternative zum aktuellen Informationsangebot der DHBW Karlsruhe zu entwickeln und alle Informationen auf einer zentralen Plattform zu sammeln. Hierfür sollen die Bedürfnisse der Nutzer analysiert und die Informationsangebote der DHBW Karlsruhe auf ihre Effektivität hin überprüft werden.
Im Rahmen der Studienarbeit sollen folgende Forschungsfragen beantwortet werden:
\begin{itemize}
	\item Welche Informationen sind für die Nutzer der DHBW Karlsruhe am wichtigsten?
	\item Welche Informationsquellen sind für die Nutzer am effektivsten?
	\item Wie können die Probleme im Informationsangebot behoben werden?
	\item Welche Features sollte die neue Informationsplattform haben?
	\item Welche Möglichkeiten gibt es zur Verbesserung?
\end{itemize}
\newpage
\section{Methodik und Vorgehen}
Um die Zielsetzung der Studienarbeit zu erreichen und die gestellten Forschungsfragen zu beantworten, werden verschiedene Methoden und Techniken angewendet. Hierzu zählen die Sammlung von Erfahrungen aus Sicht der Studierenden, die Analyse von persönlichen Erfahrungsgesprächen sowie die Entwicklung eines alternativen Informationsangebotsprototyps.
Die Analyse der Erfahrungen erfolgt durch die Auswertung von Feedback-Quellen wie beispielsweise Umfragen oder persönliche Gespräche mit Studierenden.Dabei wird besonders auf wiederkehrende Probleme und Schwierigkeiten im Umgang mit den Informationsangeboten geachtet. Diese Erfahrungen sind wichtig, um die Bedürfnisse und Erwartungen der Studierenden an das Informationsangebot der DHBW Karlsruhe besser zu verstehen und mögliche Verbesserungsmöglichkeiten zu identifizieren.
Nach der Analyse der Erfahrungen wird ein alternativer Informationsangebotsprototyp entwickelt, der auf den Bedürfnissen und Erwartungen der Studierenden basiert. Der Prototyp wird an einer Gruppe von Studierenden getestet, um Feedback und Verbesserungsvorschläge zu sammeln.
Insgesamt ist die Einbeziehung der Erfahrungen und des Feedbacks von Studierenden ein wichtiger Schritt, um ein Informationsangebot zu entwickeln, das den Bedürfnissen der Nutzer entspricht und eine positive Nutzererfahrung bietet.


\section{Beitrag und Relevanz}
Die Entwicklung einer alternativen Informationsplattform für die DHBW Karlsruhe und die anschließende Evaluation der Nutzererfahrung durch eine Umfrage haben das Ziel, die Informationsbereitstellung für Studierende und andere Interessenten zu optimieren. Die Ergebnisse dieser Studienarbeit haben somit den potentiellen Nutzen, die Effektivität der Informationsbereitstellung zu erhöhen und damit die Zufriedenheit der Nutzer zu steigern. Diese Studienarbeit leistet somit einen Beitrag zur Verbesserung des Informationsangebots an Bildungseinrichtungen und trägt damit auch zu einer erfolgreichen und zufriedenstellenden Hochschulerfahrung bei.
\newpage
\section{Aufbau der Studienarbeit}

Die Studienarbeit ist in mehrere Kapitel unterteilt, die sich mit verschiedenen Aspekten des Themas befassen. Nach der Einleitung werden in Kapitel 2 die theoretischen Grundlagen erläutert, die für die Entwicklung der App relevant sind. Kapitel 3 beschreibt die angewandten Methoden und Techniken für die Umsetzung, während in Kapitel 4 die Datenerhebung im Fokus liegt. Das Kapitel 5 befasst sich mit der Auswertung der Daten und abschließend fasst Kapitel 6 die wichtigsten Ergebnisse zusammen und gibt einen Ausblick auf zukünftige Entwicklungen.

%%%%%%%%%%%%%%%%%%%%%%%%%%%%%%%%%%%%%%%%%%%%%%%%%%%%%%%%%%%%%%%%%%%%%%%%%%%%%
%% Descr:       Vorlage für Berichte der DHBW-Karlsruhe, Ein Kapitel
%% Author:      Prof. Dr. Jürgen Vollmer, vollmer@dhbw-karlsruhe.de
%% $Id: kapitel2.tex,v 1.5 2017/10/06 14:02:51 vollmer Exp $
%%  -*- coding: utf-8 -*-
%%%%%%%%%%%%%%%%%%%%%%%%%%%%%%%%%%%%%%%%%%%%%%%%%%%%%%%%%%%%%%%%%%%%%%%%%%%%%%%



\chapter{Grundlagen}
In diesem Kapitel werden die Grundlagen der verwendeten Werkzeuge sowie die theoretischen Grundladen erläutert.
\section{React}
In diesem Unterkapitel werden die Grundlagen von React näher beschrieben um ein grundlegendes Verständnis zu generieren.

\subsection{Hintergrund und Motivation}
React ist eine JavaScript-Bibliothek zur Erstellung von Benutzeroberflächen, die von Facebook entwickelt wurde und seit 2013 öffentlich verfügbar ist.\cite{ReactGettingStarted} \\Die Hauptmotivation für die Entwicklung von React bestand darin, die Leistung und Skalierbarkeit von Facebooks eigenen Webanwendungen zu verbessern. Insbesondere suchten die Facebook-Entwickler nach einer Möglichkeit, große und komplexe Anwendungen zu erstellen, die schnell, reaktionsfreudig und einfach zu warten sind.\\
React ermöglicht es Entwicklern, wiederverwendbare Komponenten zu erstellen, die einfach zu pflegen und zu aktualisieren sind. Darüber hinaus kann React für verschiedene Anwendungen eingesetzt werden, einschließlich Single-Page-Anwendungen, Mobile-Apps und serverseitige Anwendungen.

\subsection{Überblick über React}

React ist eine bekannte JavaScript-Bibliothek, die zur Erstellung von Benutzeroberflächen verwendet wird und auf einem komponentenbasierten Ansatz basiert. Laut der offiziellen React-Dokumentation ermöglicht React eine deklarative Syntax, um UI-Komponenten zu definieren und zu rendern, die wiederverwendet werden können, um komplexe Benutzeroberflächen zu erstellen.\cite{ReactJS} \\
Es gibt mehrere Gründe, warum es sinnvoll ist, React zu erlernen. Ein wichtiger Grund ist, dass React auf JavaScript basiert und Entwicklern, die bereits JavaScript beherrschen, den Einstieg erleichtert. Ein weiterer Vorteil ist, dass React auf einem komponentenbasierten Ansatz basiert, der es einfacher macht, Webanwendungen in kleine, wiederverwendbare Komponenten aufzuteilen, die unabhängig voneinander erstellt und gewartet werden können.\cite{Kinsta} \\
Eine weitere Stärke von React ist die effektive Verwaltung des Zustands und die leistungsstarke Wiederverwendbarkeit von Komponenten. Dadurch wird die Entwicklung von Webanwendungen beschleunigt. Zudem gibt es eine engagierte Entwicklergemeinschaft, die sich ständig weiterentwickelt und verbessert, was zu einer hohen Verfügbarkeit von Lernressourcen und ständigen Verbesserungen der Bibliothek führt.\cite{Kinsta}\\
Die Beliebtheit von React in der Entwicklergemeinschaft zeigt sich auch in der Stack Overflow Umfrage 2021, bei der React als eine der beliebtesten Technologien im Bereich der Web-Frameworks genannt wurde.\cite{StackOverflowSurvey}\\
React hat auch die Entwicklung von Single-Page-Applications vereinfacht, indem es eine schnelle und nahtlose Benutzererfahrung ermöglicht, ohne dass die Seite neu geladen werden muss.\cite{Kinsta}\\
Zusammenfassend ist React eine wertvolle Fähigkeit für Webentwickler, da es eine effektive Verwaltung des Zustands und eine leistungsstarke Wiederverwendbarkeit von Komponenten bietet und die Entwicklung von Single-Page-Applications vereinfacht. Die Beliebtheit von React in der Entwicklergemeinschaft unterstreicht zudem seine Bedeutung für die Webentwicklung.

\begin{figure}[htbp]
	\centering
	\fbox{\includegraphics[height=0.3\textheight]{images/react-logo}}
	\caption{React-Logo}
\end{figure}

\subsection{Grundlagen über React}
\subsubsection{Virtual DOM}
Das Virtual DOM ist ein zentrales Konzept von React, einer JavaScript-Programmbibliothek zur Erstellung von webbasierten Benutzeroberflächen\cite{ReactWikipedia}. Es handelt sich dabei um eine leichtgewichtige JavaScript-Darstellung des Document Object Model (DOM), die in deklarativen Web-Frameworks wie React, Vue.js und Elm verwendet wird \cite{VueJsAdesso}. Das Virtual DOM ermöglicht es React, minimale DOM-Operationen auszuführen, wenn die Benutzeroberfläche neu gerendert wird. Im Gegensatz zum tatsächlichen DOM ist das Virtual DOM eine Art abstrakte Kopie, die deutlich kleiner ist und auf das Nötigste an Informationen beschränkt.\cite{ReactWikipedia}\\
Das Virtual DOM bietet mehrere Vorteile für die Entwicklung von Webanwendungen. Zum einen abstrahiert es manuelle DOM-Manipulationen vom Entwickler, was die Entwicklung von komplexen Benutzeroberflächen vereinfacht und zu einem vorhersehbareren Verhalten der Anwendung führt [1]. Zum anderen ermöglicht das Virtual DOM inkrementelles Rendering, was bedeutet, dass React nur die Komponenten rendert, die sich tatsächlich geändert haben \cite{ReactWikipedia}. Dies führt zu einer effizienteren Ausführung von Benutzeroberflächen-Updates und einer insgesamt schnelleren Anwendung.\\
Das folgende Bild illustriert den Unterschied zwischen dem tatsächlichen DOM und dem Virtual DOM:


\begin{figure}[htbp]
	\centering
	\fbox{\includegraphics[height=0.3\textheight]{images/virtualDom}}
	\caption{Virtual-DOM}
\end{figure}

Das Bild zeigt schematisch, wie das Virtual DOM in React funktioniert. Zunächst wird eine virtuelle Darstellung des DOMs in Form eines Baumes erstellt. Bei Änderungen an der Benutzeroberfläche wird ein neuer Baum erstellt und mit dem vorherigen Baum verglichen. Dabei werden nur die Unterschiede zwischen den beiden Bäumen ermittelt. Anschließend werden nur die geänderten Elemente im tatsächlichen DOM aktualisiert. Dieser Prozess wird als Reconciliation bezeichnet \cite{ReactVirtualDOM}.

\subsubsection{Komponenten und Props}
React ist eine Bibliothek für die Entwicklung von Benutzeroberflächen, die auf dem Konzept von Komponenten und Props basiert. Eine React-Komponente ist eine eigenständige und wiederverwendbare Einheit einer Benutzeroberfläche, die entweder als Klasse oder als Funktion definiert werden kann. Komponenten können andere Komponenten enthalten und selbst als Teil einer größeren Anwendung verwendet werden \cite{ReactComponentsAndProps}.\\
Props (kurz für "Properties") sind ein wichtiger Mechanismus zur Konfiguration von React-Komponenten. Sie dienen zum Übergeben von Daten von einer Komponente zur anderen und werden als Objekt an die Komponente übergeben. Innerhalb der Komponente können Props als Parameter verwendet werden. Im Gegensatz zum State, der zur Änderung der Benutzeroberfläche innerhalb einer Komponente verwendet wird, sind Props schreibgeschützt und können nicht direkt geändert werden. Durch das Verwenden von Props können Komponenten einfach wiederverwendet werden, indem sie in verschiedenen Kontexten mit unterschiedlichen Props konfiguriert werden \cite{ReactComponentsAndProps}.\\
Ein Beispiel für eine React-Komponente mit Props ist die Funktion \emph{Greeting}, die einen Gruß mit dem Namen des Benutzers anzeigt. Der Name wird als Prop an die Komponente übergeben und innerhalb der Funktion als Parameter verwendet. Hier ist das Codebeispiel:
\begin{lstlisting}[language=vhdl,
	frame=single,           % Ein Rahmen um den Code
	framexleftmargin=15pt,  % Rahmen link von den Zahlen
	style=algoBericht,
	label={Props-Komponenten},
	captionpos=b ,          % Caption unter den Code setzen
	caption={Beispiel Kompoonente mit Probs in React}]
import React from 'react';
	
function Greeting(props) {
   return <h1>Hello, {props.name}!</h1>;
}
	
	export default Greeting;
\end{lstlisting}

Alle React-Komponenten müssen sich im Bezug auf ihre Props als sogenannte "pure functions" verhalten \cite{ReactComponentsAndProps}. Das bedeutet, dass die Funktion der Komponente nur von ihren Props abhängen sollte und keine weiteren Seiteneffekte haben darf. Eine Komponente, die sich wie eine "pure function" verhält, ist leichter zu testen und zu warten.\\
Es gibt auch fortgeschrittene Konzepte im Zusammenhang mit Props, wie das Konzept des "Render Props". Hierbei handelt es sich um eine Technik zum Austauschen von Code zwischen React-Komponenten, bei der eine Komponente eine Funktion als Prop akzeptiert, die ein React-Element zurückgibt. Dadurch können Komponenten dynamisch wiederverwendet werden und die Codebasis der Anwendung wird vereinfacht\cite{ReactRenderProps}.\\
Um mit React-Komponenten und Props zu arbeiten, gibt es verschiedene Möglichkeiten. Funktionskomponenten sind die einfachste Art, eine Komponente zu definieren, indem man eine JavaScript-Funktion schreibt. Klassenkomponenten bieten mehr Funktionen, wie den Zugriff auf den State, die Möglichkeit, Lifecycle-Methoden zu definieren und vieles mehr \cite{RunebookReactComponentsAndProps}.\\
Insgesamt sind Komponenten und Props wichtige Konzepte in React, die Entwicklern helfen, wiederverwendbare Benutzeroberflächenkomponenten zu erstellen und die Effizienz der Entwicklung zu erhöhen.\\
\subsubsection{State}
In React ist der "State" ein Objekt, das in einer Komponente definiert wird und Werte speichert, die zur Laufzeit der Anwendung verändert werden können. Der Zustand wird in der Regel verwendet, um die Darstellung der Benutzeroberfläche zu aktualisieren, wenn sich etwas ändert. Wenn der Zustand eines Komponentenobjekts geändert wird, wird die Methode "render()" aufgerufen, um die Änderungen in der Benutzeroberfläche anzuzeigen \cite{W3SchoolsReactState}.\\

Der Zustand wird oft als Schlüssel-Wert-Paar-Objekt definiert, wobei jeder Schlüssel eine Eigenschaft repräsentiert, die geändert werden kann, und jeder Wert den aktuellen Wert dieser Eigenschaft darstellt. State-Objekte können als eine Art Konfiguration betrachtet werden, die der Komponente zur Verfügung gestellt werden, um ihre Eigenschaften und ihr Verhalten zu steuern \cite{FreeCodeCampStateInReact}.\\
Der Zustand ist ein beobachtbares Objekt und kann somit verändert werden. In React-Komponenten kann der Zustand durch die Verwendung der \emph{setState()}-Funktion aktualisiert werden.\\
Eine sorgfältige Verwaltung des Zustands in einer React-Anwendung ist wichtig, um Fehler zu vermeiden und die Leistung zu optimieren. Die Organisation des Zustands und die Datenfluss zwischen den Komponenten sollten gut durchdacht sein, um überflüssigen oder doppelten Zustand zu vermeiden \cite{ReactManagingState}.\\
Hier ist ein Beispiel, das zeigt, wie der Zustand in React-Komponenten implementiert werden kann:

\begin{lstlisting}[language=vhdl,
	frame=single,           % Ein Rahmen um den Code
	framexleftmargin=15pt,  % Rahmen link von den Zahlen
	style=algoBericht,
	label={State-Fkt.},
	captionpos=b ,          % Caption unter den Code setzen
	caption={Beispiel State in Reeact}]
import React, { useState } from "react";
	
function Example() {
  // Initialisieren des State-Objekts
  const [count, setCount] = useState(0);
  //Erhoehen des Zaehlers, wenn der Button geklickt wird
  function increaseCount() {
    setCount(count + 1);
  }
	
  return (
  <div>
  <p>You clicked {count} times</p>
  <button onClick={increaseCount}>Click me</button>
  </div>
  );
 }
\end{lstlisting}

In diesem Beispiel wird die \emph{useState()} -Hook verwendet, um den Zustand der Komponente zu initialisieren. Die Funktion \emph{useState()} gibt ein Array zurück, das den aktuellen Zustand und eine Funktion zur Aktualisierung des Zustands enthält. In diesem Fall wird der Zustand mit \emph{count} initialisiert und die Funktion \emph{setCount()} wird verwendet, um den Zustand zu aktualisieren. Wenn der Button geklickt wird, wird die \emph{increaseCount()} Funktion aufgerufen, die den Zähler erhöht und den Zustand mithilfe von \emph{setCount()} aktualisiert. Die Änderung des Zustands löst eine erneute Ausführung der Komponente aus, und die aktualisierten Werte werden in der Benutzeroberfläche angezeigt.


\subsection{Verwendung von React}
\subsubsection{Einrichtung von React}
 Eine der einfachsten Möglichkeiten, ein neues React-Projekt zu starten, ist mit einer einfachen HTML-Seite und einigen Skript-Tags \cite{deLegacyReactjs}. Dadurch ist es möglich in kürzester Zeit eine Grundeinstellung erstellen.\\
 Eine andere Möglichkeit, mit der Entwicklung von React-Anwendungen zu beginnen, bietet der \emph{Create-React-App}-Generator\cite{vsCodeReactTutorial}.Der Generator ist eine schnelle und einfache Möglichkeit, neue React-Projekte einzurichten. Um den Generator zu verwenden, wird der folgende Befehl im Terminal ausgeführt:
 
\begin{lstlisting}[language=vhdl,
	frame=single,           % Ein Rahmen um den Code
	framexleftmargin=15pt,  % Rahmen link von den Zahlen
	style=algoBericht,
	label={App-Erstellung},
	captionpos=b ,          % Caption unter den Code setzen
	caption={Beispiel Anwendungserstellung}]
npx create-react-app my-app
cd my-app
npm start
\end{lstlisting}
Durch  Ausführen von \emph{npx create-react-app my-app} wird ein neues React-Projekt mit dem Namen \emph{my-app} erstellt. Mit \emph{cd my-app} wird in das neu erstellte Verzeichnis gewechselt. Anschließend wird die Anwendung mit dem Befehl\emph{npm start} gestartet.
Der Webserver startet und die Anwendung ist unter \emph{http://localhost:3000} im Browser aufrufbar.\\ 
React wurde entwickelt, um keine Annahmen über den Rest des Technologie-Stacks zu treffen. So ist es möglich neue Features zu entwickeln, ohne bestehenden Code umzuschreiben\cite{deLegacyReactjsDocs}. React kann  auf dem Server mit Node oder als mobile Anwendung mit React Native gerendert werden. 
Zusammenfassend lässt sich sagen, dass die Einrichtung von React ein unkomplizierter Prozess ist, mit dem es möglich ist in kurzer Zeit mit der Entwicklung von Benutzeroberflächen für Web- und mobile Anwendungen zu beginnen.

\subsubsection{Komponentenentwicklung}
Die Entwicklung von Komponenten ist ein wichtiger Aspekt bei der Verwendung von React. React-Komponenten sind wiederverwendbare Elemente, die das Erstellen von Benutzeroberflächen erleichtern. Eine React-Komponente implementiert eine \emph{render()}-Methode, die Eingabedaten (Props) nimmt und zurückgibt, was gerendert wird. Es verwendet eine XML-ähnliche Syntax namens JSX, mit der es HTML-ähnlichen Code direkt in JavaScript schreiben kann.\cite{deLegacyReactjs}\\
Ein einfaches Beispiel für eine React-Komponente ist ein "Gruß" (Greeting) Komponente, die eine Begrüßungsnachricht basierend auf dem übergebenen Namen anzeigt:
\begin{lstlisting}[language=vhdl,
	frame=single,           % Ein Rahmen um den Code
	framexleftmargin=15pt,  % Rahmen link von den Zahlen
	style=algoBericht,
	label={Greeting-Komponente},
	captionpos=b ,          % Caption unter den Code setzen
	caption={Beispiel Komponentenentwicklung }]
import React, { Component } from 'react';

class Greeting extends Component {
     render() {
        const { name } = this.props;
        return <h1>Hallo, {name}!</h1>;
    }
}
\end{lstlisting}
In diesem Beispiel wird eine Greeting-Komponente erstellt, die die \emph{render()}-Methode implementiert. Auf die übergebenen Eingabedaten, welche an die Komponente übergeben werden, ist es mittels \emph{this.props} möglich darauf zuzugreifen\cite{deLegacyReactjs}. Mit der JSX-Syntax wird eine Willkommensnachricht direkt in ein HTML-ähnliches Element einfügen. \\

React fördert einen Komponentenentwicklungsansatz, um modulare, wiederverwendbare Benutzeroberflächen zu erstellen. Die Aufteilung der Benutzeroberfläche in kleinere Komponenten erleichtert die Organisation und Wartung des Codes.

\subsubsection{Ereignisbehandlung}
Die Ereignisbehandlung ist ein weiterer wichtiger Aspekt bei der Verwendung von React. Dadurch ist es möglich auf Benutzeraktionen wie Klicks und Tastenanschläge zu reagieren und die Anwendung entsprechend zu aktualisieren.Bei React-Elementen ist die Ereignisbehandlung ähnlich wie bei DOM-Elementen, es gibt jedoch einige syntaktische Unterschiede. Beispielsweise werden die Ereignisse von React in CamelCase statt in Kleinbuchstaben benannt, und JSX übergibt Funktionen als Event-Handler anstelle von Strings.\cite{react_de_handling_events}

Ein einfaches Beispiel für die Ereignisbehandlung in React ist eine Schaltfläche (Button), die bei einem Klick eine Nachricht in der Konsole ausgibt:

\begin{lstlisting}[language=vhdl,
	frame=single,           % Ein Rahmen um den Code
	framexleftmargin=15pt,  % Rahmen link von den Zahlen
	style=algoBericht,
	label={Events},
	captionpos=b ,          % Caption unter den Code setzen
	caption={Beispiel EventHandler}]
import React, { Component } from 'react';

class ButtonClick extends Component {
     handleClick() {
         console.log('Button wurde geklickt!');
     }
 
 render() {
 	return (
 	    <button onClick={this.handleClick}>
 	    Klick mich!
 	    </button>
 	    );
     }
 }

\end{lstlisting}

In diesem Beispiel wird eine \emph{ButtonClick} Komponente erstellt, die die \emph{render()} Methode implementiert und ein \emph{<button>} Element mit einem \emph{onClick Eventhandler} zurückgibt. Der Eventhandler ist die \emph{handleClick} Methode, die in der Komponente definiert ist. Wenn der Benutzer auf den Button klickt, wird die \emph{handleClick} Methode ausgeführt und die Nachricht "Button wurde geklickt!" in der Konsole angezeigt.\cite{react_de_handling_events}\\

Die Ereignisbehandlung in React ermöglicht es, interaktive Benutzeroberflächen zu erstellen und auf Benutzeraktionen dynamisch zu reagieren. Durch die Verwendung von Eventhandlern können Anwendungen auf Benutzerinteraktionen reagieren und den Zustand der Anwendung und die Benutzeroberfläche entsprechend aktualisieren.

\subsubsection{JSX}
JSX ist eine XML-ähnliche Syntax, die von React verwendet wird, um die Benutzeroberfläche einer Anwendung zu definieren \cite{deLegacyReactjs}. JSX ermöglicht es Entwicklern, die Struktur und das Erscheinungsbild von Komponenten ähnlich wie HTML zu beschreiben. Die Verwendung von JSX ist optional,  hat sich jedoch als effizient und bequem für die Arbeit mit React-Komponenten erwiesen.\\
In JSX werden Komponenten als Tags definiert, die HTML-Tags ähneln, mit einigen Unterschieden. Zum Beispiel behandelt React Komponenten in Kleinbuchstaben  als DOM-Tags und Komponenten in Großbuchstaben  als benutzerdefinierte Komponenten\cite{ReactComponentsAndProps}. Dies bedeutet, dass \emph{<div />} ein HTML div-Tag darstellt, während \emph{<Welcome />} eine benutzerdefinierte React-Komponente darstellt, vorausgesetzt, dass \emph{Welcome} im Scope ist.\\
JSX bietet auch die Möglichkeit, JavaScript-Ausdrücke in die Komponentenstruktur einzubetten, indem sie in geschweiften Klammern {} eingeschlossen werden. Diese Ausdrücke können Variablen, Funktionen oder Berechnungen enthalten, die zur Laufzeit ausgewertet werden.\\
Ein weiterer Aspekt von JSX ist das Übergeben von  Daten an Komponenten mithilfe von Eigenschaften. Props können in JSX  wie HTML-Attribute definiert werden und schließen bei JavaScript-Ausdrücken den Wert in geschweiften Klammern ein\cite{deLegacyReactjs}. Innerhalb der Komponente sind diese Props über this.props zugänglich.\\
Beispiel React-Komponente mit JSX: 


\begin{lstlisting}[language=vhdl,
	frame=single,           % Ein Rahmen um den Code
	framexleftmargin=15pt,  % Rahmen link von den Zahlen
	style=algoBericht,
	label={JSX},
	captionpos=b ,          % Caption unter den Code setzen
	caption={Beispiel JSX }]
class Welcome extends React.Component {
        render() {
        	return <h1>Hallo, {this.props.name}!</h1>;
        }
    }

    ReactDOM.render(
    <Welcome name="Max" />,
    document.getElementById('root')
    );
\end{lstlisting}
In diesem Beispiel wird eine benutzerdefinierte React-Komponente namens \emph{Welcome} erstellt, die einen Namen über Props erhält und eine Begrüßungsnachricht mit dem Namen anzeigt.

Insgesamt ist JSX eine leistungsstarke Syntax, die die Arbeit mit React-Komponenten erleichtert und die Entwicklung von Anwendungen effizienter gestaltet. Es ermöglicht Entwicklern, die Struktur ihrer Anwendung auf eine Weise zu definieren, die sowohl intuitiv als auch leicht verständlich ist.	

\subsection{Fortgeschrittene Themen in React}
React bietet eine Vielzahl von fortgeschrittenen Techniken und Bibliotheken, um komplexe Anwendungen zu entwickeln. Einige dieser Techniken sind Redux oder Context API, React Router, React Hooks und serverseitiges Rendern.
\subsubsection{Redux oder Context API}
Redux und Context API sind zwei Bibliotheken für React, die es ermöglichen, Daten auf einfache Weise durch die Komponenten einer Anwendung zu verteilen. Redux ist ein leistungsfähiges Tool für die Verwaltung von Anwendungsdaten, während Context API eine leichtgewichtige Alternative ist \cite{reduxgettingstarted, reactcontext}.

Redux funktioniert auf der Grundlage eines zentralisierten Speichers, in dem alle Anwendungsdaten gespeichert werden. Jede Komponente, die Daten benötigt, kann sie einfach aus dem Store abrufen und bei Bedarf aktualisieren. Redux bietet auch Tools wie Actions, Reducers und Middleware, um die Verwaltung von Anwendungsdaten zu vereinfachen und zu automatisieren.

Context API hingegen ermöglicht es Entwicklern, Daten auf einfache Weise durch die Komponentenbaumhierarchie zu verteilen. Jede Komponente kann auf den Kontext zugreifen und Daten abrufen oder aktualisieren, ohne auf Props oder eine globale Store-Instanz angewiesen zu sein. Context API ist eine einfachere Alternative zu Redux, die sich besser für kleinere Anwendungen eignet.
\subsubsection{React Router}
React Router ist eine Bibliothek für React, die es ermöglicht, dynamische URLs in einer Single-Page-Anwendung zu verwenden. Mit React Router können Entwickler die Navigation innerhalb der Anwendung auf einfache Weise definieren und steuern, ohne dass die Seite neu geladen werden muss. Die Integration von React Router in eine Anwendung ist relativ einfach und erfordert nur wenige Schritte \cite{reactroutertutorial, reactrouteroverview}.

React Router bietet verschiedene Funktionen wie die Definition von Routen und Parametern, die Möglichkeit zur Integration von Serverseitigem Rendering und die Möglichkeit, mit anderen Bibliotheken wie Redux zu interagieren. React Router ist eine wichtige Bibliothek für React-Entwickler, da es die Navigation innerhalb der Anwendung vereinfacht und eine bessere Benutzererfahrung bietet.
\subsubsection{React Hooks}
React Hooks ist eine Funktion in React, die es Entwicklern ermöglicht, Zustände und Effekte in funktionellen Komponenten zu verwenden. Mit Hooks können Entwickler den Zustand in einer Komponente speichern, ohne eine Klasse erstellen zu müssen \cite{reacthooksintro, w3schoolsHooks}.

Hooks bieten eine Reihe von Funktionen wie useState, useEffect und useContext, um den Zustand und die Effekte innerhalb der Komponente zu verwalten. Sie ermöglichen auch die Verwendung von benutzerdefinierten Hooks, um spezifische Funktionalitäten in verschiedenen Komponenten zu nutzen.
\subsubsection{Serverseitiges Rendern}
Serverseitiges Rendern (SSR) ist ein Prozess, bei dem React-Komponenten auf dem Server gerendert werden, bevor sie an den Browser gesendet werden. SSR bietet viele Vorteile, darunter eine schnellere Ladezeit der Seite, bessere Suchmaschinenoptimierung und bessere Leistung auf langsamen oder mobilen Geräten \cite{reactssr1, reactssr2}.

SSR in React kann auf verschiedene Arten implementiert werden, je nach den Anforderungen der Anwendung. Die Integration von SSR erfordert jedoch in der Regel eine Reihe von Schritten, einschließlich der Konfiguration des Servers und der Anpassung der Anwendung.

\subsection{Best Practices in React-Entwicklung}
Modularität und Wiederverwendbarkeit von Komponenten ist ein wichtiger Aspekt bei der Entwicklung von React-Anwendungen. Durch die Aufteilung einer Anwendung in kleine, wiederverwendbare Komponenten kann der Code besser organisiert und leichter gewartet werden. Es ermöglicht auch die Erstellung von Komponenten-Bibliotheken, die in anderen Projekten wiederverwendet werden können \cite{dhiwise, reactcomponentlibrary}.

Um eine Komponente modular und wiederverwendbar zu gestalten, sollte sie unabhängig und gut dokumentiert sein. Die Komponente sollte spezifische Funktionen erfüllen und keine Abhängigkeiten von anderen Komponenten haben. Es ist auch wichtig, die Komponenten sauber zu benennen und ihre Funktionen klar zu dokumentieren.
\subsubsection{Testen von React-Anwendungen}
Das Testen von React-Anwendungen ist ein wichtiger Aspekt, um sicherzustellen, dass die Anwendung robust und fehlerfrei ist. Es gibt verschiedene Arten von Tests, die in React-Anwendungen durchgeführt werden können, wie Unit-Tests, Integrationstests und End-to-End-Tests \cite{reacttesting, reacthookstesting}.

Die Unit-Tests konzentrieren sich auf das Testen einer einzelnen Komponente, um sicherzustellen, dass sie wie erwartet funktioniert. Integrationstests testen die Interaktion zwischen verschiedenen Komponenten, während End-to-End-Tests die gesamte Anwendung testen.

Es ist wichtig, regelmäßig Tests durchzuführen und Testabdeckungen zu implementieren, um potenzielle Fehler zu identifizieren und zu beheben.
\subsubsection{Performance-Optimierung}
Performance-Optimierung ist ein wichtiger Aspekt bei der Entwicklung von React-Anwendungen, da eine schlechte Leistung zu einer negativen Benutzererfahrung führen kann. Es gibt verschiedene Möglichkeiten, die Performance von React-Anwendungen zu verbessern.

Eine Möglichkeit besteht darin, sicherzustellen, dass nur die notwendigen Komponenten gerendert werden. React bietet die Möglichkeit, Komponenten als pure Funktionen zu definieren, die nur von ihren Eingaben abhängen. Dadurch können unnötige Rendervorgänge vermieden werden.

Eine weitere Möglichkeit besteht darin, das Rendering von Komponenten zu optimieren. Eine Möglichkeit, dies zu tun, ist die Verwendung von Memoization. Memoization ist eine Technik, bei der das Ergebnis einer Funktion zwischengespeichert wird, wenn dieselben Eingaben erneut verwendet werden. Dadurch können unnötige Berechnungen vermieden werden.

Außerdem sollte darauf geachtet werden, dass Komponenten nicht zu viele Props erhalten, da dies die Leistung negativ beeinflussen kann. In diesem Fall sollte die Komponente möglicherweise in kleinere Komponenten aufgeteilt werden, die jeweils nur die notwendigen Props erhalten.

Eine weitere Möglichkeit, die Performance von React-Anwendungen zu verbessern, ist die Verwendung von Code-Splitting. Code-Splitting ist eine Technik, bei der der Code einer Anwendung in kleinere Stücke aufgeteilt wird, die separat geladen werden können. Dadurch kann die Ladezeit der Anwendung verringert werden (\cite{ReactOpt}, \cite{LogRocket}).
\subsubsection{Sicherheit und React}
Sicherheit ist ein wichtiger Aspekt bei der Entwicklung von React-Anwendungen, da Sicherheitslücken zu ernsthaften Problemen führen können. Es gibt verschiedene Best Practices, die bei der Entwicklung von sicheren React-Anwendungen beachtet werden sollten.

Eine Möglichkeit besteht darin, XSS-Angriffe (Cross-Site Scripting) zu vermeiden. XSS-Angriffe können dazu führen, dass schädlicher Code in die Anwendung eingeschleust wird. Um XSS-Angriffe zu vermeiden, sollten alle Daten, die von Benutzern eingegeben werden können, validiert und gefiltert werden. Darüber hinaus sollte auch das Rendering von HTML sorgfältig überwacht werden.

Eine weitere Möglichkeit, die Sicherheit von React-Anwendungen zu verbessern, besteht darin, die Codequalität zu verbessern. Eine gute Codequalität kann dazu beitragen, potenzielle Sicherheitslücken zu identifizieren und zu beheben, bevor sie ausgenutzt werden können.

Außerdem sollten bei der Entwicklung von React-Anwendungen Best Practices für die Sicherheit von Webanwendungen beachtet werden, wie z.B. die Verwendung von sicheren Passwörtern und die Vermeidung von unsicheren Bibliotheken oder Frameworks (\cite{ReactSecurity}, \cite{FreeCodeCamp}).

\subsection{Vergleich mit anderen Frontend-Technologien}
\subsubsection{Vor- und Nachteile von React im Vergleich zu Angular und Vue.js}
React, Angular und Vue.js sind beliebte Frontend-Technologien, die bei der Entwicklung von modernen Webanwendungen eingesetzt werden. Jede Technologie hat ihre Vor- und Nachteile, die bei der Wahl der Technologie berücksichtigt werden sollten.

React zeichnet sich durch seine Flexibilität und einfache Handhabung aus. Es ist eine Bibliothek und nicht ein vollständiges Framework, was bedeutet, dass Entwickler mehr Freiheit haben und nur die Teile von React verwenden können, die sie benötigen. Dadurch ist React auch in der Regel schneller und leichter als Frameworks wie Angular oder Vue.js. Außerdem ist React sehr beliebt und hat eine große und aktive Community, was zu einer Fülle von Ressourcen und Unterstützung führt (\cite{Hosttest}).

Im Vergleich zu Angular und Vue.js hat React jedoch auch einige Nachteile. Einer der Nachteile ist, dass React weniger Features und Werkzeuge bietet als Angular oder Vue.js. Dadurch müssen Entwickler möglicherweise zusätzliche Bibliotheken und Tools verwenden, um bestimmte Aufgaben zu erledigen. Ein weiterer Nachteil von React ist, dass es nicht so viele vorgefertigte UI-Komponenten wie Vue.js bietet, was die Entwicklung von Anwendungen erschweren kann (\cite{Kruschecompany}).
\subsubsection{Verwendung von React in größeren Projekten}
React eignet sich auch sehr gut für größere Projekte, da es die Möglichkeit bietet, Anwendungen in kleine, wiederverwendbare Komponenten aufzuteilen. Dadurch können Teams effizienter arbeiten und die Anwendung leichter warten und skalieren. Darüber hinaus bietet React auch die Möglichkeit, Code-Splitting und Lazy Loading zu verwenden, um die Ladezeit von Anwendungen zu reduzieren und die Leistung zu verbessern.

Es ist jedoch wichtig, dass bei größeren Projekten Best Practices für React-Anwendungen beachtet werden, wie z.B. die Verwendung von Redux für eine effektive Verwaltung des Anwendungszustands oder die Verwendung von TypeScript, um die Codequalität zu verbessern und die Entwicklungszeit zu verkürzen.
\subsubsection{Zukunft von React}
React hat sich in den letzten Jahren zu einer der beliebtesten Frontend-Technologien entwickelt und wird voraussichtlich auch in Zukunft eine wichtige Rolle spielen. Es gibt viele Indikatoren dafür, dass React weiterhin wachsen wird, wie z.B. die große und aktive Community, die kontinuierliche Entwicklung und Verbesserung der Bibliothek, sowie die Verwendung von React in vielen großen und erfolgreichen Unternehmen.

Darüber hinaus gibt es auch neue Entwicklungen wie React Native, das es Entwicklern ermöglicht, native mobile Anwendungen mit React zu entwickeln (\cite{LeanOcean}). Durch die Verwendung von React Native können Entwickler die Vorteile von React auch in der Entwicklung von mobilen Anwendungen nutzen.
\subsection{Fazit}
\subsubsection{Ausblick auf weitere Entwicklungen in React}
React hat sich in der Webentwicklung als eine der beliebtesten Frontend-Technologien etabliert und wird auch in Zukunft eine wichtige Rolle spielen. Es bietet viele Vorteile wie eine hohe Flexibilität, Leistung und eine große Community. Auch für größere Projekte eignet sich React sehr gut, da es die Möglichkeit bietet, Anwendungen in kleine, wiederverwendbare Komponenten aufzuteilen.

Die kontinuierliche Entwicklung und Verbesserung von React bleibt ein wichtiger Aspekt, um die Bedürfnisse der Benutzer zu erfüllen und wettbewerbsfähig zu bleiben. Die Entwicklergemeinschaft arbeitet ständig an neuen Funktionen und Verbesserungen, um React noch besser zu machen. Ein Beispiel dafür ist React Fiber, eine neue interne Rekonstruktionsstrategie von React, die auf eine bessere Leistung und mehr Flexibilität abzielt.

Ein weiterer wichtiger Trend in der React-Entwicklung ist die Verwendung von React Native, das es Entwicklern ermöglicht, native mobile Anwendungen mit React zu entwickeln. Durch die Verwendung von React Native können Entwickler die Vorteile von React auch in der Entwicklung von mobilen Anwendungen nutzen und so die Entwicklung von Web- und mobilen Anwendungen besser integrieren.

Insgesamt bleibt React eine der führenden Technologien in der Frontend-Entwicklung und wird voraussichtlich auch in Zukunft eine wichtige Rolle spielen. Durch die ständige Entwicklung und Verbesserung von React sowie die Integration von React Native in die Entwicklung von mobilen Anwendungen wird React auch weiterhin eine wettbewerbsfähige Technologie bleiben.


\section{Docker}

\begin{wrapfigure}{l}{0.4\textwidth}
\centering
\fbox{\includegraphics[width=0.25\textwidth,angle=270]{dhbw-logo}}
\end{wrapfigure}

Lorem ipsum dolor sit amet, consetetur sadipscing elitr, sed diam nonumy eirmod tempor invidunt ut labore et dolore magna aliquyam erat, sed diam voluptua. At vero eos et accusam et justo duo dolores et ea rebum. Stet clita kasd gubergren, no sea takimata sanctus est Lorem ipsum dolor sit amet. Lorem ipsum dolor sit amet, consetetur sadipscing elitr, sed diam nonumy eirmod tempor invidunt ut labore et dolore magna aliquyam erat, sed diam voluptua. At vero eos et accusam et justo duo dolores et ea rebum. Stet clita kasd gubergren, no sea takimata sanctus est Lorem ipsum dolor sit amet. Lorem ipsum dolor sit amet, consetetur sadipscing elitr, sed diam nonumy eirmod tempor invidunt ut labore et dolore magna aliquyam erat, sed diam voluptua. At vero eos et accusam et justo duo dolores et ea rebum. Stet clita kasd gubergren, no sea takimata sanctus est Lorem ipsum dolor sit amet.

Duis autem vel eum iriure dolor in hendrerit in vulputate velit esse molestie consequat, vel illum dolore eu feugiat nulla facilisis at vero eros et accumsan et iusto odio dignissim qui blandit praesent luptatum zzril delenit augue duis dolore te feugait nulla facilisi. Lorem ipsum dolor sit amet, consectetuer adipiscing elit, sed diam nonummy nibh euismod tincidunt ut laoreet dolore magna aliquam erat volutpat.

Ut wisi enim ad minim veniam, quis nostrud exerci tation ullamcorper suscipit lobortis nisl ut aliquip ex ea commodo consequat. Duis autem vel eum iriure dolor in hendrerit in vulputate velit esse molestie consequat, vel illum dolore eu feugiat nulla facilisis at vero eros et accumsan et iusto odio dignissim qui blandit praesent luptatum zzril delenit augue duis dolore te feugait nulla facilisi.

Nam liber tempor cum soluta nobis eleifend option congue nihil imperdiet doming id quod mazim placerat facer possim assum. Lorem ipsum dolor sit amet, consectetuer adipiscing elit, sed diam nonummy nibh euismod tincidunt ut laoreet dolore magna aliquam erat volutpat. Ut wisi enim ad minim veniam, quis nostrud exerci tation ullamcorper suscipit lobortis nisl ut aliquip ex ea commodo consequat.

Duis autem vel eum iriure dolor in hendrerit in vulputate velit esse molestie consequat, vel illum dolore eu feugiat nulla facilisis.

At vero eos et accusam et justo duo dolores et ea rebum. Stet clita kasd gubergren, no sea takimata sanctus est Lorem ipsum dolor sit amet. Lorem ipsum dolor sit amet, consetetur sadipscing elitr, sed diam nonumy eirmod tempor invidunt ut labore et dolore magna aliquyam erat, sed diam voluptua. At vero eos et accusam et justo duo dolores et ea rebum. Stet clita kasd gubergren, no sea takimata sanctus est Lorem ipsum dolor sit amet. Lorem ipsum dolor sit amet, consetetur sadipscing elitr, At accusam aliquyam diam diam dolore dolores duo eirmod eos erat, et nonumy sed tempor et et invidunt justo labore Stet clita ea et gubergren, kasd magna no rebum. sanctus sea sed takimata ut vero voluptua. est Lorem ipsum dolor sit amet. Lorem ipsum dolor sit amet, consetetur sadipscing elitr, sed diam nonumy eirmod tempor invidunt ut labore et dolore magna aliquyam erat.

Consetetur sadipscing elitr, sed diam nonumy eirmod tempor invidunt ut labore et dolore magna aliquyam erat, sed diam voluptua. At vero eos et accusam et justo duo dolores et ea rebum. Stet clita kasd gubergren, no sea takimata sanctus est Lorem ipsum dolor sit amet. Lorem ipsum dolor sit amet, consetetur sadipscing elitr, sed diam nonumy eirmod tempor invidunt ut labore et dolore magna aliquyam erat, sed diam voluptua. At vero eos et accusam et justo duo dolores et ea rebum. Stet clita kasd gubergren, no sea takimata sanctus est Lorem ipsum dolor sit amet. Lorem ipsum dolor sit amet, consetetur sadipscing elitr, sed diam nonumy eirmod tempor invidunt ut labore et dolore magna aliquyam erat, sed diam voluptua. At vero eos et accusam et justo duo dolores et ea rebum. Stet clita kasd gubergren, no sea takimata sanctus.

Lorem ipsum dolor sit amet, consetetur sadipscing elitr, sed diam nonumy eirmod tempor invidunt ut labore et dolore magna aliquyam erat, sed diam voluptua. At vero eos et accusam et justo duo dolores et ea rebum. Stet clita kasd gubergren, no sea takimata sanctus est Lorem ipsum dolor sit amet. Lorem ipsum dolor sit amet, consetetur sadipscing elitr, sed diam nonumy eirmod tempor invidunt ut labore et dolore magna aliquyam erat, sed diam voluptua. At vero eos et accusam et justo duo dolores et ea rebum. Stet clita kasd gubergren, no sea takimata sanctus est Lorem ipsum dolor sit amet. Lorem ipsum dolor sit amet, consetetur sadipscing elitr, sed diam nonumy eirmod tempor invidunt ut labore et dolore magna aliquyam erat, sed diam voluptua. At vero eos et accusam et justo duo dolores et ea rebum. Stet clita kasd gubergren, no sea takimata sanctus est Lorem ipsum dolor sit amet.

Duis autem vel eum iriure dolor in hendrerit in vulputate velit esse molestie consequat, vel illum dolore eu feugiat nulla facilisis at vero eros et accumsan et iusto odio dignissim qui blandit praesent luptatum zzril delenit augue duis dolore te feugait nulla facilisi. Lorem ipsum dolor sit amet, consectetuer adipiscing elit, sed diam nonummy nibh euismod tincidunt ut laoreet dolore magna aliquam erat volutpat.

Ut wisi enim ad minim veniam, quis nostrud exerci tation ullamcorper suscipit lobortis nisl ut aliquip ex ea commodo consequat. Duis autem vel eum iriure dolor in hendrerit in vulputate velit esse molestie consequat, vel illum dolore eu feugiat nulla facilisis at vero eros et accumsan et iusto odio dignissim qui blandit praesent luptatum zzril delenit augue duis dolore te feugait nulla facilisi.

Nam liber tempor cum soluta nobis eleifend option congue nihil imperdiet doming id quod mazim
placerat facer possim assum. Lorem ipsum dolor sit amet, consectetuer adipiscing elit, sed diam
nonummy nibh euismod tincidunt ut laoreet dolore magna aliquam erat volutpat. Ut wisi enim ad minim
veniam, quis nostrud exerci tation ullamcorper suscipit lobortis nisl ut aliquip ex ea commodo.



\printbibliography
%%%%%%%%%%%%%%%%%%%%%%%%%%%%%%%%%%%%%%%%%%%%%%%%%%%%%%%%%%%%%%%%%%%%%%%%%%%%%%%

\chapter{Umsetzung der Webseite}
Die Digitalisierung hat enorme Auswirkungen auf Bildung und Lehre, und Hochschulen bilden da keine Ausnahme. Die DHBW Karlsruhe setzt sich, wie viele andere Bildungseinrichtungen auch, für die Nutzung digitaler Technologien ein, um das Lernen und die Bewältigung des Hochschullebens zu erleichtern. In diesem Zusammenhang entstand die Idee einer modernen Webseite, die speziell auf die Bedürfnisse der Studierenden der DHBW Karlsruhe zugeschnitten ist. Diese moderne Webseite soll die wichtigsten Informationen, die Studierende benötigen, auf effiziente und benutzerfreundliche Weise zusammenfassen.\\
\section{Ziel der Webseite}
Das Hauptziel der App besteht darin, den Informationsfluss zu optimieren und den Zugriff auf relevante Daten zu erleichtern. Ob Mensa-Speiseplan, aktuelle Vorlesung, Wochenplan oder eine Ansammlung von Links. Die Webseite sollte all diese Informationen an einem  zentralen Ort bereitstellen. Darüber hinaus möchten wir eine intuitive und ansprechende Benutzererfahrung bieten, die den Benutzern den Zugriff und die Navigation erleichtert.\\

\subsection{Wissenschaftliche Erkenntnisse} 
Zu den drei Hauptinformationen, die den Nutzern der DHBW Karlsruhe am wichtigsten sind, gehören Speisepläne, Stundenpläne und eine Sammlung nützlicher Links.\\
Der Kantinenplanung kommt eine große Bedeutung zu, da sie es Studierenden der DHBW Karlsruhe ermöglicht über den Tagesplan Bescheid zu wissen.\\
Auch Stundenpläne spielen eine wichtige Rolle, da sie den Studierenden helfen, ihre Kurse und Vorlesungen zu überschauen  und einen Überblick über ihre wöchentlichen Aktivitäten zu erhalten.\\ 
Darüber hinaus ist die Sammlung nützlicher Links eine wertvolle Ressource. Über diese Links erhalten Studierende einen einfachen Zugriff auf wichtige Online-Ressourcen und -Dienste, die sie regelmäßig nutzen. Dies ermöglicht einen schnellen Zugriff auf Informationen und unterstützende Materialien im Zusammenhang mit Ihren Lernbedürfnissen.\\ 
Insgesamt sind Mensapläne, Stundenpläne und eine Sammlung weiterführender Links die drei Hauptinformationen, die wir auf Basis unserer Erfahrung  für Studierende der DHBW Karlsruhe als am wichtigsten erachten. Durch die Optimierung dieser Informationen können Benutzer ihre täglichen Mahlzeiten einsehen, Unterrichtspläne effektiv überblicken und auf wichtige Online-Ressourcen zugreifen, um ihre Lernanforderungen zu erfüllen.\\
Die effektivsten Informationsquellen sind diejenigen, die den Benutzern einen schnellen und einfachen Zugriff auf die benötigten Informationen ermöglichen. In diesem Zusammenhang sind moderne  Webseiten , die auch für mobile Geräte optimiert ist, eine ideale Lösung, da sie die Möglichkeit bieten, alle notwendigen Informationen an einem Ort zu sammeln und sie den Benutzern jederzeit und überall zur Verfügung zu stellen.\\
Das Hauptproblem des Informationsangebots der DHBW Karlsruhe besteht in der Fragmentierung  und dem erschwerten Zugriff auf spezifische Informationen, da diese auf verschiedenen Plattformen verbreitet sind. Unser Vorschlag zur Lösung dieser Probleme ist die Entwicklung mobiler Webseiten. Diese Anwendung verbessert das Benutzererlebnis, indem sie die benötigten Informationen zentralisiert und es Benutzern ermöglicht, die benötigten Informationen schnell und einfach zu finden und zu verwenden.\\ 

\newpage
\section{Grundkonzept der Webseite}
Als mobile Informationsplattform konzipiert, zielt die Webseite darauf ab, die Informationsbeschaffung für Studierende der DHBW Karlsruhe zu optimieren und zu zentralisieren. Durch die Bündelung einer Vielzahl zusammengehöriger Dienste und Informationen und deren Darstellung auf einer intuitiven Benutzeroberfläche wollen wir den Informationsfluss optimieren und gleichzeitig einen hohen Komfort gewährleisten.
\subsection{Benutzeroberfläche}
Das Design der Benutzeroberfläche folgt den Designprinzipien Usability und User Experience (UX)  mit dem Ziel,  intuitive und benutzerfreundliche Anwendungen zu entwickeln\cite{hartmann2017usability}. Diese Prinzipien sind  für die Erstellung effizienter und effektiver Anwendungen, die die Benutzerzufriedenheit erhöhen, von wesentlicher Bedeutung\cite{14all}. Es orientiert sich an zeitgenössischen Designstandards und ist klar und minimalistisch gestaltet,da es dazu beiträgt, dass die Benutzeroberfläche übersichtlich und selbsterklärend bleibt\cite{massiveart}
Die wichtigsten Informationen und Dienste wie Speisepläne, Fahrpläne und Links sind direkt über die Startseite der Webseite zugänglich und werden mit sofort erkennbarer Ikonographie angezeigt..Dies erleichtert Benutzern die Navigation auf Ihrer Website und das Auffinden der benötigten Informationen. Durch die  Verwendung von Ikonografie zur Darstellung dieser Dienste sind diese sofort erkennbar und lassen sich in die Anwendung einfacher verwenden\cite{99designs}. 



\subsection{Navigation und Struktur}
Die Anwendung verwendet ein Seitenmenü zur Navigation. Dieses Menü enthält Symbole, die die Hauptbereiche der Anwendung darstellen, wie z. B. den Speiseplan, den Stundenplan und verschiedene Links.Durch Auswahl eines dieser Symbole gelangt der Benutzer direkt zu den entsprechenden Informationen. Diese Navigationsverwendung sorgt für eine uneingeschränkte  Anzahl von Menüpunkten, die sortierung der Menüpunkte nach Wichtigkeit un der Inhalt der Webseite beginnt direkt am oberen Rand und hat keine Einschränkung\cite{eology2023}.
Darüber hinaus ist die Struktur der Webseite logisch und hierarchisch aufgebaut, mit Seiten, die weitere Details oder spezifische Informationen bieten. Benutzer können zwischen diesen Seiten navigieren, indem sie wischen oder die entsprechenden Optionen in den Menüs auswählen.

\subsection{Funktion und Komponenten}
Die Webseite enthält verschiedene Funktionen und Komponenten, die darauf abzielen, Informationen effektiv anzuzeigen und gleichzeitig ein hohes Maß an Benutzerfreundlichkeit zu gewährleisten. 

Zu den Hauptmerkmalen gehören: 
\begin{itemize}
	\item Speiseplan: Diese Funktion zeigt das Tagesmenü an. Benutzer können sich auch zukünftige Menüs anzeigen lassen.
	\item Stundenplan: Eine Funktion, die den Stundenplan anzeigen lässt.
	\item Links: Diese Funktion bietet eine Sammlung verwandter Links, die Benutzern den einfachen Zugriff auf wichtige Online-Ressourcen und -Dienste ermöglichen.
\end{itemize}
\newpage
\section{Website-Komponenten}
In diesem Kapitel werden die einzelnen Funktionen und Komponenten erläutert-
\subsection{Home}
\subsubsection{Konzept}
Die Homepage ist als zentrale Informations- und Navigationsplattform konzipiert.Diese bietet einen  Überblick über die wichtigsten Informationen und Funktionen und dient gleichzeitig als Ausgangspunkt für die Navigation zu spezifischeren Seiten und Funktionen.  
Die Homepage ist in drei Hauptbereiche unterteilt.
\begin{itemize}
	\item Header:Die Kopfzeile enthält das Logo der Webseite sowie ein die Copyright Bezeichnung auf der rechten Seite. 
	\item Navigation: Die Navigation ist  in vertikaler Form am rechten Rand der Webseite konzipiert. Die enthält die Menüitems mit Symbolen, die den Hauptfunktionen der Webseite entsprechen. Durch Klicken auf diese Symbole gelangt der Benutzer direkt zur entsprechenden Seite.
	\item  Hauptbereich: Im Hauptbereich werden die Informationen angezeigt, die aktuell am relevantesten sind. Dazu gehören <die Angabe der jetzigen Vorlesung, der Mensaplan für den aktuellen Tag und eine Information für die aktuele Zeit.
\end{itemize} 
In dieser Abbildung ist die Visualisierung der Homepage dargestellt.\newpage
\begin{figure}[htbp]
	\centering
	\fbox{\includegraphics[height=0.3\textheight]{images/homepage}}
	\caption{Homepage}
\end{figure}
\subsubsection{Implementierung}
In diesem Codeblock befindet sich die Homepage-Komponente:\\
\begin{lstlisting}[language=JavaScript,
	frame=single,           % Ein Rahmen um den Code
	framexleftmargin=15pt,  % Rahmen link von den Zahlen
	style=algoBericht,
	label={Homepage-Komponente},
	captionpos=b ,          % Caption unter den Code setzen
	caption={Homepage-Komponente}]
import React, { useState, useEffect } from 'react';
import './home.css';
import SchedulerNow from '../Scheduler/ScheduleNow';
import FoodNow from '../Food/FoodNow';

function Homepage() {
    const [time, setTime] = useState(new Date());
    const [currentEvent, setCurrentEvent] = useState(null);
    useEffect(() => {
    	const interval = setInterval(() => {
          setTime(new Date());}, 1000);
        return () => clearInterval(interval);
     }, []);
    const days = ['Sonntag', 'Montag', 'Dienstag', 
    'Mittwoch', 'Donnerstag', 'Freitag', 'Samstag'];
    const today = days[time.getDay()];
    const date = time.toLocaleDateString();
    return (
    <div className="homepage">
    <div className="header">
    <h1>Willkommen bei DHBW-Star</h1>
    </div>
    <div className="current-lecture">
    <h2>Aktuelle Vorlesung</h2>
    <SchedulerNow setCurrentEvent={setCurrentEvent} />
    </div>
    <div className="todays-food">
    <h2>Heutiges Essen</h2>
    <FoodNow />
    </div>
    <div className="date-time">
    <h2>{today}, den {date}</h2>
    <h2>{time.toLocaleTimeString()}</h2>
    </div>
    </div>
    );
}
export default Homepage;
	
\end{lstlisting}

Die Homepage-Implementierung der DHBW Star-Webseite basiert auf der Nutzung grundlegender und erweiterter Funktionen der React-Bibliothek zur Erstellung der Benutzeroberfläche. Dieser Code verwendet React-Funktionen und Hooks, um eine dynamische und reaktionsfähige Benutzeroberfläche zu erstellen.\\ 
Die Hauptkomponente \emph{Homepage} wird als funktionale Komponente definiert.Funktionale Komponenten sind in modernen React-Anwendungen weit verbreitet, da sie einfacher zu lesen und zu testen sind und React-Hooks verwenden können.\\
Der Komponentenstatus wird über den \emph{useState}-Hook verwaltet. Dieser Hook ermöglicht es Komponenten, ihren eigenen Status beizubehalten und zu aktualisieren. In diesem Fall werden zwei Zustandsvariablen definiert: \emph{time} und \emph{currentEvent}.\\
Zur Darstellung der aktuellen Uhrzeit wird eine Zeitvariable verwendet, die jede Sekunde aktualisiert wird. Dies wird durch die Verwendung des \emph{useEffect}-Hooks erreicht. Dieser Hook führt eine Funktion aus, sobald die Komponente gerendert wird und wann immer sich der Status der Komponente ändert. Diese Funktion legt das Intervall zwischen Aufrufen der \emph{setTime}-Funktion jede Sekunde fest und aktualisiert die Zeitvariable. Die Variable \emph{currentEvent} wird verwendet, um das aktuell auftretende Vorlesung darzustellen. Dies wird durch Einbinden der \emph{SchedulerNow}-Komponente  und Übergeben der \emph{setCurrentEvent}-Funktion aktualisiert.\\ 
Die Homepage-Implementierung ist auf eine klare und einheitliche Struktur ausgelegt, sodass Nutzer auf einen Blick die wichtigsten Informationen wie aktuelle Uhrzeit, aktuelle Vorlesung, heutige Mahlzeit erkennen können. Darüber hinaus erleichtert diese Struktur die Erweiterung und Änderung der Homepage, wenn Sie in Zukunft weitere Funktionen oder Informationen hinzufügen müssen. React und seine Hooks ermöglichen eine effiziente Statusverwaltung und eine reaktionsfähige Benutzeroberfläche, die  automatisch aktualisiert wird, wenn sich der Status ändert.
\subsubsection{Programmablaufplan Homepage}
In dieser Abbildung wird der Ablauf der Komponenten visualisiert:
\begin{figure}[htbp]
	\centering
	\fbox{\includegraphics[height=0.5\textheight]{images/PAPHomepage}}
	\caption{PAP Homepage}
\end{figure}

\subsection{Food}
\subsubsection{Konzept}
Die Food-Komponente der DHBW Star-Webseite ist ein wichtiger Teil der Anwendung,die Speisepläne anzeigt.
Die Daten werden von einer Mensa-APi abgerufen die in einem Docker-Container erstellt wird. Dieser Vorgang wird in dem Kapitel Docker näher beschrieben.
Der Default-Speiseplan ist immer der aktuelle Tag, zu dem werden für die nächsten zehn Tage die Speisepläne angezeigt. Außer an Wochenenden, da ist die Mensa geschlossen.
In diesem Programmablaufplan ist die Food-Komponente visuell erklärt.\\
\begin{figure}[htbp]
	\centering
	\fbox{\includegraphics[height=0.65\textheight]{images/PAPFood}}
	\caption{PAP Food}
\end{figure}
 \newpage
 Zudem ist in der nächsten Abbildung die Webseite der Food-Komponente abgebildet.
 \begin{figure}[htbp]
 	\centering
 	\fbox{\includegraphics[height=0.3\textheight]{images/Food}}
 	\caption{Food Website}
 \end{figure}
\subsubsection{Implementierung}
Die Implementierung der Foodkomponente der DHBW Star-Webseite basiert auf der Verwendung grundlegender und erweiterter Funktionen der React- und \emph{Axios}-Bibliotheken zur Verarbeitung von HTTP-Anfragen.
Die \emph{Food}-Komponente wird als funktionelle Komponenten definiert und nutzt React-Hooks, um den internen Zustand und Nebenwirkungen zu verwalten. Mit dem Hook \emph{useState} werden drei Statusvariablen definiert: \emph{dates}, \emph{activeTab} und \emph{meals}. \emph{Dates} speichert  Daten, an denen Mahlzeiten verfügbar sind, \emph{activeTab} speichert den Index des aktuell ausgewählten Datums und \emph{Meals} speichert Mahlzeitinformationen für das aktuell ausgewählte Datum. Der \emph{useEffect}-Hook wird zweimal verwendet. Einmal, um das Datum abzurufen, an dem die Komponente zum ersten Mal gerendert wird, und einmal, um die Informationen zur Essenszeit abzurufen, wenn sich das aktive Datum ändert. Die Funktionen \emph{fetchDates} und \emph{fetchMeals} verwenden die \emph{Axios}-Bibliothek, um eine HTTP-GET-Anfrage an den Server zu senden, um Daten abzurufen. 
Informationen zu Mahlzeiten werden angezeigt, indem das Array \emph{Meals} durchlaufen und  für jede Mahlzeit ein JSX-Element generiert wird. Es verwendet die Funktion \emph{getEmoji}, um passende Emojis basierend auf Essensklassifikatoren zu generieren.\\
Dies ist der dazugehörige Code:
\newpage
\begin{lstlisting}[language=JavaScript,
	frame=single,           % Ein Rahmen um den Code
	framexleftmargin=15pt,  % Rahmen link von den Zahlen
	style=algoBericht,
	label={Food-Komponente},
	captionpos=b ,          % Caption unter den Code setzen
	caption={Food-Komponente}]
import React, { useState, useEffect } from "react";
import axios from "axios";
import "./Food.css";

const getEmoji = (classifier,name) => {
     switch (classifier) {
     	case "S":
        return "\emoji{pig face}";
        case "SAT":
        return "\emoji{pig face}";
        case "R":
        return "\emoji{cow face}";
        case "RAT":
        return "\emoji{cow face}";
        case "MSC":
        return "\emoji{fish}";
        case "VEG":
        return "\emoji{broccoli}";
        case "VG":
        return "\emoji{seedling}";
        case "N":
        return "\\emoji{person gesturing NO}";
        default:
        return "\emoji{forkandknifewithplate️}";
    }
};
const Food = () => {
    const [dates, setDates] = useState([]);
    const [activeTab, setActiveTab] = useState(0);
    const [meals, setMeals] = useState([]);
    
    useEffect(() => {
    fetchDates();
     }, []);
	
    useEffect(() => {
        if (dates.length > 0) {
        fetchMeals(dates[activeTab]);
    }
    }, [dates, activeTab]);
	
    const fetchDates = async () => {
       try {
         const response = await axios.get(process.env.
         REACT_APP_MENSA_ADDRESS+"/plans");
         const dateList = response.data.data.map((date) =>
         `{date.date.year}-{(date.date.month + 1).toString()
         	.padStart(2, "0")}-{date.date.day.toString()
         	.padStart(2, "0")}`
          ).slice(0, 10);
          setDates(dateList);
         } catch (error) {
		  console.error("Error fetching dates:", error);
			}
		};
	
      const fetchMeals = async (date) => {
         try {
            const response = await axios.get(
            process.env.REACT_APP_MENSA_ADDRESS+
            `/plans/{date}?canteens=erzberger`
			);
            let mealData = response.data.data[0].lines
            .slice(0, 3)
            .map((line) => ({
                main: line.meals[0],
                description: line.meals[1]?.name,}));
            if (!mealData[0]['main']){
            	console.log("FEIERTAG")
            	mealData=[{
            	    main: {
            	        classifiers: 'Nein',
            	        name:"Feiertag/kein Essen",
            	        price:"0 Euro"
                    },
                    description: "An dem heutigen Tag ist 
                    die Mensa geschlossen."}]
                }
            setMeals(mealData);
        } catch (error) {
           console.error("Error fetching meals:", error);
	    }
    };
    
    return (
    <div className="mensa-plan">
    <div className="tabs">
    {dates.map((date, index) => (
        <button
        key={index}
        className={`tab {index === activeTab ? "active" : ""}`}
        onClick={() => setActiveTab(index)}
        >
        {date}
        </button>
        ))}
    </div>
    <div className="tab-content">
    {meals.map((mealObj, index) => {
            const meal = mealObj.main;
            if (meal?.empty) {
                return (
                <div key={index} className="no-meal">
                <h3>Feiertag/Kein Essen</h3>
                </div>
                );
            } else {
            return (
            <div key={index} className="meal">
            <h3>
            {getEmoji(meal.classifiers[0],meal.name)} {meal.name}
            </h3>
            <p>{mealObj.description}</p>
            <p>{meal.price}</p>
            </div>
            );}
        })}
    </div>
	</div>
	);
};

export default Food;

\end{lstlisting}

Die Implementierung der \emph{Food}-komponente wurde so konzipiert, dass Datenerfassung und -präsentation sauber getrennt werden. Mit React-Hooks ist es möglich den Komponentenstatus effizient zu verwalten und die Benutzeroberfläche automatisch zu aktualisieren, wenn sich der Status ändert. Die \emph{Axios}-Bibliothek macht die Bearbeitung von HTTP-Anfragen einfach und zuverlässig. Diese Struktur ermöglicht auch eine einfache Anpassung und Erweiterung der Komponente, wenn in Zukunft zusätzliche Funktionen oder Änderungen erforderlich sind.

\subsubsection{FoodNow}
Die \emph{FoodNow}-Komponente der DHBW-Star-Webseite wurde entwickelt, um Nutzern aktuelle Informationen zur Essensverfügbarkeit in der Mensa zur Verfügung zu stellen. Es wurde mithilfe der React- und Axios-Bibliotheken zur Verwaltung von HTTP-Anfragen implementiert. 
Diese Komponente besteht aus Funktionskomponenten, die mithilfe der \emph{useState}- und \emph{useEffect}-Hooks von React den Status verwalten und Nebenwirkungen behandeln. Es werden zwei Zustandsvariablen definiert. Das eine ist \emph{meals}, das die aktuell verfügbaren Mahlzeiten speichert, und das andere ist \emph{mensaStatus}, das den aktuellen Status des Restaurants speichert (geöffnet oder kurz vor der Eröffnung). 
Der \emph{useEffect}-Hook wird verwendet, um die \emph{fetchMeals}-Funktion auszulösen, sobald die Komponente in der Benutzeroberfläche gerendert wird. Diese Funktion ruft zunächst die Funktion \emph{isMensaOpen} auf, um zu sehen, ob die Mensa geöffnet ist oder kurz vor der Eröffnung steht. Diese Funktion überprüft das aktuelle Datum und die aktuelle Uhrzeit und vergleicht sie mit den  Öffnungs- und Schließzeiten des angegebenen Restaurants. 
Wenn die Mensa geöffnet ist oder kurz vor der Eröffnung steht, wird eine HTTP-GET-Anfrage an den Mensa-Server gesendet, um die verfügbaren Mahlzeiten für den  Tag abzurufen. Erhaltene Mahlzeiten werden  im Status \emph{meals} gespeichert und der Status der Mensa wird in \emph{mensaStatus} gespeichert. 
Die Rendermethode der Komponente prüft, ob die Cafeteria geöffnet ist oder kurz vor der Eröffnung steht und stellt die verfügbaren Mahlzeiten entsprechend dar oder zeigt eine Meldung an, dass die Cafeteria geschlossen ist oder kurz vor der Eröffnung steht. Die Implementierung der \emph{FoodNow}-Komponente auf diese Weise ermöglicht eine effiziente Handhabung von Statusänderungen und Nebenwirkungen und hält die Benutzeroberfläche auf dem neuesten Stand. Die Axios-Bibliothek erleichtert Komponenten das Senden und Empfangen von HTTP-Anfragen. Diese Implementierung trägt dazu bei, dass der Code sauber und  organisiert bleibt, wodurch er einfacher zu lesen und zu warten ist.
In diesem Codeblock befindet sich der dazugehörige Code:
\begin{lstlisting}[language=JavaScript,
	frame=single,           % Ein Rahmen um den Code
	framexleftmargin=15pt,  % Rahmen link von den Zahlen
	style=algoBericht,
	label={FoodNow-Komponente},
	captionpos=b ,          % Caption unter den Code setzen
	caption={FoodNow-Komponente}]
import React, { useState, useEffect } from "react";
import axios from "axios";
import "./Food.css";
const getEmoji = (classifier, name) => {
  switch (classifier) {
  	case "S":
  	return "🐷";
  	case "SAT":
  	return "🐷";
  	case "R":
  	return "🐮";
  	case "RAT":
  	return "🐮";
  	case "MSC":
  	return "🐟";
  	case "VEG":
  	return "🥦";
  	case "VG":
  	return "🌱";
  	case "N":
  	return "🙅";
  	default:
  	return "🍽️";
 }};

const FoodNow = () => {
   const [meals, setMeals] = useState([]);
   const [mensaStatus, setMensaStatus] = 
   useState({ isOpen: false, isBeforeOpening: false });
   
   useEffect(() => {
   fetchMeals();}, []);

   const isMensaOpen = () => {
    const now = new Date();
    const day = now.getDay();
    const time = now.getHours() * 60 + now.getMinutes();
    const openingTime = 11 * 60 + 15;
    const closingTime = 13 * 60 + 30;
 
    const beforeOpening = time < openingTime;
    return {
    	isOpen: day >= 1 && day <= 5 && time >= openingTime 
    	&& time <= closingTime,
    	isBeforeOpening: beforeOpening,};
    };
   const fetchMeals = async () => {
   	 try {
   	  const mensaStatus = isMensaOpen();
   	  if (mensaStatus.isOpen || mensaStatus.isBeforeOpening) {
   	    const today = new Date().toISOString().slice(0, 10);
   	    const response = await axios.get(
   	    process.env.REACT_APP_MENSA_ADDRESS+
   	    `/plans/${today}?canteens=erzberger`);
   	    let mealData = response.data.data[0].lines.slice(0, 3)
		.map((line) => ({
		  main: line.meals[0],
		  description: line.meals[1]?.name,}));
		 if (!mealData[0]["main"]) {
		 	console.log("FEIERTAG");
		 	mealData = [{
		 		main: {
		 		  classifiers: "Nein",
		 		  name: "Feiertag/kein Essen",
		 		  price: "0",},
	 		    description: "An dem heutigen Tag ist 
	 		    die Mensa geschlossen.",},];
 		    }
 	     setMeals(mealData);
 	     setMensaStatus(mensaStatus);
      } else {
         setMensaStatus({ isOpen: false, isBeforeOpening: false });}
	} catch (error) {
	    console.error("Error fetching meals:", error);}
   };

   return (
   <div className="mensa-plan">
   <div className="tab-content">
   {mensaStatus.isOpen || mensaStatus.isBeforeOpening ? (
   	 meals.length > 0 ? (
   	 meals.map((mealObj, index) => {
   	   const meal = mealObj.main;
   	   return (
   	   <div key={index} className="meal">
   	   <h3>
   	   {getEmoji(meal.classifiers[0], meal.name)}{meal.name}
   	   </h3>
   	   <p>{mealObj.description}</p>
   	   <p>{meal.price}</p>
   	   </div>
	);})
	) : (
	  <div className="no-meal">
	  <h3>Keine Informationen fuer 
	  heute verfuegbar</h3>
	  </div>)
	) : (
	  <div className="no-meal">
	  <h3>Die Mensa ist geschlossen</h3>
	  </div>)}
   {mensaStatus.isBeforeOpening && (
   	<div className="before-opening">
   	<h3>Die Mensa ist noch geschlossen. 
   	Sie oeffnet um 11:15 Uhr.</h3>
   	</div>)}
	</div>
	</div>
	);
};
export default FoodNow;
	
\end{lstlisting}
\newpage
\subsection{Scheduler}
\subsubsection{Konzept}
Die \emph{Scheduler}-Komponente wurde entwickelt, um Nutzern einen Überblick über geplante Vorlesungen in der DHBW-Star-Webseite zu geben. Es wurde mithilfe der React-Bibliothek, der Axios-Bibliothek zum Verwalten der HTTP-Anfragen und der FullCalendar-Bibliothek zum Rendern der Ereignisse im Kalender implementiert.\\
Die Webseite der \emph{Scheduler}-Komponente ist in dieser Abbildung dargestellt.\\
\begin{figure}[htbp]
	\centering
	\fbox{\includegraphics[height=0.35\textheight]{images/Scheduler}}
	\caption{Scheduler}
\end{figure}
\newpage
Die Grundschritte der Implementierung werden in diesem Programmablaufplans visualisiert um die Grundfunktion verständlicher zu machen.
\begin{figure}[htbp]
	\centering
	\fbox{\includegraphics[height=0.65\textheight]{images/PAPScheduler}}
	\caption{PAP Scheduler}
\end{figure}
\newpage
\subsubsection{Implementierung}
Das ist der Code für die \emph{Scheduler}-Komponente:
\begin{lstlisting}[language=JavaScript,
	frame=single,           % Ein Rahmen um den Code
	framexleftmargin=15pt,  % Rahmen link von den Zahlen
	style=algoBericht,
	label={Scheduler-Komponente},
	captionpos=b ,          % Caption unter den Code setzen
	caption={Scheduler-Komponente}]
import React, { useState, useEffect } from 'react';
import axios from 'axios';
import FullCalendar from '@fullcalendar/react';
import dayGridPlugin from '@fullcalendar/daygrid';
import timeGridPlugin from '@fullcalendar/timegrid';
import interactionPlugin from '@fullcalendar/interaction';
import './Scheduler.css';
import ICAL from 'ical.js';
import '@fullcalendar/core/locales-all';
import { isWithinInterval } from 'date-fns';

const Scheduler = ({ setCurrentEvent }) => {
    const [events, setEvents] = useState([]);
    
    useEffect(() => {
         const fetchData = async () => {
            try {
               const response = await 
               axios.get(
               process.env.REACT_APP_PROXY_ADDRESS+'/schedule');
               const data = response.data;
               const parsedEvents = parseEvents(data);
               setEvents(parsedEvents);
               const currentEvent = getCurrentEvent(parsedEvents);
               setCurrentEvent(currentEvent);
           } catch (error) {
               console.error(
               'Error fetching schedule data:', error);
           }};
       fetchData();
   }, [setCurrentEvent]);

       const getCurrentEvent = (events) => {
       	  const now = new Date();
       	  const currentEvent = events.find(
       	  (event) =>
       	  isWithinInterval(now, 
       	  { start: event.start, end: event.end }) &&
       	  now.getTime() <= event.end.getTime()
       	  );
       	  return currentEvent;
       };
      
      const parseEvents = (icalData) => {
        const jcalData = ICAL.parse(icalData);
        const comp = new ICAL.Component(jcalData);
        const events = comp.getAllSubcomponents('vevent');
        const now = new Date();
        const startDate = ICAL.Time.fromJSDate(
        new Date(now.getFullYear() - 1, 
        now.getMonth(), now.getDate()));
        const endDate = ICAL.Time.fromJSDate(
        new Date(now.getFullYear() + 1, 
        now.getMonth(), now.getDate()));
        const parsedEvents = [];
        events.forEach(eventComponent => {
           const event = new ICAL.Event(eventComponent);
           const start = event.startDate;
           const end = event.endDate;
           
           if (event.isRecurring()) {
           	const iterator = event.iterator();
           	let next;
            while ((next = iterator.next())) {
                if (next.compare(startDate) < 0) {
                    continue;}
                if (next.compare(endDate) > 0) {
                	break;}
                const duration = end.subtractDate(start);
                const eventStart = next;
                const eventEnd = eventStart.clone();
                eventEnd.addDuration(duration);
                    parsedEvents.push({
                       title: event.summary,
                       start: eventStart.toJSDate(),
                       end: eventEnd.toJSDate(),
                       location: event.location ? 
                       event.location.value : '',
                       description: event.description,});}
             } else {
               parsedEvents.push({
                title: event.summary ,
               	start: start.toJSDate(),
               	end: end.toJSDate(),
               	location: event.location,
               	description: event.description,
               });
           }
       });
        console.log(parsedEvents);
        return parsedEvents;
    };
    return (
    <div className="scheduler">
    <FullCalendar
    plugins={[dayGridPlugin, timeGridPlugin, interactionPlugin]}
    initialView="timeGridWeek"
    headerToolbar={{
    	left: 'prev,next today',
    	center: 'title',
    	right: 'timeGridWeek,timeGridDay',}}
    events={events}
    eventColor="#007bff" //// Blaue Akzente
    locale="de"
    allDaySlot={false} // Entfernt die Anzeige fuer "all day"-Events
    nowIndicator={true} 
    // Fuegt einen roten Zeitstrahl fuer die aktuelle Zeit hinzu
    weekends={false} // Versteckt Wochenenden
    slotMinTime="07:00:00" // Startzeit fuer Zeitslots
    slotMaxTime="18:00:00" // Endzeit fuer Zeitslots
    />
    </div>
    );
};
export default Scheduler;
\end{lstlisting}
Die \emph{Scheduler}-Komponente besteht aus Funktionskomponenten, die die React-Hooks \emph{useState} und \emph{useEffect} verwenden, um den Status zu verwalten und Nebenwirkungen zu behandeln. Der \emph{useState}-Hook wird verwendet, um den Status des im Kalender angezeigten Ereignisses zu speichern,  und der \emph{useEffect}-Hook wird verwendet, um die \emph{fetchData}-Funktion auszulösen, sobald die Komponente in der Benutzeroberfläche angezeigt wird.\\
Die \emph{fetchData}-Funktion sendet eine HTTP-GET-Anfrage an den Server,um \emph{ical}-Daten für geplante Ereignisse abzurufen. Diese Daten werden dann mithilfe der \emph{parseEvents}-Funktion in ein Format konvertiert, das von der \emph{FullCalendar}-Bibliothek verstanden wird. Diese Funktion verwendet die \emph{ICAL.js}-Bibliothek, um die \emph{ical}-Daten zu analysieren und in ein JavaScript-Objekt zu konvertieren. Es berücksichtigt auch wiederkehrende Ereignisse und generiert für jede Wiederholung innerhalb eines bestimmten Zeitraums ein eigenes Ereignis.\\
Die Funktion \emph{getCurrentEvent} wird verwendet, um das aktuelle Ereignis abzurufen, also das Ereignis, das gerade auftritt. Diese durchsucht die Liste der Ereignisse nach Ereignissen, deren Start- und Endzeit das aktuelle Datum enthalten. Dieses Ereignis wird zur Anzeige durch eine andere Komponente in der Anwendung an die übergeordnete Komponente zurückgegeben.\\
Schließlich gibt die \emph{Scheduler}-Komponente eine \emph{FullCalendar}-Komponente zurück, die mit  entsprechenden Eigenschaften konfiguriert ist, um die Ereignisse benutzerfreundlich anzuzeigen.Die Komponente nutzt mehrere Plugins von FullCalendar, um unterschiedliche Ansichten und Interaktionen zu ermöglichen.\\
Diese Art und Weise der Implementierung bietet eine effiziente und flexible Möglichkeit, geplante Ereignisse anzuzeigen. Es stellt sicher, dass die Daten immer auf dem neuesten Stand sind und ermöglicht Benutzern eine einfache und intuitive Übersicht ihrer Veranstaltungen. Die Verwendung etablierter Bibliotheken wie \emph{Axios}, \emph{FullCalendar} und \emph{ICAL.js} reduziert die Codekomplexität und erhöht die Zuverlässigkeit und Wartbarkeit.
\subsubsection{Proxy-Server}

Der Proxy-Server ist so konzipiert, dass er die Cross-Origin Resource Sharing (CORS)-Richtlinie umgeht, die von APIs auferlegt werden. Eine CORS-Richtlinie ist eine Sicherheitsmaßnahme, die von Browsern angewendet wird, um  unerwünschten Datenzugriff über verschiedene Domänen hinweg zu verhindern. In diesem Fall verbietet die CORS-Richtlinie der API \emph{'http://rapla.dhbw-karlsruhe.de/rapla?page=iCal\&user=vollmer\&file=tinf20b3'} den direkten Clientzugriff auf die Kalenderdaten. Um dieses Problem zu lösen, wurde ein Proxy-Server entwickelt, der als Vermittler zwischen Clients und API/Server fungiert. Dieses Bild veranschaulicht die Rolle des Proxy-Servers genauer.
\begin{figure}[htbp]
	\centering
	\fbox{\includegraphics[height=0.25\textheight]{images/Proxyserver}}
	\caption{Proxy-Server}
\end{figure}

Der Proxy-Server wurde in \emph{Node.js} mithilfe des Express-Frameworks implementiert, das ein benutzerfreundliches Framework zum Erstellen von Webanwendungen und APIs bietet. Zudem wurde das CORS-Paket importiert, um \emph{Axios} zum Senden von HTTP-Anfragen an CORS-Funktionen und APIs bereitzustellen.\\
Ein Cache-Objekt wurde eingeführt, um  Antwortdaten für einen bestimmten Zeitraum (in diesem Fall 1 Stunde) zu speichern und unnötige Anfragen an die API zu vermeiden. Diese Cache-Implementierung trägt dazu bei, die API-Last zu reduzieren und die Anwendungsleistung zu verbessern, indem  die Anzahl der an die API gesendeten Anforderungen reduziert wird. Wenn die Anfrage innerhalb des Cache-Zeitraums wiederholt wird, werden die zwischengespeicherten Daten an den Client gesendet, ohne erneut auf die API zuzugreifen.\\
Die Proxy-Anwendung stellt einen einzelnen Endpunkt \emph{/schedule} bereit, um Kalenderdaten auf Anfrage asynchron abzurufen. Zunächst prüft es, ob sich die Daten im Cache befinden und innerhalb des Cache-Zeitraums noch  gültig sind. Falls vorhanden, werden die zwischengespeicherten Daten an den Client gesendet. Andernfalls wird eine Anfrage an die API gesendet, um die neuesten Kalenderdaten abzurufen.Antwortdaten werden zwischengespeichert und an den Client gesendet.\\
Der Proxy-Server läuft auf Port 3002 oder einem anderen Port, der durch die Umgebungsvariable \emph{PROXYPORT} angegeben wird. Der Proxy-Server ermöglicht es der Scheduler-Anwendung, CORS-Richtlinien zu umgehen und auf die Kalenderdaten der API zuzugreifen, um sie im Client darzustellen.\
Dies ist der zu entsprechende Code:
\begin{lstlisting}[language=JavaScript,
	frame=single,           % Ein Rahmen um den Code
	framexleftmargin=15pt,  % Rahmen link von den Zahlen
	style=algoBericht,
	label={Proxy-Server},
	captionpos=b ,          % Caption unter den Code setzen
	caption={Proxy-Server}]
const express = require('express');
const cors = require('cors');
const axios = require('axios');
const cach = {}
const cachDuration = 3600000; //1h
const kurs = 'tinf20b3'
const app = express();
app.use(cors());
app.get('/schedule', async (req, res) => {
  if (kurs in cach && cach['tinf20b3']
  .time > new Date().getTime())
  { console.log("Get cached Data");
  	res.send(cach['tinf20b3'].data);}
  else{
    try {
      const response = await axios.get(
      'http://rapla.dhbw-karlsruhe.de/rapla?
      page=iCal&user=vollmer&file=tinf20b3');
      cach[kurs] = {
      	time: new Date().getTime() + cachDuration,
      	data: response.data};
      console.log("save Data in cache:" + cach['tinf20b3'].time);
      res.send(response.data);} 
   catch (error) {res.status(500).send({ error: 
   	   'An error occurred while fetching the data.' });}}});
const PORT = process.env.PROXY_PORT || 3002;
app.listen(PORT, () => console.log(`Proxy 
server running on port ${PORT}`));

\end{lstlisting}
\subsubsection{SchedulerNow}
Die Komponente \emph{SchedulerNow} ist eine spezielle Komponente, die entwickelt wurde, um  aktuelle Ereignisse (z. B. laufende Vorlesungen) aus dem iCalendar-Format (iCal) zu extrahieren und anzuzeigen. Diese Komponente ist in React geschrieben und verwendet \emph{useState}- und \emph{useEffect}-Hooks, um den internen Status zu verwalten. Dieser Code entspricht der \emph{SchedulerNow}-Komponente:

\begin{lstlisting}[language=JavaScript,
	frame=single,           % Ein Rahmen um den Code
	framexleftmargin=15pt,  % Rahmen link von den Zahlen
	style=algoBericht,
	label={SchedulerNow-Komponente},
	captionpos=b ,          % Caption unter den Code setzen
	caption={SchedulerNow-Komponente}]
import React, { useState, useEffect } from 'react';
import axios from 'axios';
import ICAL from 'ical.js';
import { isWithinInterval } from 'date-fns';
import './SchedulerNow.css';

const SchedulerNow = () => {
   const [currentEvent, setCurrentEvent] = useState(null);
   useEffect(() => {
   	const fetchData = async () => {
   	  try {
   	    const response = await axios.get(process.env.
   	    REACT_APP_PROXY_ADDRESS+'/schedule');
   	    const data = response.data;
   	    const parsedEvents = parseEvents(data);
   	    const currentEvent = getCurrentEvent(parsedEvents);
   	    setCurrentEvent(currentEvent);
       } catch (error) {
        console.error('Error fetching schedule data:', error);}
    };
    fetchData();}, []);
    
    const getCurrentEvent = (events) => {
       const now = new Date();
       const currentEvent = events.find(
       (event) => isWithinInterval(now, 
       { start: event.start, end: event.end }));
       return currentEvent;};
   
   const parseEvents = (icalData) => {
     const jcalData = ICAL.parse(icalData);
     const comp = new ICAL.Component(jcalData);
     const events = comp.getAllSubcomponents('vevent');
     const now = new Date();
     const startDate = ICAL.Time.fromJSDate(new Date(
     now.getFullYear(), now.getMonth()-1, now.getDate()));
     const endDate = ICAL.Time.fromJSDate(new Date(
     now.getFullYear() + 1, now.getMonth(), now.getDate()));
     const parsedEvents = [];
     events.forEach(eventComponent => {
       const event = new ICAL.Event(eventComponent);
       const start = event.startDate;
       const end = event.endDate;
       if (event.isRecurring()) {
       	const iterator = event.iterator();
       	let next;
       	while ((next = iterator.next())) {
       	  if (next.compare(startDate) < 0) {
       	  	continue;}
          if (next.compare(endDate) > 0) {
          	break;}
         const duration = end.subtractDate(start);
         const eventStart = next;
         const eventEnd = eventStart.clone();
         eventEnd.addDuration(duration);
          parsedEvents.push({
            title: event.summary,
            start: eventStart.toJSDate(),
            end: eventEnd.toJSDate(),
            location: event.location ? event.location.value : '',
            description: event.description,});
        }
    } else {
     parsedEvents.push({
       title: event.summary ,
       start: start.toJSDate(),
       end: end.toJSDate(),
       location: event.location,
       description: event.description,});
   }});
  console.log('Parsed events:', parsedEvents);
  return parsedEvents;};

  const renderCurrentEvent = () => {
    console.log('Current event:', currentEvent);
    if (!currentEvent) {
    	return <div className="no-lecture">Keine Vorlesung</div>;}
    return (
    <div className="current-lecture">
    <div className="title">{currentEvent.title}</div>
    {currentEvent.location && (
      <div className="location">{currentEvent.location}</div>
	)}
    </div>);
   };
 return (
 <div className="scheduler-now">
 {renderCurrentEvent()}</div>);};

export default SchedulerNow;
\end{lstlisting}

Zunächst wird der Zustand \emph{currentEvent} initialisiert, der das aktuelle Ereignis darstellt. Dieser Status wird später aktualisiert, wenn Daten von der API abgerufen und verarbeitet werden. 
Der \emph{useEffect}-Hook wird verwendet, um die \emph{fetchData}-Funktion aufzurufen, nachdem die Komponente gerendert wurde. \emph{fetchData} verwendet die \emph{Axios}-Bibliothek, um eine HTTP-GET-Anfrage an den Proxy-Server zu senden, um die iCal-Daten abzurufen. Diese asynchrone Funktion verwendet \emph{Try/Catch}, um  Fehler zu behandeln, die beim Abfragen der Daten auftreten können. Mit der Funktion \emph{parseEvents} werden \emph{iCal}-Daten in ein für JavaScript geeignetes Format konvertiert. Die \emph{ICAL.js}-Bibliothek wird verwendet,um\emph{iCal}-Daten zu analysieren und  Ereignisse zu extrahieren. Es berücksichtigt auch wiederkehrende Ereignisse und erstellt  für jede Wiederholung innerhalb des angegebenen Zeitraums ein separates Ereignis. 
Die Funktion \emph{getCurrentEvent} findet das aktuelle Ereignis, indem diese das aktuelle Datum und die aktuelle Uhrzeit mit den Start- und Endzeiten jedes Ereignisses vergleicht. Das ermittelte Ereignis wird  im Zustand gespeichert.Die Funktion \emph{renderCurrentEvent} ist für das Rendern des aktuellen Ereignisses verantwortlich. Diese prüft, ob es aktuelle Veranstaltungen gibt und zeigt eine Meldung an, dass keine Vorträge stattfinden bzw. die aktuellen Veranstaltungsdetails entsprechend. 

Diese Implementierung ermöglicht eine effiziente und reaktive Darstellung aktueller Ereignisse. Durch die Verwendung des React-Frameworks zum asynchronen Abrufen von Daten können Komponenten schnell auf Änderungen reagieren und aktuelle Ereignisse zeitnah aktualisieren. Ein Proxy-Server ermöglicht es Komponenten, CORS-Richtlinien zu umgehen und gleichzeitig die  Haupt-API auszulagern. Durch die Verwendung von Caching-Techniken und das Parsen von \emph{iCal}-Daten in Ihrer Clientanwendung wird die Leistung weiter optimiert und die Skalierbarkeit verbessert.
\newpage
\subsection{Links}
\subsubsection{Konzept}
Die \emph{Link}-Komponente von React ist eine spezielle Komponente, die dazu dient, eine Sammlung von Links zu verschiedenen Ressourcen anzuzeigen. Diese Komponente verwendet die \emph{useState}- und \emph{useEffect}-Hooks von React, um den internen Status zu verwalten und Effekte zu behandeln.
In den folgenden Abbildungen befinden sich die Webseite von der Links-Komponente und der Programmablaufplan
\begin{figure}[htbp]
	\centering
	\fbox{\includegraphics[height=0.7\textheight]{images/PAPLinks}}
	\caption{PAP Links}
\end{figure}
\begin{figure}[htbp]
	\centering
	\fbox{\includegraphics[height=0.3\textheight]{images/Links}}
	\caption{Links-Website}
\end{figure}
\newpage
\subsubsection{Implementierung}
Die Komponente \emph{Links} definiert zunächst eine Array-Struktur \emph{links}, die die verschiedenen Kategorien von Links und die jeweiligen Linkobjekte selbst enthält. Jedes Linkobjekt besteht aus einem Titel und einer URL.  
Der \emph{useState}-Hook wird verwendet, um zwei interne Zustandsvariablen zu definieren. \emph{selectedBox} verfolgt, welches Linkfeld gerade ausgewählt ist, und \emph{loadingLink} verfolgt, welcher Link gerade geladen wird.\\  
Der \emph{useEffect}-Hook wird verwendet, um den Status \emph{loadingLink} zurückzusetzen, wenn die Komponente nicht zusammengebaut ist. Hierbei handelt es sich um eine Sicherheitsmaßnahme, um sicherzustellen, dass der Status korrekt zurückgesetzt wird und es nicht zu unerwarteten Statusänderungen kommt, wenn Komponenten in  Zukunft wieder zusammengebaut werden. Die Funktion \emph{handleBoxClick} wird verwendet, um den Status der \emph{selectedBox} zu aktualisieren, wenn auf die Box geklickt wird. Durch erneutes Auswählen eines bereits ausgewählten Felds wird dessen Auswahl aufgehoben (d. h. \emph{selectedBox} wird auf -1 gesetzt). \\
Die Funktion \emph{handleLinkClick} dient zur Steuerung des Link-Klickverhaltens. Dadurch wird verhindert, dass das Standardereignisverhalten den Status \emph{loadingLink} aktualisiert und den Link in einem neuen Tab öffnet. Anschließend wird nach einer Verzögerung von 2000 ms der Status \emph{loadingLink} zurückgesetzt.  Die Rendermethode der Komponente durchläuft das Array \emph{links} und erstellt ein Element für jedes Feld und die darin enthaltenen Links. CSS-Klassen für  Boxen und Links werden basierend auf ihrem aktuellen Status dynamisch generiert.\\
Dies ist der zusammenhängende Code der Link-Komponente:
\begin{lstlisting}[language=JavaScript,
	frame=single,           % Ein Rahmen um den Code
	framexleftmargin=15pt,  % Rahmen link von den Zahlen
	style=algoBericht,
	label={Links-Komponente},
	captionpos=b ,          % Caption unter den Code setzen
	caption={Links-Komponente}]
import React, { useState,useEffect } from "react";
import "./Link.css";
const Link = () => {
  const links = [
  {
  	\\ Ein Haufen Links die fuers Studieren wichtig sind.
  	\\........................................
  }
];

  useEffect(() => {
    return () => {
      setLoadingLink(null); // Setzt den Ladezustand zurueck, 
      wenn die Komponente unmountet wird
  };}, []);
  const [selectedBox, setSelectedBox] = useState(-1);
  const [loadingLink, setLoadingLink] = useState(null); 
  // Initialer State zu null
  const handleBoxClick = (index) => {
    if (selectedBox === index) {
      setSelectedBox(-1);} 
    else {
    	setSelectedBox(index);
    }};
  const handleLinkClick = (e, boxIndex, linkIndex, link) => {
  	e.stopPropagation();
  	setLoadingLink({ boxIndex, linkIndex }); 
  	// Setzen Sie den Lade-Link auf das aktuelle Link-Objekt
  	window.open(link.url, "_blank");
	
    setTimeout(() => {
      setLoadingLink(null); // Setzt den Lade-Link 
      nach einer Verzoegerung zurueck
  }, 2000);};

 return (
 <div className="link-collection">
 {links.map((box, boxIndex) => (
 	<div key={boxIndex}
 	className={`link-box ${selectedBox === boxIndex ? 
 			"expanded" : ""}`}
 	onClick={() => handleBoxClick(boxIndex)}>
      <h3>{box.title}</h3>+
      <ul>
        {box.items.map((link, linkIndex) => (
         <li key={linkIndex}>
         <a className={loadingLink && 
         	loadingLink.boxIndex === boxIndex && 
         	loadingLink.linkIndex === linkIndex 
         	? "loading" : ""}
          onClick={(e) => handleLinkClick(
          	e, boxIndex, linkIndex, link)}>
          {link.title}
          </a>
          </li>))}
       </ul>
       </div>
		))}
	</div>
	);
};
export default Link;

\end{lstlisting}

Die Implementierung dieser Komponente ermöglicht eine bequeme und effiziente Darstellung von Linkgruppen. Die Struktur und das Verhalten der Komponenten sind klar und intuitiv, und der Einsatz von Statusverwaltung und Event-Handlern stellt sicher, dass die Komponenten korrekt auf Benutzerinteraktionen reagieren und  aktualisiert werden. Durch die Verwendung von React für diese Komponente ist sie wiederverwendbar und lässt sich leicht in andere Teile einer größeren Anwendung integrieren.
\newpage
\section{Zusammenfassung}
DHBW-Star ist ein  neues Webportal-Angebot für Studierende der DHBW Karlsruhe. Es wurde  entwickelt, um einen zentralen, benutzerfreundlichen Zugangspunkt zu allen relevanten Ressourcen und Informationen bereitzustellen. Das Portal vereint die drei Hauptkomponenten Stundenplan, Essensplan und Linksammlung und ist auf die spezifischen Bedürfnisse der Studierenden zugeschnitten. 
Um sicherzustellen, dass DHBW-Star den tatsächlichen Bedürfnissen der Studierenden entspricht und einen echten Mehrwert bietet, haben wir eine Umfrage durchgeführt, um Feedback und Verbesserungsvorschläge zu sammeln.
Diese Umfrage wird in den nächsten zwei Kapiteln erklärt und ausgewertet.
\newpage
\section{Docker}

Damit die React App vom System unabhängig funktioniert, läuft sie in einem Docker Container. Dadurch werden Probleme mit unterschiedlichen Programmversionen und Konfigurationen verhindert.
Da die geplante Mensa API nicht funktioniert, wird als Alternative ein Container verwendet, der die Mensa API lokal bereitstellt.

\subsection{Die Dockerfile, compose.yaml und .env Dateien}

Für den React Containers wird ein Dockerfile Datei benötigt. In dieser wird beschrieben wie der Container zusammengebaut wird. Ein neuer Container kann auf vorhanden aufbauen oder komplett neu erstellt werden.

\begin{lstlisting}[language=vhdl,
	frame=single,           % Ein Rahmen um den Code
	framexleftmargin=15pt,  % Rahmen link von den Zahlen
	style=algoBericht,
	label={Dockerfile},
	captionpos=b           % Caption unter den Code setzen
	caption={Dockerfile für DHBW-Star}]
FROM node:current-alpine

RUN mkdir /reactApp
WORKDIR /reactApp

# Install React and dependencies
COPY . .
RUN npm init -y
RUN npm install express cors axios
RUN npm install
RUN npm install ical.js
RUN npm install -D concurrently
RUN yarn build

# Start both React app and proxy server
CMD ["npm", "run", "start:all"]
\end{lstlisting}

Für DHBD Star wird der node Container von Dockerhub.com als Basis verwendet. Dieser besteht aus \emph{alpine}, einem minimalistischen Betriebssystem und NoteJS welcher für React benötigt wird.
Mit \emph{COPY} werden die Dateien des Projekts in den Container kopiert und mit \emph{RUN npm init} und \emph{RUN npm install} werden die benötigten Pakete für NoteJS installiert.
Zum Schluss wird mit \emph{RUN yarn build} das React Projekt gebaut und mit \emph{CMD ["npm", "run", "start:all"]} React und der Proxy gestartet. Der \emph{CMD} befehlt wird beim erstellen eines Container aus dem Image, immer ausgeführt. 

Damit die Mensa API und React zusammen starten, wird die \emph{compose.yaml} Datei und Docker compose verwendet.

\begin{lstlisting}[language=vhdl,
	frame=single,           % Ein Rahmen um den Code
	framexleftmargin=15pt,  % Rahmen link von den Zahlen
	style=algoBericht,
	label={Dockerfile},
	captionpos=b           % Caption unter den Code setzen
	caption={compose.yaml für DHBW-Star}]
version: '3.9'
services:
  react-app:
    build:
      dockerfile: ./Dockerfile
    tags:
      - "react:latest"
    ports:
      - "3003:3003"
      - "3002:3002"
    volumes:
      - ./src:/reactApp/src:ro
      - ./proxy-server.js:/reactApp/proxy-server.js:ro
      - ./.env:/reactApp/.env:ro
  mensa-api:
    image: meyfa/ka-mensa-api
    ports:
      - "3001:8080"
    environment:
      - MENSA_CORS_ALLOWORIGIN=*
\end{lstlisting}

Für den Service \emph{react-app} (DHBW Star) wird mit \emph{build} die oben beschrieben Dockerfiel Datei angeben. Somit baut der Befehl \emph{docker compose --build} ein neue Image mit dem React Projekt.
Der \emph{tags} gibt dem Container einen eindeutigen Namen.
Mit \emph{ports} werden die Ports vom Proxy(3002) und React (3003) nach außen frei gegeben.
Damit bei kleineren Änderungen im Projekt nicht immer ein neues Image erstellt werden muss, wird mit \emph{volums} die Projektdateien in den laufenden Container eingebunden.
Für die \emph{mensa-api} wird noch die Umgebungsvariable \emph{MENSA\_CORS\_ALLOWORIGIN=*} definiert.

Um die Container an unterschiedliche Gegebenheiten des Systems anpassen zu können, ohne etwas am React Projekt ändern zu müssen, bietet docer compose die Möglichkeit Umgebungsvariablen aus einer \emph{.env} Datei heraus zu definieren.
Da diese nicht teil von git sein sollte, wird eine \emph{.env\_template} angelegt. In dies stehen sind die verwendeten Umgebungsvariable mit Beispielen drin.
React verlangt für alle Umgebungsvariablen, die in der App zur Verfügung stehen sollen, dass sie mit \emph{REACT\_APP\_} beginnen.

\begin{lstlisting}[language=vhdl,
	frame=single,           % Ein Rahmen um den Code
	framexleftmargin=15pt,  % Rahmen link von den Zahlen
	style=algoBericht,
	label={Dockerfile},
	captionpos=b           % Caption unter den Code setzen
	caption={.env für DHBW-Star}]
REACT_APP_MENSA_ADDRESS=http://localhost:3001 
REACT_APP_PROXY_ADDRESS=http://localhost:3002
PROXY_PORT=3002
PORT=3003
\end{lstlisting}

\emph{REACT\_APP\_MENSA\_ADDRESS} und \emph{REACT\_APP\_PROXY\_ADDRESS} sind die Adressen der Services. Dies müssen mit der in \emph{compose.yaml} definierten Ports übereinstimmen. 

\section{Webserver}
Damit das Projekt auch über das Internet erreichbar ist, wird ein Root-Webserver mit einer Domain Adresse verwendet. Auf diesem ist bereits Docker, Nginx und Certbot Installiert.
Per github.com wird das Projekt auf den Server kopiert und die \emph{.env} Datei angepasst. Die Ports bleiben die Selben, nur das \emph{localhost} wird durch die Subdomains \emph{mensa.Domain} für den Mensa Container und \emph{ical.Domain} für den Proxy ersetzt.

Damit die Container unter ihrer Subdomain erreichbar sind, müssen diese in der Nginx Konfiguriert werden. In der Datei \emph{.../sites-enabled/star.conf} wird für jeden Service mit \emph{proxy\_pass} eine Weiterleitung eingerichtet.
Da auf dem Server bereits andere Webdienst laufen, ist Startseite unter der Subdomain \emph{star.Domain} erreichbar. Mit \emph{Certbot} werden noch SSL Zertifikate für alle Subdomains erstellt und automatisch eingerichtet.

Nach dem Neustart von Nginx, steht DHBW Star für alle Tester zur Verfügung.


\chapter{Umfrage/Datenerhebung}
Test test tes tes 


\chapter{Datenanalyse}
Für die Auswertung eines Fragebogens gibt es verschiedene Methoden die zum Einsatz kommen können. Der hier verwendetet Fragebogen verwendet drei arten von Antwortmöglichkeiten
\begin{itemize}
	\item[Typ 1:] {Ordinalskala: Antworten mit einer Rangfolge\\
		(z.B. sehr nützlich, etwas nützlich oder unnütz)\\
		Fragen 1, 3, 4, 6, 8 und 9}
	\item[Typ 2:] {Nominalskala: Antworten ohne Rangfolge\\
		(z.B. PC, Tablett und Mobile)\\
		Fragen 2 und 10}
	\item[Typ 3:] {Freie Antworten \\
		Fragen 5, 7 und 10 (Sonstiges)}
\end{itemize}
Für die Verschiedenen Typen gibt es jeweils eine Verfahren zur Bewertung.
Typ 1 und 2 werden mit der Häufigkeitsmethode ausgewertet, wobei bei Typ 1 auch noch nach Rangfolge ausgewertet werden kann. Bei Typ 3 kommt eine Qualitative Methode zum Einsatz.

Da für die Auswertung maximal nur 10 Datenpunkte zu Verfügung stehen, können keinen statistischen verfahren verwendet werden.

\section{Einschränkungen der Aussagekraft}
Da DHBW-Star speziell für den Tinf20B3 Kurz entwickelt worden ist, sind die befragten Personen alle im selben Kurs. Dadurch ist die Stichprobe sehr klein und somit lässt sich das Ergebnis auch nur sehr eingeschränkt auf die anderen Studenten Übertagen.

Zudem ist ein davon auszugehen, dass die Befragten eher wohlwollend bewerten, da sie die Entwickler Persönlich kennen.

Es ist dennoch möglich abzuschätzen ob das Design in die richtige Richtung geht und ob DHBW-Star eine Verbesserung bringen würde.

Die nachfolgende Auswertung ist dementsprechend auch eher vorsichtig mit ihren Aussagen und daraus resultierenden Schlussfolgerungen.

\section{Aufbereitung von surveymonkey.de}
Die Webseite www.surveymonkey.de bereitet die Antworten einer Umfrage angemessen auf. Die Ergebnisse vom Typen 1 und 2 sind jeweils Tabellarisch als auch mit einem Balkendiagramm dargestellt.
Die von Antworten zu Typ 3 sind in einer Tabelle zur weiteren Auswertung Festgehalten.

Da ein Export nur für zahlenden Kunden angeboten wird, müssen die Daten manuelle kopiert und per Screenshot gesichert werden.

Ein erste Überblick zeigt bereits, dass die Bewertungen überwiegend positiv ausgefallen sind.

\section{Die Daten}
Es folgt ein Überblick über die erhobenen Daten.

Sechs der Befragten haben angeben, dass sie die DHBW-Star nur einmal benutzt zu haben.
Jeweils eine Person hat sie Täglich und eine kein mal verwendet.
Die restlichen zwei, haben mehrmals angegeben.\\
\emph{Anmerkung: Das Kein mal hat sich im nach hinein als ein versehen des Befragten herausgestellt.}\\
\\
Die genutzten Geräte sind nur PC von 8 Personen und Mobile von 4 Personen.\\
\\
Als unnütz befand nur einer die Startseite. Der Reste empfand sie als mindestens nützlich.\\
\\
Der \emph{Scheduler} wurde von zwei Personen als gleich gut beurteilt. Der restlichen Mehrheit gefiel es (viel) besser.\\
\\
Bei \emph{Food} ist es ähnlich wie bei \emph{Scheduler}. Auch zwei Personen hat es gleich gut gefallen. Allerdings nur einer Person hat viel besser angegeben. Die restlichen 7 einigten sich auf besser.
\\
Die \emph{Link-Sammlung} ist einheitlich als mindestens nützlich bewertet.\\
\\
Genau so ist es bei der gesamt Bewertung von DHBW-Star. Unnütz gab keiner an und der Reste teilte sich gleichmäßig zwischen etwas und sehr nützlich auf.\\
\\
Auf die Letzte Frage, was DHBW-Star noch fehlt, hat jeder Vorschlag mindestens 2 Stimmen bekommen. Die meisten davon entfielen auf 'Integration von Dualis' und 'Individueller Scheduler' mit jeweils 7 Stimmen.

\section{Bewertung}
In diesem Abschnitt werden die einzelnen Ergebnisse bewertet und versucht erklängen dafür zu geben.

Da sich nicht alles mit Daten aus der Umfrage erklären lässt, kommen auch persönliche Erfahrungen zum tragen. Diese sind durch ein vorangestelltes \textbf{Pers:} zu erkennen. 

\begin{itemize}
	\item[Frage 1:]
		{\textbf{Pers:} Es ist zu vermuten, dass sich die meiste Befragten die Seite nur einmal Angeschaut haben weil sie sich bereits so sehr an die bisherigen Informationsangebote gewohnt haben. Denkbar ist auch das einige von ihnen eine eigene Lösung umgesetzt haben.}
	\item[Frage 2:]
		{\textbf{Pers:} War zu erwarten. Es gib nur wenige Personen mit Tablet im Kurs.}
	\item[Frage 3:]
		{\textbf{Pers:} Auch wenn die Startseite von der großen Mehrheit als mindestens etwas nützlich beurteilt wurde, ist offensichtlich noch Raum für Verbesserungen.}
	\item[F. 4 \& 5:]
		{Die meisten sind finden den \emph{Scheduler} wegen seinem schlichten Design und der Day-Ansicht zwar besser, vermissen aber auch noch Funktionalitäten wie Filter bzw. das ausblenden von Vorlesungen.\\
		\textbf{Pers:} Da Rapla schon lange für ärger im Kurs geführt hat, war das Ergebnis zu erwarten.}
	\item[F. 6 \& 7:]
		{Auch hier wird das Simpel Design als gut bewertet. Negativ wird ein ein Fehler im Software beurteilt, weshalb nicht automatisch die Karte des heutigen Tages angezeigt wird. \todo{ref zu Fehler}}
	\item [Frage 8:]
		{Die \emph{Link-Sammlung} ist mit Abstand die best bewertete Funktion von DHB-Star.\\
		\textbf{Pers:} Aus eigener Erfahrung sind nachfragen zu einer der Links auch einer der häufigsten Fragen im Gruppenchat, die immer wieder gestellt werden.}
	\item [Frage 9:]
		{Die Positiven Bewertungen aus den vorherigen fragen, Spiegeln sich auch in der Frage zum Gesamteindruck wieder. Auch die Kritikpunkte sind in der 50/50 Aufteilung zu wieder zu erkennen.}
	\item [Frage 10:]
		{\textbf{Pers:} Diese Frage dient Hauptsächlich dazu weiter Verbesserungsvorschläge für das Informationsangebot der DHBW machen zu können.}
\end{itemize}



\chapter{Fazit}
Test test tes tes 

\todo {Daniel und Burak noch fertig machen }


% Ab hier beginnt der Anhang
\appendix
%\addcontentsline{toc}{chapter}{Anhang}
\chapter{Umfrage Daten}
%\begin{center}
%	\begin{table}[h]
%		\begin{tabularx}{4cm}{|l|l|p{4.5 cm}|ll}
%			\cline{1-3}
%			\textbf{Technolgie} & \textbf{Vorteil}                         & \textbf{Nachteil}                                                &  &  \\ \cline{1-3}
%			Angular             & Schlanker Kerncode mit hoher Modularität & nicht geeignet für dynamiche Web-App  &  &  \\ \cline{1-3}
%			React               & Einfach zu erlernen, aktive Community    & Weniger leistungsstark         &  &  \\ \cline{1-3}
%			Vue.js              & Einfache Satzlehre, schnelle Startzeit   & Kleinere Community             &  &  \\ \cline{1-3}
%		\end{tabularx}
%	\end{table}
%\end{center}+
\section{Frage 1}
\begin{wrapfigure}[2]{r}{0.5\textwidth}
	\centering
	\fbox{\includegraphics[width=0.4\textwidth]{images/frage-01}}
\end{wrapfigure}
\textbf{Frage:} Wie oft hast du DHBW Star in den letzten 7 tagen benutzt?\\
\textbf{Ergebnisse:}\\
\begin{tabular}{|l|c|}\hline
	\textbf{Antwortoptionen} & \textbf{Beantwortungen} \\\hline
	kein mal & 1 \\\hline
	1 mal	 & 6 \\\hline
	Mehr mal & 2 \\\hline
	Täglich  & 1 \\\hline
	GESAMT	 & 10 \\\hline			
\end{tabular}
%\begin{figure}[htbp]
%	\fbox{\includegraphics[width=0.45\textwidth]{images/frage-01}}
%\end{figure}



\section{Frage 2}
\begin{wrapfigure}[2]{r}{0.5\textwidth}
	\centering
	\fbox{\includegraphics[width=0.40\textwidth]{images/frage-02}}
\end{wrapfigure}
\textbf{Frage:} Auf welchem gerät hast du DHBW Star benutzt?\\
\textbf{Ergebnisse:}\\
\\
\begin{tabular}{|l|c|}\hline
	\textbf{Antwortoptionen} & \textbf{Beantwortungen} \\\hline
	PC 			& 8 \\\hline
	Tablet		& 0 \\\hline
	Mobile 		& 4 \\\hline
	GESAMT		& 10 \\\hline			
\end{tabular}

\section{Frage 3}
\begin{wrapfigure}[2]{r}{0.5\textwidth}
	\centering
	\fbox{\includegraphics[width=0.40\textwidth]{images/frage-03}}
\end{wrapfigure}

\textbf{Frage:} Wie nützlich findest du die Startseite?\\
\textbf{Ergebnisse:}\\
\\
\begin{tabular}{|l|c|}\hline
	\textbf{Antwortoptionen} & \textbf{Beantwortungen} \\\hline
	sehr nützlich  	& 5 \\\hline
	etwas nützlich	& 4 \\\hline
	unnütz 			& 1 \\\hline
	GESAMT			& 10 \\\hline			
\end{tabular}

\section{Frage 4}
\textbf{Frage:} Wie gefällt dir die Darstellung des 'Scheduler' im verglich zum Rapla?
\begin{wrapfigure}[2]{r}{0.5\textwidth}
	\centering
	\fbox{\includegraphics[width=0.40\textwidth]{images/frage-04}}
\end{wrapfigure}
\textbf{Ergebnisse:}\\
\\
\begin{tabular}{|l|c|}\hline
	\textbf{Antwortoptionen} & \textbf{Beantwortungen} \\\hline
	viel besser  	& 3 \\\hline
	besser			& 5 \\\hline
	gleich 			& 2 \\\hline
	schlechter 		& 0 \\\hline
	viel Schlechter	& 0 \\\hline
	GESAMT			& 10 \\\hline			
\end{tabular}

\section{Frage 5}
\textbf{Frage:} Was gefällt oder fehlt dir beim Scheduler?\\
\begin{tabular}{|l|}\hline
	\textbf{Antworten} \\\hline
	Simples Design nicht zu überladen / Tages Funktion cool  \\\hline
	ein paar Bugs verbessern, schon visualisiert week und day ansicht \\\hline
	Direkter Link zu EvaSys der Vorlesung :D \\\hline
	Die Räume fehlen zumindest auf Mobile und eine Auswahl der Wahlmodule wäre\\
	praktisch, da man wenn alle angezeigt werden die Freitagsmodule kaum lesen kann 
	\\\hline
	Fehlt: Dozenten, Raum, Individualisierung (nicht jeder hat jede Vorlesung) \\\hline
	Filter für kurse \\\hline			
\end{tabular}

\section{Frage 6}
\textbf{Frage:} Wie gefällt dir die Darstellung des Food im verglich zum sw-ka.de?
\begin{wrapfigure}[1]{r}{0.5\textwidth}
	\centering
	\fbox{\includegraphics[width=0.40\textwidth]{images/frage-06}}
\end{wrapfigure}
\textbf{Ergebnisse:}\\
\\
\begin{tabular}{|l|c|}\hline
	\textbf{Antwortoptionen} & \textbf{Beantwortungen} \\\hline
	viel besser  	& 1 \\\hline
	besser			& 7 \\\hline
	gleich 			& 2 \\\hline
	schlechter 		& 0 \\\hline
	viel Schlechter	& 0 \\\hline
	GESAMT			& 10 \\\hline			
\end{tabular}

\section{Frage 7}
\textbf{Frage:} Was gefällt oder fehlt dir beim Food?\\
\begin{tabular}{|l|}\hline
	\textbf{Antworten} \\\hline
	Simpel nicht zu überladen  \\\hline
	emojis, benutzerfreundlich , schön gesammelt in einem ort \\\hline
	Ich sehe direkt was es in der Mensa hier an der DH gibt \\\hline
	die simple Darstellung ist toll, was man noch hinzufügen könnte wäre ein\\
	heute knopf oder so, da aktuell ja nur die Daten drin stehen 
	\\\hline
	Fehlt: "Heute"-Button, Tage (Montag/Dienstag/...)\\
	Gefällt: Man muss sich nicht erst durch einen Haufen Links klicken bis man zum Essen kommt. \\\hline
	Wochentagbezeichnung, nachhaltigkeitsbewertungen, allergien und so \\\hline			
\end{tabular}

\section{Frage 8}
\textbf{Frage:} Wie nützlich findest du die Link-Sammlung?
\begin{wrapfigure}[2]{r}{0.5\textwidth}
	\centering
	\fbox{\includegraphics[width=0.40\textwidth]{images/frage-08}}
\end{wrapfigure}
\textbf{Ergebnisse:}\\
\\
\begin{tabular}{|l|c|}\hline
	\textbf{Antwortoptionen} & \textbf{Beantwortungen} \\\hline
	sehr nützlich  	& 7 \\\hline
	etwas nützlich	& 3 \\\hline
	unnütz 			& 0 \\\hline
	GESAMT			& 10 \\\hline			
\end{tabular}

\section{Frage 9}
\begin{wrapfigure}[2]{r}{0.5\textwidth}
	\centering
	\fbox{\includegraphics[width=0.40\textwidth]{images/frage-09}}
\end{wrapfigure}

\textbf{Frage:} Wie nützlich findest du DHBW Star?\\
\textbf{Ergebnisse:}\\
\\
\begin{tabular}{|l|c|}\hline
	\textbf{Antwortoptionen} & \textbf{Beantwortungen} \\\hline
	sehr nützlich  	& 5 \\\hline
	etwas nützlich	& 5 \\\hline
	unnütz 			& 0 \\\hline
	GESAMT			& 10 \\\hline			
\end{tabular}

\section{Frage 10}

\textbf{Frage:} Was Fehlt dir noch bei DHBW Star?\\
\textbf{Ergebnisse:}\\
\\
\begin{tabular}{|l|c|}\hline
	\textbf{Antwortoptionen} & \textbf{Beantwortungen} \\\hline
	Integration von Dualis  	& 7 \\\hline
	Integration von Moodle 	& 4 \\\hline
	Integration ÖPNV Fahrplan  			& 5 \\\hline
	Umfragen für Kurs Erstellen & 5 \\\hline
	Benachrichtigungen für Kurs  & 2 \\\hline
	Individueller Scheduler (Vorlesungen ein- und ausblenden,\\ eigene Termin hinzufügen) & 7 \\\hline
	Todo Liste (Hausaufgaben / Abgaben) & 2 \\\hline
	Sonstiges (bitte angeben)\\
	Wäre fürs 1.Semester sehr sinnvoll gewesen\\
	Wäre am anfang des studiums nützlich gewesen & 2 \\\hline 
	Befragte gesamt:			& 10 \\\hline			
\end{tabular}

\begin{figure}[htbp]
	\fbox{\includegraphics[width=\textwidth]{images/frage-10}}
\end{figure}
\end{onehalfspace}
\addcontentsline{toc}{chapter}{Index}
\printindex

\addcontentsline{toc}{chapter}{Literaturverzeichnis}

% Haben Sie das "biblatex"-Paket nicht installiert, benutzen Sie folgendes:
% Ohne das "biblatex"-Paket (s. bericht.sty) produziert folgendes
% "deutsche" Zitate in Literaturverzeichnissen gemaß der Norm DIN 1505,
% Teil 2 vom Jan. 1984.
% Die Zitatmarken werden alphabetisch nach Verfassern
% sortiert und sind durch abgekürzte Verfasserbuchstaben plus
% Erscheinungsjahr in eckigen Klammern gekennzeichnet.

% \bibliographystyle{alphadin}
% \bibliography{bericht}

%%%%%%%%%%%%%%%%%%%%%%%%%%%%%%%%%%%%%%%5
% BIBLATEX
% Benutzt man das "biblatex"-Paket, muß man folgendes schreiben:
\def\refname{Literaturverzeichnis}
\printbibliography
%%%%%%%%%%%%%%%%%%%%%%%%%%%%%%%%%%%%%%%5


%\include{changelog}

\newpage
%\addcontentsline{toc}{chapter}{Liste der ToDo's}
%\listoftodos[Liste der ToDo's]


\end{document}
