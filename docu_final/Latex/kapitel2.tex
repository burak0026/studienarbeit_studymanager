%%%%%%%%%%%%%%%%%%%%%%%%%%%%%%%%%%%%%%%%%%%%%%%%%%%%%%%%%%%%%%%%%%%%%%%%%%%%%
%% Descr:       Vorlage für Berichte der DHBW-Karlsruhe, Ein Kapitel
%% Author:      Prof. Dr. Jürgen Vollmer, vollmer@dhbw-karlsruhe.de
%% $Id: kapitel2.tex,v 1.5 2017/10/06 14:02:51 vollmer Exp $
%%  -*- coding: utf-8 -*-
%%%%%%%%%%%%%%%%%%%%%%%%%%%%%%%%%%%%%%%%%%%%%%%%%%%%%%%%%%%%%%%%%%%%%%%%%%%%%%%

\chapter{Grundlagen}
In diesem Kapitel werden die Grundlagen der verwendeten Werkzeuge sowie die theoretischen Grundladen erläutert.
\section{React}
In diesem Unterkapitel werden die Grundlagen von React näher beschrieben um ein grundlegendes Verständnis zu generieren.

\subsection{Hintergrund und Motivation}
React ist eine JavaScript-Bibliothek zur Erstellung von Benutzeroberflächen, die von Facebook entwickelt wurde und seit 2013 öffentlich verfügbar ist. Die Motivation für die Entwicklung von React lag in der Notwendigkeit, die Leistung und Skalierbarkeit von Facebooks eigenen Webanwendungen zu verbessern. Insbesondere wollten die Entwickler von Facebook eine Möglichkeit finden, große und komplexe Anwendungen zu erstellen, die schnell, reaktiv und einfach zu pflegen sind.

In einem Blogbeitrag aus dem Jahr 2011, der von Jordan Walke, einem Entwickler bei Facebook, verfasst wurde, beschrieb er die Motivation für die Entwicklung von React wie folgt:

"Wir haben erkannt, dass viele unserer bestehenden Werkzeuge für die Web-Entwicklung nicht dafür geeignet waren, große und komplexe Anwendungen zu entwickeln und zu warten. Wir mussten eine neue Art von Werkzeug entwickeln, die uns erlauben würde, Anwendungen zu erstellen, die schnell, reaktiv und einfach zu pflegen sind." \cite{react-why}

React bietet viele Vorteile für Entwickler, darunter die Möglichkeit, wiederverwendbare Komponenten zu erstellen, die einfach zu warten und zu aktualisieren sind. Darüber hinaus kann React für eine Vielzahl von Anwendungen eingesetzt werden, einschließlich Single-Page-Anwendungen, Mobile-Apps und serverseitigen Anwendungen.

\begin{center}
	\includegraphics[width=0.3\textwidth]{react-logo.png}
\end{center}

\subsection{Überblick über React}

React ist eine beliebte JavaScript-Bibliothek, die zur Erstellung von Benutzeroberflächen verwendet wird. Wie in der offiziellen React-Dokumentation beschrieben wird, bietet React eine deklarative Syntax, um UI-Komponenten zu definieren und zu rendern. Diese Komponenten sind unabhängig voneinander und können wiederverwendet werden, um komplexe Benutzeroberflächen zu erstellen \cite{react-docs}.

Wie auf \cite{makeuseof} beschrieben wird, gibt es mehrere Gründe, warum es sinnvoll ist, React zu erlernen. Einer der wichtigsten Gründe ist, dass React auf JavaScript basiert, was es für Entwickler, die bereits JavaScript beherrschen, einfacher macht, sich in React einzuarbeiten. Ein weiterer Grund ist, dass React auf einem komponentenbasierten Ansatz basiert, der es erleichtert, Webanwendungen in kleine, wiederverwendbare Komponenten aufzuteilen, die unabhängig voneinander erstellt und gewartet werden können \cite{makeuseof}.

Darüber hinaus bietet React eine effektive Verwaltung des Zustands und eine leistungsstarke Wiederverwendbarkeit von Komponenten, was zu einer schnelleren Entwicklung von Webanwendungen führt. Ein weiterer Vorteil von React ist, dass es eine große und engagierte Entwicklergemeinschaft gibt, die sich ständig weiterentwickelt und verbessert. Dies sorgt für eine hohe Verfügbarkeit von Lernressourcen und eine ständige Verbesserung der Bibliothek \cite{makeuseof}.

Laut der Stack Overflow Umfrage 2021 ist React eine der beliebtesten Technologien im Bereich der Web-Frameworks. Dies zeigt, dass React von der Entwicklergemeinschaft aktiv genutzt und unterstützt wird \cite{stackoverflow-survey}.

Eine weitere Quelle auf \cite{makeuseof} betont auch die Bedeutung von React für die Entwicklung von Single-Page-Applications (SPA). React hat die Entwicklung von SPAs vereinfacht, indem es eine schnelle und nahtlose Benutzererfahrung ermöglicht, ohne dass die Seite neu geladen werden muss.

Zusammenfassend lässt sich sagen, dass React eine beliebte JavaScript-Bibliothek zur Erstellung von Benutzeroberflächen ist, die auf einem komponentenbasierten Ansatz basiert und eine effektive Verwaltung des Zustands sowie eine leistungsstarke Wiederverwendbarkeit von Komponenten bietet. React hat die Entwicklung von Single-Page-Applications erleichtert und ist aufgrund seiner zahlreichen Vorteile eine wertvolle Fähigkeit für Webentwickler.

\subsection{Grundlagen über React}
\subsubsection{Virtual DOM}
Das Virtual Document Object Model (DOM) ist eine innovative Technologie, die von React verwendet wird, um die Leistungsfähigkeit und Reaktionsfähigkeit von Webanwendungen zu verbessern. Im Gegensatz zum herkömmlichen DOM, bei dem der Browser bei jeder Änderung den gesamten Baum neu berechnen und aktualisieren muss, arbeitet das Virtual DOM effizienter und reduziert die Verzögerungen bei der Aktualisierung der Benutzeroberfläche \cite{goswami2020}.
React verwendet das Virtual DOM, um Änderungen an einer abstrakten Repräsentation des tatsächlichen DOM vorzunehmen und anschließend einen effizienten Vergleichsalgorithmus, genannt Reconciliation, durchzuführen \cite{reactinternals}. Dieser Algorithmus vergleicht das Virtual DOM mit dem tatsächlichen DOM und aktualisiert nur die tatsächlich geänderten Teile des DOM, wodurch die Leistung verbessert und die Aktualisierung der Benutzeroberfläche beschleunigt wird \cite{banks2016}.

\subsubsection{Komponenten und Props}
Eine React-Komponente ist eine unabhängige, wiederverwendbare Einheit einer Benutzeroberfläche. Sie kann als Klasse oder als Funktion definiert werden. Komponenten können andere Komponenten enthalten und können selbst als Teil einer größeren Anwendung verwendet werden.

Komponenten werden durch Props (kurz für "Properties") konfiguriert. Props sind ein Weg, um Daten von einer Komponente zur anderen weiterzugeben. Sie werden als Objekt an die Komponente übergeben und können dort als Parameter verwendet werden. Props sind im Allgemeinen schreibgeschützt und können nicht direkt geändert werden \cite{easeout-blog}.

Ein Beispiel für eine React-Komponente mit Props:

\begin{verbatim}
	import React from 'react';
	
	function Greeting(props) {
		return <h1>Hello, {props.name}!</h1>;
	}
	
	export default Greeting;
\end{verbatim}

In diesem Beispiel definiert die Funktion \verb|Greeting| eine Komponente, die einen Gruß mit dem Namen des Benutzers anzeigt. Der Name wird als Prop an die Komponente übergeben und als Parameter verwendet.

Props sind ein wichtiger Mechanismus zur Wiederverwendbarkeit von Komponenten. Eine Komponente kann in verschiedenen Kontexten verwendet werden, indem sie mit unterschiedlichen Props konfiguriert wird \cite{react-book}.
\subsubsection{State}
Der State ist ein Objekt, das Daten enthält, die von einer Komponente verwaltet werden. Eine Komponente kann ihren State ändern und die Änderungen werden automatisch auf der Benutzeroberfläche aktualisiert. Der State kann als interner Speicher der Komponente betrachtet werden \cite{state-docs}.

Ein Beispiel für eine React-Komponente mit State:

\begin{verbatim}
	import React, { Component } from 'react';
	
	class Counter extends Component {
		constructor(props) {
			super(props);
			this.state = { count: 0 };
		}
		
		increment() {
			this.setState({ count: this.state.count + 1 });
		}
		
		render() {
			return (
			<div>
			<p>Count: {this.state.count}</p>
			<button onClick={() => this.increment()}>Increment</button>
			</div>
			);
		}
	}
	
	export default Counter;
\end{verbatim}

In diesem Beispiel definiert die Klasse \verb|Counter| eine Komponente, die einen Zähler mit einem Button zum Inkrementieren anzeigt. Der Zählerwert wird im State gespeichert und bei jeder Änderung aktualisiert.

Der State ist ein wichtiger Mechanismus zur Verwaltung von Daten in React-Komponenten. Durch das Aktualisieren des State können Komponenten auf Benutzerinteraktionen reagieren und ihre Darstellung aktualisieren \cite{w3schools}.
\subsection{Verwendung von React}
\subsubsection{Einrichtung von React}
React ist eine Bibliothek für JavaScript, die von Facebook entwickelt wurde und zur Erstellung von Benutzeroberflächen eingesetzt wird. Um React in einem Projekt verwenden zu können, muss es zuerst eingerichtet werden. Die offizielle Dokumentation von React bietet eine ausführliche Anleitung zur Einrichtung von React \cite{ReactDocsSetup}.

Eine Möglichkeit, React in einem neuen Projekt zu installieren, besteht darin, ein neues Projektverzeichnis zu erstellen und dann das React-Paket über NPM zu installieren:

\begin{verbatim}
	npx create-react-app my-app
	cd my-app
	npm start
\end{verbatim}

Dieser Befehl erstellt ein neues React-Projekt mit dem Namen "my-app" und startet den Entwicklungsserver.

Es ist auch möglich, React manuell in ein bestehendes Projekt einzubinden. In diesem Fall müssen die React- und React-DOM-Pakete installiert werden:

\begin{verbatim}
	npm install react react-dom
\end{verbatim}

Es gibt auch alternative Möglichkeiten, React in einem Projekt zu integrieren. Eine solche Möglichkeit ist die Verwendung von Create-React-App oder anderen Boilerplate-Vorlagen, die eine vorkonfigurierte React-Umgebung bereitstellen.
\subsubsection{Komponentenentwicklung}
In React werden UI-Elemente in Komponenten zerlegt. Eine Komponente kann entweder als Funktion oder als Klasse definiert werden. Eine Klasse-Komponente bietet zusätzliche Funktionen und Möglichkeiten, während Funktions-Komponenten eine einfachere Syntax haben.

Eine Beispielkomponente könnte wie folgt aussehen:

\begin{verbatim}
	import React from 'react';
	
	function Greeting(props) {
		return <h1>Hello, {props.name}!</h1>;
	}
	
	export default Greeting;
\end{verbatim}

Diese Komponente heißt "Greeting" und nimmt ein Objekt mit einem Namen als Prop entgegen. Wenn diese Komponente gerendert wird, wird sie "Hello, {props.name}!" als Überschrift ausgeben.

Es ist auch möglich, eine Komponente als Klasse zu definieren:

\begin{verbatim}
	import React, { Component } from 'react';
	
	class Greeting extends Component {
		render() {
			return <h1>Hello, {this.props.name}!</h1>;
		}
	}
	
	export default Greeting;
\end{verbatim}

Diese Klasse-Komponente heißt auch "Greeting" und nimmt ein Objekt mit einem Namen als Prop entgegen. Wenn diese Komponente gerendert wird, wird sie "Hello, {this.props.name}!" als Überschrift ausgeben.
\subsubsection{Ereignisbehandlung}
In React werden Ereignisse behandelt, indem man einem bestimmten UI-Element (z.B. einem Button oder einer Input-Box) eine Event-Handler-Funktion zuweist. Diese Funktion wird aufgerufen, wenn das Ereignis ausgelöst wird, z.B. wenn auf den Button geklickt wird.

Hier ist ein Beispiel einer Button-Komponente mit einem Event-Handler:

\begin{verbatim}
	import React from 'react';
	
	function Button(props) {
		function handleClick() {
			console.log('Button clicked');
		}
		
		return (
		<button onClick={handleClick}>
		{props.label}
		</button>
		);
	}
	
	export default Button;
\end{verbatim}

Diese Button-Komponente nimmt ein Label als Prop entgegen und weist dem Button-Element einen Event-Handler (handleClick) zu. Wenn der Button geklickt wird, wird die Funktion handleClick aufgerufen und 'Button clicked' wird in der Konsole ausgegeben.

Es ist auch möglich, den Event-Handler in einer Klasse-Komponente zu definieren:

\begin{verbatim}
	import React, { Component } from 'react';
	
	class Button extends Component {
		handleClick() {
			console.log('Button clicked');
		}
		
		render() {
			return (
			<button onClick={this.handleClick}>
			{this.props.label}
			</button>
			);
		}
	}
	
	export default Button;
\end{verbatim}

Diese Button-Komponente definiert den Event-Handler in der Klasse und verwendet die Syntax this.handleClick, um den Handler dem Button zuzuweisen.
\subsubsection{JSX}
JSX ist eine von React verwendete Syntax, die es ermöglicht, HTML-ähnlichen Code innerhalb von JavaScript zu schreiben. JSX wird während der Kompilierung in normales JavaScript umgewandelt, so dass es vom Browser verstanden werden kann.

Hier ist ein Beispiel einer JSX-Komponente:

\begin{verbatim}
	import React from 'react';
	
	function Greeting(props) {
		return <h1>Hello, {props.name}!</h1>;
	}
	
	export default Greeting;
\end{verbatim}

Diese Komponente verwendet JSX, um das HTML-ähnliche Element <h1>Hello, {props.name}!</h1> zurückzugeben. Die geschweiften Klammern {} werden verwendet, um JavaScript-Code innerhalb von JSX auszuführen.

Es ist auch möglich, JavaScript-Ausdrücke innerhalb von JSX zu verwenden, um dynamischen Inhalt zu erzeugen:

\begin{verbatim}
	import React from 'react';
	
	function Greeting(props) {
		const greeting = Hello, ${props.name}!;
		
		return <h1>{greeting}</h1>;
	}
	
	export default Greeting;
\end{verbatim}

In diesem Beispiel wird die JavaScript-Variablendeklaration const greeting verwendet, um den dynamischen Inhalt des Elements zu erzeugen.

\subsection{Fortgeschrittene Themen in React}
React bietet eine Vielzahl von fortgeschrittenen Techniken und Bibliotheken, um komplexe Anwendungen zu entwickeln. Einige dieser Techniken sind Redux oder Context API, React Router, React Hooks und serverseitiges Rendern.
\subsubsection{Redux oder Context API}
Redux und Context API sind zwei Bibliotheken für React, die es ermöglichen, Daten auf einfache Weise durch die Komponenten einer Anwendung zu verteilen. Redux ist ein leistungsfähiges Tool für die Verwaltung von Anwendungsdaten, während Context API eine leichtgewichtige Alternative ist \cite{reduxgettingstarted, reactcontext}.

Redux funktioniert auf der Grundlage eines zentralisierten Speichers, in dem alle Anwendungsdaten gespeichert werden. Jede Komponente, die Daten benötigt, kann sie einfach aus dem Store abrufen und bei Bedarf aktualisieren. Redux bietet auch Tools wie Actions, Reducers und Middleware, um die Verwaltung von Anwendungsdaten zu vereinfachen und zu automatisieren.

Context API hingegen ermöglicht es Entwicklern, Daten auf einfache Weise durch die Komponentenbaumhierarchie zu verteilen. Jede Komponente kann auf den Kontext zugreifen und Daten abrufen oder aktualisieren, ohne auf Props oder eine globale Store-Instanz angewiesen zu sein. Context API ist eine einfachere Alternative zu Redux, die sich besser für kleinere Anwendungen eignet.
\subsubsection{React Router}
React Router ist eine Bibliothek für React, die es ermöglicht, dynamische URLs in einer Single-Page-Anwendung zu verwenden. Mit React Router können Entwickler die Navigation innerhalb der Anwendung auf einfache Weise definieren und steuern, ohne dass die Seite neu geladen werden muss. Die Integration von React Router in eine Anwendung ist relativ einfach und erfordert nur wenige Schritte \cite{reactroutertutorial, reactrouteroverview}.

React Router bietet verschiedene Funktionen wie die Definition von Routen und Parametern, die Möglichkeit zur Integration von Serverseitigem Rendering und die Möglichkeit, mit anderen Bibliotheken wie Redux zu interagieren. React Router ist eine wichtige Bibliothek für React-Entwickler, da es die Navigation innerhalb der Anwendung vereinfacht und eine bessere Benutzererfahrung bietet.
\subsubsection{React Hooks}
React Hooks ist eine Funktion in React, die es Entwicklern ermöglicht, Zustände und Effekte in funktionellen Komponenten zu verwenden. Mit Hooks können Entwickler den Zustand in einer Komponente speichern, ohne eine Klasse erstellen zu müssen \cite{reacthooksintro, w3schoolsHooks}.

Hooks bieten eine Reihe von Funktionen wie useState, useEffect und useContext, um den Zustand und die Effekte innerhalb der Komponente zu verwalten. Sie ermöglichen auch die Verwendung von benutzerdefinierten Hooks, um spezifische Funktionalitäten in verschiedenen Komponenten zu nutzen.
\subsubsection{Serverseitiges Rendern}
Serverseitiges Rendern (SSR) ist ein Prozess, bei dem React-Komponenten auf dem Server gerendert werden, bevor sie an den Browser gesendet werden. SSR bietet viele Vorteile, darunter eine schnellere Ladezeit der Seite, bessere Suchmaschinenoptimierung und bessere Leistung auf langsamen oder mobilen Geräten \cite{reactssr1, reactssr2}.

SSR in React kann auf verschiedene Arten implementiert werden, je nach den Anforderungen der Anwendung. Die Integration von SSR erfordert jedoch in der Regel eine Reihe von Schritten, einschließlich der Konfiguration des Servers und der Anpassung der Anwendung.

\subsection{Best Practices in React-Entwicklung}
Modularität und Wiederverwendbarkeit von Komponenten ist ein wichtiger Aspekt bei der Entwicklung von React-Anwendungen. Durch die Aufteilung einer Anwendung in kleine, wiederverwendbare Komponenten kann der Code besser organisiert und leichter gewartet werden. Es ermöglicht auch die Erstellung von Komponenten-Bibliotheken, die in anderen Projekten wiederverwendet werden können \cite{dhiwise, reactcomponentlibrary}.

Um eine Komponente modular und wiederverwendbar zu gestalten, sollte sie unabhängig und gut dokumentiert sein. Die Komponente sollte spezifische Funktionen erfüllen und keine Abhängigkeiten von anderen Komponenten haben. Es ist auch wichtig, die Komponenten sauber zu benennen und ihre Funktionen klar zu dokumentieren.
\subsubsection{Testen von React-Anwendungen}
Das Testen von React-Anwendungen ist ein wichtiger Aspekt, um sicherzustellen, dass die Anwendung robust und fehlerfrei ist. Es gibt verschiedene Arten von Tests, die in React-Anwendungen durchgeführt werden können, wie Unit-Tests, Integrationstests und End-to-End-Tests \cite{reacttesting, reacthookstesting}.

Die Unit-Tests konzentrieren sich auf das Testen einer einzelnen Komponente, um sicherzustellen, dass sie wie erwartet funktioniert. Integrationstests testen die Interaktion zwischen verschiedenen Komponenten, während End-to-End-Tests die gesamte Anwendung testen.

Es ist wichtig, regelmäßig Tests durchzuführen und Testabdeckungen zu implementieren, um potenzielle Fehler zu identifizieren und zu beheben.
\subsubsection{Performance-Optimierung}
Performance-Optimierung ist ein wichtiger Aspekt bei der Entwicklung von React-Anwendungen, da eine schlechte Leistung zu einer negativen Benutzererfahrung führen kann. Es gibt verschiedene Möglichkeiten, die Performance von React-Anwendungen zu verbessern.

Eine Möglichkeit besteht darin, sicherzustellen, dass nur die notwendigen Komponenten gerendert werden. React bietet die Möglichkeit, Komponenten als pure Funktionen zu definieren, die nur von ihren Eingaben abhängen. Dadurch können unnötige Rendervorgänge vermieden werden.

Eine weitere Möglichkeit besteht darin, das Rendering von Komponenten zu optimieren. Eine Möglichkeit, dies zu tun, ist die Verwendung von Memoization. Memoization ist eine Technik, bei der das Ergebnis einer Funktion zwischengespeichert wird, wenn dieselben Eingaben erneut verwendet werden. Dadurch können unnötige Berechnungen vermieden werden.

Außerdem sollte darauf geachtet werden, dass Komponenten nicht zu viele Props erhalten, da dies die Leistung negativ beeinflussen kann. In diesem Fall sollte die Komponente möglicherweise in kleinere Komponenten aufgeteilt werden, die jeweils nur die notwendigen Props erhalten.

Eine weitere Möglichkeit, die Performance von React-Anwendungen zu verbessern, ist die Verwendung von Code-Splitting. Code-Splitting ist eine Technik, bei der der Code einer Anwendung in kleinere Stücke aufgeteilt wird, die separat geladen werden können. Dadurch kann die Ladezeit der Anwendung verringert werden (\cite{ReactOpt}, \cite{LogRocket}).
\subsubsection{Sicherheit und React}
Sicherheit ist ein wichtiger Aspekt bei der Entwicklung von React-Anwendungen, da Sicherheitslücken zu ernsthaften Problemen führen können. Es gibt verschiedene Best Practices, die bei der Entwicklung von sicheren React-Anwendungen beachtet werden sollten.

Eine Möglichkeit besteht darin, XSS-Angriffe (Cross-Site Scripting) zu vermeiden. XSS-Angriffe können dazu führen, dass schädlicher Code in die Anwendung eingeschleust wird. Um XSS-Angriffe zu vermeiden, sollten alle Daten, die von Benutzern eingegeben werden können, validiert und gefiltert werden. Darüber hinaus sollte auch das Rendering von HTML sorgfältig überwacht werden.

Eine weitere Möglichkeit, die Sicherheit von React-Anwendungen zu verbessern, besteht darin, die Codequalität zu verbessern. Eine gute Codequalität kann dazu beitragen, potenzielle Sicherheitslücken zu identifizieren und zu beheben, bevor sie ausgenutzt werden können.

Außerdem sollten bei der Entwicklung von React-Anwendungen Best Practices für die Sicherheit von Webanwendungen beachtet werden, wie z.B. die Verwendung von sicheren Passwörtern und die Vermeidung von unsicheren Bibliotheken oder Frameworks (\cite{ReactSecurity}, \cite{FreeCodeCamp}).

\subsection{Vergleich mit anderen Frontend-Technologien}
\subsubsection{Vor- und Nachteile von React im Vergleich zu Angular und Vue.js}
React, Angular und Vue.js sind beliebte Frontend-Technologien, die bei der Entwicklung von modernen Webanwendungen eingesetzt werden. Jede Technologie hat ihre Vor- und Nachteile, die bei der Wahl der Technologie berücksichtigt werden sollten.

React zeichnet sich durch seine Flexibilität und einfache Handhabung aus. Es ist eine Bibliothek und nicht ein vollständiges Framework, was bedeutet, dass Entwickler mehr Freiheit haben und nur die Teile von React verwenden können, die sie benötigen. Dadurch ist React auch in der Regel schneller und leichter als Frameworks wie Angular oder Vue.js. Außerdem ist React sehr beliebt und hat eine große und aktive Community, was zu einer Fülle von Ressourcen und Unterstützung führt (\cite{Hosttest}).

Im Vergleich zu Angular und Vue.js hat React jedoch auch einige Nachteile. Einer der Nachteile ist, dass React weniger Features und Werkzeuge bietet als Angular oder Vue.js. Dadurch müssen Entwickler möglicherweise zusätzliche Bibliotheken und Tools verwenden, um bestimmte Aufgaben zu erledigen. Ein weiterer Nachteil von React ist, dass es nicht so viele vorgefertigte UI-Komponenten wie Vue.js bietet, was die Entwicklung von Anwendungen erschweren kann (\cite{Kruschecompany}).
\subsubsection{Verwendung von React in größeren Projekten}
React eignet sich auch sehr gut für größere Projekte, da es die Möglichkeit bietet, Anwendungen in kleine, wiederverwendbare Komponenten aufzuteilen. Dadurch können Teams effizienter arbeiten und die Anwendung leichter warten und skalieren. Darüber hinaus bietet React auch die Möglichkeit, Code-Splitting und Lazy Loading zu verwenden, um die Ladezeit von Anwendungen zu reduzieren und die Leistung zu verbessern.

Es ist jedoch wichtig, dass bei größeren Projekten Best Practices für React-Anwendungen beachtet werden, wie z.B. die Verwendung von Redux für eine effektive Verwaltung des Anwendungszustands oder die Verwendung von TypeScript, um die Codequalität zu verbessern und die Entwicklungszeit zu verkürzen.
\subsubsection{Zukunft von React}
React hat sich in den letzten Jahren zu einer der beliebtesten Frontend-Technologien entwickelt und wird voraussichtlich auch in Zukunft eine wichtige Rolle spielen. Es gibt viele Indikatoren dafür, dass React weiterhin wachsen wird, wie z.B. die große und aktive Community, die kontinuierliche Entwicklung und Verbesserung der Bibliothek, sowie die Verwendung von React in vielen großen und erfolgreichen Unternehmen.

Darüber hinaus gibt es auch neue Entwicklungen wie React Native, das es Entwicklern ermöglicht, native mobile Anwendungen mit React zu entwickeln (\cite{LeanOcean}). Durch die Verwendung von React Native können Entwickler die Vorteile von React auch in der Entwicklung von mobilen Anwendungen nutzen.
\subsection{Fazit}
\subsubsection{Ausblick auf weitere Entwicklungen in React}
React hat sich in der Webentwicklung als eine der beliebtesten Frontend-Technologien etabliert und wird auch in Zukunft eine wichtige Rolle spielen. Es bietet viele Vorteile wie eine hohe Flexibilität, Leistung und eine große Community. Auch für größere Projekte eignet sich React sehr gut, da es die Möglichkeit bietet, Anwendungen in kleine, wiederverwendbare Komponenten aufzuteilen.

Die kontinuierliche Entwicklung und Verbesserung von React bleibt ein wichtiger Aspekt, um die Bedürfnisse der Benutzer zu erfüllen und wettbewerbsfähig zu bleiben. Die Entwicklergemeinschaft arbeitet ständig an neuen Funktionen und Verbesserungen, um React noch besser zu machen. Ein Beispiel dafür ist React Fiber, eine neue interne Rekonstruktionsstrategie von React, die auf eine bessere Leistung und mehr Flexibilität abzielt.

Ein weiterer wichtiger Trend in der React-Entwicklung ist die Verwendung von React Native, das es Entwicklern ermöglicht, native mobile Anwendungen mit React zu entwickeln. Durch die Verwendung von React Native können Entwickler die Vorteile von React auch in der Entwicklung von mobilen Anwendungen nutzen und so die Entwicklung von Web- und mobilen Anwendungen besser integrieren.

Insgesamt bleibt React eine der führenden Technologien in der Frontend-Entwicklung und wird voraussichtlich auch in Zukunft eine wichtige Rolle spielen. Durch die ständige Entwicklung und Verbesserung von React sowie die Integration von React Native in die Entwicklung von mobilen Anwendungen wird React auch weiterhin eine wettbewerbsfähige Technologie bleiben.


\section{Docker}

\begin{wrapfigure}{l}{0.4\textwidth}
\centering
\fbox{\includegraphics[width=0.25\textwidth,angle=270]{lion}}
\end{wrapfigure}

Lorem ipsum dolor sit amet, consetetur sadipscing elitr, sed diam nonumy eirmod tempor invidunt ut labore et dolore magna aliquyam erat, sed diam voluptua. At vero eos et accusam et justo duo dolores et ea rebum. Stet clita kasd gubergren, no sea takimata sanctus est Lorem ipsum dolor sit amet. Lorem ipsum dolor sit amet, consetetur sadipscing elitr, sed diam nonumy eirmod tempor invidunt ut labore et dolore magna aliquyam erat, sed diam voluptua. At vero eos et accusam et justo duo dolores et ea rebum. Stet clita kasd gubergren, no sea takimata sanctus est Lorem ipsum dolor sit amet. Lorem ipsum dolor sit amet, consetetur sadipscing elitr, sed diam nonumy eirmod tempor invidunt ut labore et dolore magna aliquyam erat, sed diam voluptua. At vero eos et accusam et justo duo dolores et ea rebum. Stet clita kasd gubergren, no sea takimata sanctus est Lorem ipsum dolor sit amet.

Duis autem vel eum iriure dolor in hendrerit in vulputate velit esse molestie consequat, vel illum dolore eu feugiat nulla facilisis at vero eros et accumsan et iusto odio dignissim qui blandit praesent luptatum zzril delenit augue duis dolore te feugait nulla facilisi. Lorem ipsum dolor sit amet, consectetuer adipiscing elit, sed diam nonummy nibh euismod tincidunt ut laoreet dolore magna aliquam erat volutpat.

Ut wisi enim ad minim veniam, quis nostrud exerci tation ullamcorper suscipit lobortis nisl ut aliquip ex ea commodo consequat. Duis autem vel eum iriure dolor in hendrerit in vulputate velit esse molestie consequat, vel illum dolore eu feugiat nulla facilisis at vero eros et accumsan et iusto odio dignissim qui blandit praesent luptatum zzril delenit augue duis dolore te feugait nulla facilisi.

Nam liber tempor cum soluta nobis eleifend option congue nihil imperdiet doming id quod mazim placerat facer possim assum. Lorem ipsum dolor sit amet, consectetuer adipiscing elit, sed diam nonummy nibh euismod tincidunt ut laoreet dolore magna aliquam erat volutpat. Ut wisi enim ad minim veniam, quis nostrud exerci tation ullamcorper suscipit lobortis nisl ut aliquip ex ea commodo consequat.

Duis autem vel eum iriure dolor in hendrerit in vulputate velit esse molestie consequat, vel illum dolore eu feugiat nulla facilisis.

At vero eos et accusam et justo duo dolores et ea rebum. Stet clita kasd gubergren, no sea takimata sanctus est Lorem ipsum dolor sit amet. Lorem ipsum dolor sit amet, consetetur sadipscing elitr, sed diam nonumy eirmod tempor invidunt ut labore et dolore magna aliquyam erat, sed diam voluptua. At vero eos et accusam et justo duo dolores et ea rebum. Stet clita kasd gubergren, no sea takimata sanctus est Lorem ipsum dolor sit amet. Lorem ipsum dolor sit amet, consetetur sadipscing elitr, At accusam aliquyam diam diam dolore dolores duo eirmod eos erat, et nonumy sed tempor et et invidunt justo labore Stet clita ea et gubergren, kasd magna no rebum. sanctus sea sed takimata ut vero voluptua. est Lorem ipsum dolor sit amet. Lorem ipsum dolor sit amet, consetetur sadipscing elitr, sed diam nonumy eirmod tempor invidunt ut labore et dolore magna aliquyam erat.

Consetetur sadipscing elitr, sed diam nonumy eirmod tempor invidunt ut labore et dolore magna aliquyam erat, sed diam voluptua. At vero eos et accusam et justo duo dolores et ea rebum. Stet clita kasd gubergren, no sea takimata sanctus est Lorem ipsum dolor sit amet. Lorem ipsum dolor sit amet, consetetur sadipscing elitr, sed diam nonumy eirmod tempor invidunt ut labore et dolore magna aliquyam erat, sed diam voluptua. At vero eos et accusam et justo duo dolores et ea rebum. Stet clita kasd gubergren, no sea takimata sanctus est Lorem ipsum dolor sit amet. Lorem ipsum dolor sit amet, consetetur sadipscing elitr, sed diam nonumy eirmod tempor invidunt ut labore et dolore magna aliquyam erat, sed diam voluptua. At vero eos et accusam et justo duo dolores et ea rebum. Stet clita kasd gubergren, no sea takimata sanctus.

Lorem ipsum dolor sit amet, consetetur sadipscing elitr, sed diam nonumy eirmod tempor invidunt ut labore et dolore magna aliquyam erat, sed diam voluptua. At vero eos et accusam et justo duo dolores et ea rebum. Stet clita kasd gubergren, no sea takimata sanctus est Lorem ipsum dolor sit amet. Lorem ipsum dolor sit amet, consetetur sadipscing elitr, sed diam nonumy eirmod tempor invidunt ut labore et dolore magna aliquyam erat, sed diam voluptua. At vero eos et accusam et justo duo dolores et ea rebum. Stet clita kasd gubergren, no sea takimata sanctus est Lorem ipsum dolor sit amet. Lorem ipsum dolor sit amet, consetetur sadipscing elitr, sed diam nonumy eirmod tempor invidunt ut labore et dolore magna aliquyam erat, sed diam voluptua. At vero eos et accusam et justo duo dolores et ea rebum. Stet clita kasd gubergren, no sea takimata sanctus est Lorem ipsum dolor sit amet.

Duis autem vel eum iriure dolor in hendrerit in vulputate velit esse molestie consequat, vel illum dolore eu feugiat nulla facilisis at vero eros et accumsan et iusto odio dignissim qui blandit praesent luptatum zzril delenit augue duis dolore te feugait nulla facilisi. Lorem ipsum dolor sit amet, consectetuer adipiscing elit, sed diam nonummy nibh euismod tincidunt ut laoreet dolore magna aliquam erat volutpat.

Ut wisi enim ad minim veniam, quis nostrud exerci tation ullamcorper suscipit lobortis nisl ut aliquip ex ea commodo consequat. Duis autem vel eum iriure dolor in hendrerit in vulputate velit esse molestie consequat, vel illum dolore eu feugiat nulla facilisis at vero eros et accumsan et iusto odio dignissim qui blandit praesent luptatum zzril delenit augue duis dolore te feugait nulla facilisi.

Nam liber tempor cum soluta nobis eleifend option congue nihil imperdiet doming id quod mazim
placerat facer possim assum. Lorem ipsum dolor sit amet, consectetuer adipiscing elit, sed diam
nonummy nibh euismod tincidunt ut laoreet dolore magna aliquam erat volutpat. Ut wisi enim ad minim
veniam, quis nostrud exerci tation ullamcorper suscipit lobortis nisl ut aliquip ex ea commodo.



\printbibliography
%%%%%%%%%%%%%%%%%%%%%%%%%%%%%%%%%%%%%%%%%%%%%%%%%%%%%%%%%%%%%%%%%%%%%%%%%%%%%%%
