\chapter{Einleitung}
Die fortschreitende Digitalisierung hat in den letzten Jahren weitreichende Veränderungen in der Hochschulwelt mit sich gebracht. Immer mehr Bildungseinrichtungen nutzen digitale Informationskanäle, um Studierende, Mitarbeiter und andere Interessengruppen über aktuelle Ereignisse, Veranstaltungen und wichtige Informationen auf dem Laufenden zu halten\cite{aachenerzeitung2022}. Die Duale Hochschule Baden-Württemberg (DHBW) Karlsruhe bildet da keine Ausnahme.

Mit langjähriger Erfahrung als Hochschule für angewandte Wissenschaften hat die DHBW Karlsruhe vielfältige Informationen geschaffen, um den Bedürfnissen ihrer Nutzer gerecht zu werden \cite{degruyter2021}. Bei der Bereitstellung von Informationen für eine große Anzahl von Benutzern können jedoch verschiedene Probleme auftreten.

Die Digitalisierung der Hochschulbildung stellt nicht nur akademische Inhalte in digitaler Form bereit, sondern eröffnet auch neue pädagogische Möglichkeiten zur Wissens- und Kompetenzvermittlung \cite{hochschulforumdigitalisierung}. Es ist daher Aufgabe der Hochschulleitung sowie der Konzeption konkreter Lehrveranstaltungen und Lernmaterialien, diese digitalen Möglichkeiten zu nutzen und weiterzuentwickeln \cite{springerlink2023}. 

Bei der Umsetzung dieser digitalen Transformation können jedoch auch Hindernisse auftreten, wie beispielsweise unsere natürliche Widerstandsfähigkeit gegenüber schnellen Veränderungen\cite{degruyter2021_2}. Dennoch müssen diese Herausforderungen angegangen werden, wenn wir die Chancen der Digitalisierung maximieren und die Hochschulbildung zukunftssicher machen wollen.
\\\\
\emph{Für die Rechtschreib- und Grammatikprüfung dieser Studienarbeit wird\\ https://www.scribbr.de/rechtschreibpruefung/ verwendet} 

\section{Hintergrund und Kontext}
Die Duale Hochschule Baden-Württemberg (DHBW) Karlsruhe ist eine der größten Hochschulen für angewandte Wissenschaften in Deutschland. Mit über 8.000 Studierenden und mehr als 1.000 Mitarbeitern bietet die DHBW Karlsruhe ein breites Spektrum an Studiengängen in verschiedenen Fachbereichen an. Um den Anforderungen ihrer Nutzer gerecht zu werden, hat die DHBW Karlsruhe in den letzten Jahren eine Vielzahl von Informationsangeboten entwickelt, wie z.B. die Website, die Intranet-Plattform und verschiedene soziale Medien.

Trotz dieser Informationsangebote können jedoch Herausforderungen bei der Bereitstellung von Informationen für Studierende auftreten. Zum Beispiel können Informationen nicht immer auf den ersten Blick leicht zugänglich sein oder die Kommunikation kann nicht immer effektiv gestaltet werden. Daher ist es wichtig, die Bedürfnisse der Nutzer zu verstehen und Lösungen zu finden, um die Informationsbereitstellung zu optimieren.


\section{Zieldefinition}
Das Ziel dieser Studienarbeit ist es, eine Alternative zum aktuellen Informationsangebot der DHBW Karlsruhe zu entwickeln und alle Informationen auf einer zentralen Plattform zu sammeln. Hierfür sollen die Bedürfnisse der Nutzer analysiert und die Informationsangebote der DHBW Karlsruhe auf ihre Effizienz hin überprüft werden.
Im Rahmen der Studienarbeit sollen folgende Forschungsfragen beantwortet werden:
\begin{itemize}
	\item Welche Informationen sind für die Nutzer der DHBW Karlsruhe am wichtigsten?
	\item Welche Informationsquellen sind für die Nutzer am effektivsten?
	\item Wie können die Probleme im Informationsangebot behoben werden?
	\item Welche Features sollte die neue Informationsplattform haben?
	\item Welche Möglichkeiten gibt es zur Verbesserung?
\end{itemize}
\newpage
\section{Methodik und Vorgehen}
Um die Zielsetzung der Studienarbeit zu erreichen und die gestellten Forschungsfragen zu beantworten, werden verschiedene Methoden und Techniken angewendet. Hierzu zählen die Sammlung von Erfahrungen aus Sicht der Studierenden, die Analyse von persönlichen Erfahrungsgesprächen sowie die Entwicklung eines alternativen Informationsangebotsprototyps.

Die Analyse der Erfahrungen erfolgt durch die Auswertung von Feedback-Quellen wie beispielsweise Umfragen oder persönliche Gespräche mit Studierenden. Dabei wird besonders auf wiederkehrende Probleme und Schwierigkeiten im Umgang mit den Informationsangeboten geachtet. Diese Erfahrungen sind wichtig, um die Bedürfnisse und Erwartungen der Studierenden an das Informationsangebot der DHBW Karlsruhe besser zu verstehen und mögliche Verbesserungsmöglichkeiten zu identifizieren.

Nach der Analyse der Erfahrungen wird ein alternativer Informationsangebotsprototyp entwickelt, der auf den Bedürfnissen und Erwartungen der Studierenden basiert. Der Prototyp wird an einer Gruppe von Studierenden getestet, um Feedback und Verbesserungsvorschläge zu sammeln.

Insgesamt ist die Einbeziehung der Erfahrungen und des Feedbacks von Studierenden ein wichtiger Schritt, um ein Informationsangebot zu entwickeln, das den Bedürfnissen der Nutzer entspricht und eine positive Nutzererfahrung bietet.


\section{Beitrag und Relevanz}
Die Entwicklung einer alternativen Informationsplattform für die DHBW Karlsruhe und die anschließende Evaluation der Nutzererfahrung durch eine Umfrage haben das Ziel, die Informationsbereitstellung für Studierende und andere Interessenten zu optimieren. Die Ergebnisse dieser Studienarbeit haben somit den potentiellen Nutzen, die Effektivität der Informationsbereitstellung zu erhöhen und damit die Zufriedenheit der Nutzer zu steigern.

Diese Studienarbeit leistet somit einen Beitrag zur Verbesserung des Informationsangebots an Bildungseinrichtungen und trägt damit auch zu einer erfolgreichen und zufrieden stellenden Hochschulerfahrung bei.
\newpage

\section{Aufbau der Studienarbeit}
Die Studienarbeit ist in mehrere Kapitel unterteilt, die sich mit verschiedenen Aspekten des Themas befassen. Nach der Einleitung werden in Kapitel 2 die theoretischen Grundlagen erläutert, die für die Entwicklung der App relevant sind. Kapitel 3 beschreibt die angewandten Methoden und Techniken für die Umsetzung, während in Kapitel 4 die Datenerhebung im Fokus liegt. Das Kapitel 5 befasst sich mit der Auswertung der Daten und abschließend fasst Kapitel 6 die wichtigsten Ergebnisse zusammen und gibt einen Ausblick auf zukünftige Entwicklungen.
